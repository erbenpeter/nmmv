\documentclass[a4paper,10pt]{article} 
\usepackage[utf8]{inputenc}
\usepackage[a4paper]{geometry}
\usepackage[magyar]{babel}
\usepackage{amsmath}
\usepackage{amssymb}
\frenchspacing 
\pagestyle{empty}
\newcommand{\ki}[2]{\hfill {\it #1 (#2)}\medskip}
\newcommand{\vonal}{\hbox to \hsize{\hskip2truecm\hrulefill\hskip2truecm}}
\newcommand{\degre}{\ensuremath{^\circ}}
\newcommand{\tg}{\mathop{\mathrm{tg}}\nolimits}
\newcommand{\ctg}{\mathop{\mathrm{ctg}}\nolimits}
\newcommand{\arc}{\mathop{\mathrm{arc}}\nolimits}
\begin{document}
\begin{center} \Large {\em 25. Nemzetközi Magyar Matematika Verseny} \end{center}
\begin{center} \large{\em Budapest, 2016. március 11-15.} \end{center}
\smallskip
\begin{center} \large{\bf 12. osztály} \end{center}
\bigskip 

{\bf 1. feladat: } Az $ABC$ szabályos háromszög köré írt körön a rövidebb $AB$ íven
kijelölünk egy $M$ pontot. Bizonyítsa be, hogy $AB^2\ge 3\cdot AM\cdot MB$.

\ki{Olosz Ferenc}{Szatmárnémeti}\medskip

{\bf 2. feladat: } Az ötös lottón 5 számot kell megjelölni az
$1,2,\dots,90$ számok közül. Peti egy olyan szelvénnyel játszik, amelyen az 5
megjelölt számban  az $1,2,\dots,9$ számjegyek mindegyike pontosan egyszer
szerepel, és a 0 számjegy nem fordul elő. Petinek szól a barátja, hogy
az aznapi sorsoláson ilyen 5 számot húztak ki, de magukról a kihúzott
számokról nem tud semmit sem mondani. Mi a valószínűsége annak, hogy
Petinek legalább 4 találata van?

\ki{Remeténé Orvos Viola}{Debrecen}\medskip

{\bf 3. feladat: } Igazolja, hogy
$1992\cdot2012\cdot2016\cdot2022\cdot2042+5^6$ összetett szám.

\ki{Kovács Béla}{Szatmárnémeti}\medskip

{\bf 4. feladat: } Igazolja, hogy ha a $P$ polinom minden együtthatója nemnegatív valós szám, akkor $x>0$ esetén $P(x)P(\frac1x)\geq (P(1))^2$.

\ki{Keke\v{n}ák Szilvia}{Kassa}\medskip

{\bf 5. feladat: } Egy konvex négyszög oldalainak és átlóinak
hossza racionális szám. Mutassa meg, hogy az átlókat a metszéspontjuk
racionális hosszúságú szakaszokra osztja.

\ki{Tóth Sándor}{Kisvárda}\medskip

{\bf 6. feladat: } Oldja meg az $x^3+2=5\sqrt[3]{5x-2}$ egyenletet a valós számok halmazán.

\ki{Bíró Béla}{Sepsiszentgyörgy}\medskip


\end{document}