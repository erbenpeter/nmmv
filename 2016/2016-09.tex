\documentclass[a4paper,10pt]{article} 
\usepackage[utf8]{inputenc}
\usepackage[a4paper]{geometry}
\usepackage[magyar]{babel}
\usepackage{amsmath}
\usepackage{amssymb}
\frenchspacing 
\pagestyle{empty}
\newcommand{\ki}[2]{\hfill {\it #1 (#2)}\medskip}
\newcommand{\vonal}{\hbox to \hsize{\hskip2truecm\hrulefill\hskip2truecm}}
\newcommand{\degre}{\ensuremath{^\circ}}
\newcommand{\tg}{\mathop{\mathrm{tg}}\nolimits}
\newcommand{\ctg}{\mathop{\mathrm{ctg}}\nolimits}
\newcommand{\arc}{\mathop{\mathrm{arc}}\nolimits}
\begin{document}
\begin{center} \Large {\em 25. Nemzetközi Magyar Matematika Verseny} \end{center}
\begin{center} \large{\em Budapest, 2016. március 11-15.} \end{center}
\smallskip
\begin{center} \large{\bf 9. osztály} \end{center}
\bigskip 

{\bf 1. feladat: } Nevezzünk egy számot prímösszegűnek, ha a tízes számrendszerben felírt szám számjegyeinek összege prímszám. Legfeljebb hány prímösszegű szám lehet öt egymást követő pozitív egész szám között?

\ki{Róka Sándor}{Nyíregyháza}\medskip

{\bf 2. feladat: } Melyek azok az $x$ egész számok, amelyekre $x^2+3x+24$ négyzetszám?

\ki{Szabó Magda}{Zenta--Szabadka}\medskip

{\bf 3. feladat: } Bizonyítsa be, hogy az 
\[(n^2+7n)(n^2+7n+6)(n^2+7n+10)(n^2+7n+12)\] 
kifejezés minden egész $n$ esetén osztható $2016$-tal.


\ki{Nagy Piroska Mária}{Dunakeszi}

\ki{Szoldatics József}{Budapest}\medskip

{\bf 4. feladat: } Egy egyenlő szárú háromszög alapon fekvő szögének felezője kétszer olyan hosszú, mint a szárak szögének felezője. Mekkorák a háromszög szögei?

\ki{Katz Sándor}{Bonyhád}\medskip

{\bf 5. feladat: } Adva van a síkban $2016$ olyan pont, hogy minden ponthármasból kiválasztható két pont, amelyek $1$ egységnél kisebb távolságra vannak egymástól. Bizonyítsa be, hogy létezik olyan egységsugarú kör, amely a $2016$ pont közül legalább $1008$-at tartalmaz.

\ki{Bálint Béla}{Zsolna}\medskip

{\bf 6. feladat: } Az $x_1, x_2, x_3, \ldots, x_n$ számok mindegyikének értéke $+1$ vagy $-1$. 
Bizonyítsa be, hogy ha  
\[x_1x_2x_3x_4+x_2x_3x_4x_5+x_3x_4x_5x_6+\ldots+x_{n-1}x_nx_1x_2+x_nx_1x_2x_3=0,\] 
akkor az $n$ szám $4$-gyel osztható.

\ki{Keke\v{n}ák Szilvia}{Kassa}\medskip


\end{document}