\documentclass[a4paper,10pt]{article} 
\usepackage[utf8]{inputenc}
\usepackage[a4paper]{geometry}
\usepackage[magyar]{babel}
\usepackage{amsmath}
\usepackage{amssymb}
\frenchspacing 
\pagestyle{empty}
\newcommand{\ki}[2]{\hfill {\it #1 (#2)}\medskip}
\newcommand{\vonal}{\hbox to \hsize{\hskip2truecm\hrulefill\hskip2truecm}}
\newcommand{\degre}{\ensuremath{^\circ}}
\newcommand{\tg}{\mathop{\mathrm{tg}}\nolimits}
\newcommand{\ctg}{\mathop{\mathrm{ctg}}\nolimits}
\newcommand{\arc}{\mathop{\mathrm{arc}}\nolimits}
\begin{document}
\begin{center} \Large {\em 26. Nemzetközi Magyar Matematika Verseny} \end{center}
\begin{center} \large{\em Somorja, 2017. március 23-27.} \end{center}
\smallskip
\begin{center} \large{\bf 10. osztály} \end{center}
\bigskip 

{\bf 1. feladat: } Adott a síkon 2017 (különböző) pont úgy, hogy bármely 3 közül kiválasztható 2, melyek távolsága 1-nél
kisebb. Bizonyítsátok be, hogy a 2017 pont között található 1009 olyan, amelyek egységsugarú körben lesznek.

\ki{Mészáros József}{Jóka}\medskip

{\bf 2. feladat: } Egy derékszögű trapéz egyik alapja 5~cm, a másik alap és a derékszögű szár összege 10~ cm. Mekkora lehet a
trapéz területének legnagyobb értéke? Mekkora lehet a trapéz kerületének legkisebb értéke?

\ki{Dr. Katz Sándor}{Bonyhád}\medskip

{\bf 3. feladat: } Oldjátok meg a következő egyenletet a valós számok halmazán:
$$\frac{6}{\sqrt{x-2017}-9}+
\frac{1}{\sqrt{x-2017}-4}+
\frac{7}{\sqrt{x-2017}+4}+
\frac{12}{\sqrt{x-2017}+9}=0.
$$

\ki{Nemecskó István}{Budapest}\medskip

{\bf 4. feladat: } Van-e 1000 olyan egymást követő egész szám, melyek között pontosan 5 prímszám van?

\ki{Róka Sándor}{Nyíregyháza}\medskip

{\bf 5. feladat: } Az $ABC$ háromszögben
$CAB\sphericalangle=40^\circ$
és
$CBA\sphericalangle=80^\circ$, a háromszög körülírt körének középpontja az $O$ pont.
$CBA\sphericalangle$ szögfelezője a $D$
pontban metszi az $AC$
oldalt. Bizonyítsátok be, hogy a $DOB$
háromszög körülírt
körének középpontja az $ABC$
háromszög $C$ pontból induló magasságvonalára illeszkedik!

\ki{Bíró Bálint}{Eger}\medskip

{\bf 6. feladat: } A valós számok halmazán oldjátok meg az $x^4+3x^3+px^2+3x+1=0$ egyenletet 
$p=3{,}25$ esetén! A $p$
paraméter mely értékei esetén lesz az 
$x^4+3x^3+px^2+3x+1=0$
egyenletnek négy különböző valós gyöke?

\ki{Dr. Katz Sándor}{Bonyhád}\medskip


\end{document}