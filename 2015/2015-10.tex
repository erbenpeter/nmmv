\documentclass[a4paper,10pt]{article} 
\usepackage[utf8]{inputenc}
\usepackage[a4paper]{geometry}
\usepackage[magyar]{babel}
\usepackage{amsmath}
\usepackage{amssymb}
\frenchspacing 
\pagestyle{empty}
\newcommand{\ki}[2]{\hfill {\it #1 (#2)}\medskip}
\newcommand{\vonal}{\hbox to \hsize{\hskip2truecm\hrulefill\hskip2truecm}}
\newcommand{\degre}{\ensuremath{^\circ}}
\newcommand{\tg}{\mathop{\mathrm{tg}}\nolimits}
\newcommand{\ctg}{\mathop{\mathrm{ctg}}\nolimits}
\newcommand{\arc}{\mathop{\mathrm{arc}}\nolimits}
\begin{document}
\begin{center} \Large {\em 24. Nemzetközi Magyar Matematika Verseny} \end{center}
\begin{center} \large{\em Szabadka, 2015. április 8-12.} \end{center}
\smallskip
\begin{center} \large{\bf 10. osztály} \end{center}
\bigskip 

{\bf 1. feladat: } A XXIV. Nemzetközi Magyar Matematika Verseny tiszteletére Frici rajzolt Szabadka főterére egy 24 oldalú szabályos sokszöget. Hány olyan egyenlő szárú háromszöget rajzolhatna, amelynek minden csúcsa ennek a sokszögnek egy csúcsa, és minden oldala ennek a sokszögnek egy átlója?

\ki{Erdős Gábor}{Nagykanizsa, Magyarország}\medskip

{\bf 2. feladat: } Ha $x,y,z\in[-3, 5]$, akkor igazold, hogy
$$\sqrt{5x-3y-xy+15}+
\sqrt{5y-3z-yz+15}+\sqrt{5z-3x-xz+15}\le 12.$$
Mikor állhat fenn az egyenlőség?

\ki{Kovács Béla}{Szatmárnémeti, Erdély}\medskip

{\bf 3. feladat: } Hány olyan egyenlőszárú trapéz létezik, amelynek a kerülete 2015 és az oldalak mérőszáma egész szám?

\ki{Szabó Magda}{Szabadka, Vajdaság}\medskip

{\bf 4. feladat: } Határozd meg mindazokat az $a$ valós számokat, melyekre az 
$$ax^2+(1-a^2)x-a>0$$
egyenlőtlenség egyetlen $x$ megoldására sem igaz, hogy $|x|>2$.

\ki{Csikós Pajor Gizella}{Szabadka, Vajdaság}\medskip

{\bf 5. feladat: } Oldd meg a következő egyenletet a valós számok halmazán:
$$\left|2x-57-2\cdot\sqrt{x-55}+\frac{1}{x-54-2\cdot\sqrt{x-55}}\right|=|1-x|.$$

\ki{Bíró Bálint}{Eger, Magyarország}\medskip

{\bf 6. feladat: } Egy konvex négyszöget átlói négy háromszögre bontanak. Ha mind a négy háromszög területének a mértéke egész szám, akkor végződhet-e 2015-re a négy terület mértékének szorzata? Lehet-e ez a szorzat olyan egész szám, amelynek utolsó négy jegye 2015, azaz lehet-e 
$t_1\cdot t_2\cdot t_3 \cdot t_4 = \overline{\ldots2015}$, ha $t_1, t_2, t_3, t_4$ jelöli a háromszögek területeinek mértékét?

\ki{Katz Sándor}{Bonyhád, Magyarország}\medskip


\end{document}