\documentclass[a4paper,10pt]{article} 
\usepackage[utf8]{inputenc}
\usepackage[a4paper]{geometry}
\usepackage[magyar]{babel}
\usepackage{amsmath}
\usepackage{amssymb}
\frenchspacing 
\pagestyle{empty}
\newcommand{\ki}[2]{\hfill {\it #1 (#2)}\medskip}
\newcommand{\vonal}{\hbox to \hsize{\hskip2truecm\hrulefill\hskip2truecm}}
\newcommand{\degre}{\ensuremath{^\circ}}
\newcommand{\tg}{\mathop{\mathrm{tg}}\nolimits}
\newcommand{\ctg}{\mathop{\mathrm{ctg}}\nolimits}
\newcommand{\arc}{\mathop{\mathrm{arc}}\nolimits}
\begin{document}
\begin{center} \Large {\em 24. Nemzetközi Magyar Matematika Verseny} \end{center}
\begin{center} \large{\em Szabadka, 2015. április 8-12.} \end{center}
\smallskip
\begin{center} \large{\bf 12. osztály} \end{center}
\bigskip 

{\bf 1. feladat: } A szabályos hatoldalú csonka gúla alapélei $a$ és $b$ ( $a > b$ ). A csonka gúla
oldalfelülete megegyezik az alaplapok területének összegével. Határozd meg a csonka gúla magasságát!

\ki{Angyal Andor}{Szabadka, Vajdaság}\medskip

{\bf 2. feladat: } Egy $9 \times 9$-es négyzetrácsba beírtuk a számokat 1-től 81-ig. Bizonyítsd be,
hogy a számok bármely elrendeződése mellett van két olyan szomszédos négyzet,
amelyben a számok közötti különbség legalább 6. (Szomszédosnak tekintjük
azokat a négyzeteket, amelyeknek közös oldaluk van.)

\ki{Béres Zoltán}{Szabadka, Vajdaság}\medskip

{\bf 3. feladat: } Ha $\alpha$ hegyesszög, akkor bizonyítsd be, hogy teljesül az
$$\left(1+\frac{1}{\sin \alpha}\right)\cdot\left(1+\frac{1}{\cos \alpha}\right)\ge 3+2\sqrt 2$$
egyenlőtlenség!

\ki{Csikós Pajor Gizella}{Szabadka, Vajdaság}\medskip

{\bf 4. feladat: } Oldd meg a következő egyenletet a valós számok halmazán:
$$(2x+1)\sqrt{(2x-1)^3}+16x^4=2x(4x-1).$$

\ki{Olosz Ferenc}{Szatmárnémeti, Erdély}\medskip

{\bf 5. feladat: } Oldd meg a következő egyenletet a valós számok halmazán:
$$2x^2+\sqrt{2}+\log_2^2\left(2x^2+\sqrt 2\right)=2^{\frac{\sqrt{x^2+1}}{x^2+2}}+\frac{x^2+1}{(x^2+2)^2}$$
\ki{Bence Mihály}{Brassó, Erdély}\medskip

{\bf 6. feladat: } Egy $\overline{AB}=42$~cm és egy $\overline{CD}=58$~cm hosszú szakasz $\alpha$ szög alatt metszi
egymást az $O$ pontban. Mekkora a szakaszok végpontjaival (mint csúcsokkal)
alkotott $ACBD$ négyszög pontos területe, ha tudjuk, hogy 
$\tg\dfrac{\alpha}{2}=\dfrac{3}{7}$?

\ki{Gecse Frigyes}{Kisvárda, Magyarország}\medskip


\end{document}