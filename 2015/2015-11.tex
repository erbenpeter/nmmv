\documentclass[a4paper,10pt]{article} 
\usepackage[utf8]{inputenc}
\usepackage[a4paper]{geometry}
\usepackage[magyar]{babel}
\usepackage{amsmath}
\usepackage{amssymb}
\frenchspacing 
\pagestyle{empty}
\newcommand{\ki}[2]{\hfill {\it #1 (#2)}\medskip}
\newcommand{\vonal}{\hbox to \hsize{\hskip2truecm\hrulefill\hskip2truecm}}
\newcommand{\degre}{\ensuremath{^\circ}}
\newcommand{\tg}{\mathop{\mathrm{tg}}\nolimits}
\newcommand{\ctg}{\mathop{\mathrm{ctg}}\nolimits}
\newcommand{\arc}{\mathop{\mathrm{arc}}\nolimits}
\begin{document}
\begin{center} \Large {\em 24. Nemzetközi Magyar Matematika Verseny} \end{center}
\begin{center} \large{\em Szabadka, 2015. április 8-12.} \end{center}
\smallskip
\begin{center} \large{\bf 11. osztály} \end{center}
\bigskip 

{\bf 1. feladat: } Legyen $P(x)$ egész együtthatós polinom. Tudjuk, hogy a $P(x)$ polinom helyettesítési értéke 2015 különböző egész értékre 2014-et ad eredményül. Bizonyítsd be, hogy nincs olyan $x_0$ egész szám, amelyre $P(x_0)=2016$ teljesül!

\ki{Kántor Sándor}{Debrecen, Magyarország}\medskip

{\bf 2. feladat: } A hegyesszögű $ABC$ háromszögben legyen $D$ pont a $C$ csúcsból húzott magasság talppontja úgy, hogy $AD=BC$ érvényes. Ha $L$ pont a $D$ pontból húzott merőleges talppontja az $A$ csúcsból szerkesztett magasságra, akkor igazold, hogy a $BL$ az $ABC\sphericalangle$ szögfelezője!

\ki{Ripcó Sipos Elvira}{Zenta, Vajdaság}\medskip

{\bf 3. feladat: } A Mesebeli Órán a beosztások nem 1-től 12-ig, hanem 1-től 2015-ig vannak jelölve. A Furfangos Manók azt a játékot játszák, hogy eltüntetik az Óráról az 1-es számot, a 16-ost, 31-est, majd így sorban minden 15-ik beosztáshoz tartozó számot. Amikor olyan helyre érkeznek, amelyikről már eltüntették a számot, oda visszavarázsolják az eredeti számot, ami ott állt. Melyik lesz az első olyan szám, amelyet visszavarázsolnak a Furfangos Manók, hány kört kell addig megtenniük és hány szám látható abban a pillanatban a Mesebeli Óra beosztásainál? 

\ki{Péics Hajnalka}{Szabadka, Vajdaság}\medskip

{\bf 4. feladat: } Hány megoldása van az $x=2015\sin x$ egyenletnek?


\ki{Mikó István}{Felvidék}\medskip

{\bf 5. feladat: }  Legyen $K_n$ az $1,2,\ldots,n$ számok ($n\in\mathbb{Z}^+$) legkisebb közös többszöröse, pl. $K_1=1$, $K_2=2$, $K_3=6$, $K_4=12$, $K_5=60$, $K_6=60$, és így tovább. Mely pozitív egész számokra teljesül, hogy $K_{n-1}=K_n$? Fogalmazd meg a sejtést és bizonyítsd be az állítást!			

\ki{Kántor Sándorné}{Debrecen, Magyarország}\medskip

{\bf 6. feladat: } Egy 3~cm sugarú kör érinti egy 16~cm magasságú húrtrapéz mindkét szárát és a rövidebb alapját. A trapéz átlói illeszkednek a kör középpontjára. Mekkora a trapéz területe?

\ki{Katz Sándor}{Bonyhád, Magyarország}\medskip


\end{document}