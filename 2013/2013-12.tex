\documentclass[a4paper,10pt]{article} 
\usepackage[utf8]{inputenc}
\usepackage[a4paper]{geometry}
\usepackage[magyar]{babel}
\usepackage{amsmath}
\usepackage{amssymb}
\frenchspacing 
\pagestyle{empty}
\newcommand{\ki}[2]{\hfill {\it #1 (#2)}\medskip}
\newcommand{\vonal}{\hbox to \hsize{\hskip2truecm\hrulefill\hskip2truecm}}
\newcommand{\degre}{\ensuremath{^\circ}}
\newcommand{\tg}{\mathop{\mathrm{tg}}\nolimits}
\newcommand{\ctg}{\mathop{\mathrm{ctg}}\nolimits}
\newcommand{\arc}{\mathop{\mathrm{arc}}\nolimits}
\begin{document}
\begin{center} \Large {\em XXII. Nemzetközi Magyar Matematika Verseny} \end{center}
\begin{center} \large{\em Győr, 2013. március 14--18.} \end{center}
\smallskip
\begin{center} \large{\bf 12. osztály} \end{center}
\bigskip 

{\bf 1. feladat: } Határozza meg az összes $k$ egész számot úgy, hogy az $E = k\cdot 3^{2013}-2012$ osztható legyen 11-gyel.


\ki{Olosz Ferenc}{Erdély}\medskip

{\bf 2. feladat: } Igazolja, hogy az
$\displaystyle{y=\frac{5}{3}x+1}$
 egyenletű egyenestől minden rácspont (olyan pont, amelynek mindkét koordinátája egész szám) 
 $\displaystyle{\frac{1}{6}}$-nál távolabb van.


\ki{Dr. Kántor Sándor}{Magyarország}\medskip

{\bf 3. feladat: } Hány olyan háromszög van, amelynek oldalai centiméterben mérve egész számok és a
területe 24 cm$^2$? Határozza meg ezeket a háromszögeket.


\ki{Kallós Béla}{Magyarország}\medskip

{\bf 4. feladat: } A kúpba gömböt, majd ebbe a gömbbe kúpot szerkesztünk, amely hasonló az elsőhöz, a
tengelyes metszetek szárszögei egyenlők, a nagyobb kúp térfogata 27-szer nagyobb a
kisebb kúp térfogatánál. Határozza meg a kisebb kúp magasságának és sugarának az
arányát.


\ki{R. Sipos Elvira}{Délvidék}\medskip

{\bf 5. feladat: } Van $n$ városunk ($n$ egynél nagyobb egész szám) úgy, hogy közülük bármely kettőt
egyirányú vasútvonal köt össze. Bizonyítsa be, hogy a városok közt van olyan, ahonnan
bármelyik városba legfeljebb egy átszállással (ezen a vasúthálózaton) el lehet jutni.


\ki{Dr. Kántor Sándor}{Magyarország}\medskip

{\bf 6. feladat: } Legyen $x_1$
pozitív, 1-nél kisebb szám. Képezzük az
$$x_{k+1}=x_k-x_k^2\quad (k=1,2,\ldots)\ \text{sorozatot.}$$
Bizonyítsa, hogy minden pozitív $n$ egész számra
$$x_1^2+x_2^2+\ldots+x_n^2<\frac{2014}{2013}.$$

\ki{Oláh György}{Felvidék}\medskip


\end{document}