\documentclass[a4paper,10pt]{article} 
\usepackage[utf8]{inputenc}
\usepackage[a4paper]{geometry}
\usepackage[magyar]{babel}
\usepackage{amsmath}
\usepackage{amssymb}
\frenchspacing 
\pagestyle{empty}
\newcommand{\ki}[2]{\hfill {\it #1 (#2)}\medskip}
\newcommand{\degre}{\ensuremath{^\circ}}
\newcommand{\tg}{\mathop{\mathrm{tg}}\nolimits}
\newcommand{\ctg}{\mathop{\mathrm{ctg}}\nolimits}
\newcommand{\arc}{\mathop{\mathrm{arc}}\nolimits}
\begin{document}
\begin{center} \Large {\em XXII. Nemzetközi Magyar Matematikaverseny} \end{center}
\begin{center} \large{\em Győr, 2013. március 14--18.} \end{center}
\smallskip
\begin{center} \large{\bf 11. osztály} \end{center}
\bigskip 

{\bf 1. feladat: } Oldja meg a valós számok halmazán a következő egyenletet:
\[x+\frac{x}{x-1}+\frac{x^2}{x^2-x+1}=\frac{49}{6}\]

\ki{Kovács Béla}{Erdély}\medskip

{\bf 2. feladat: } Az $x$, $y$, $z$ valós számok eleget tesznek az
\[x^2+3y^2+z^2=2\]
egyenletnek. Mekkora lehet a $2x+y-z$ kifejezés legnagyobb értéke és mely $x$, $y$, $z$
számokra veszi ezt fel?

\ki{Pintér Ferenc}{Magyarország}\medskip

{\bf 3. feladat: } Mutassa meg, hogy a következő egyenletnek nincs megoldása az $(x; y)$ pozitív egész számpárok halmazán:
\[(3x+3y)^2+12x+12y=8048+(x-y)^2.\]

\ki{Nemecskó István}{Magyarország}\medskip

{\bf 4. feladat: } Tekintsünk egy $5\times5$-ös méretű ,,sakktáblát''. Jelentse az adott sakktábla egy kitöltését az az eljárás, melynek során minden mezőbe beírunk pontosan egyet az 1, 2, 3, \ldots{} 25 számok közül. Adja meg a fenti sakktáblának egy olyan kitöltését, melyben a számok soronkénti összegeinek szorzata a lehető legnagyobb.

\ki{Bíró Béla}{Erdély}\medskip

{\bf 5. feladat: } Legyen az $ABC$ háromszög $AB$ oldalának belső pontja $P$. Az $AC$ egyenest az $A$ pontban érintő, illetve a $BC$ egyenest a $B$ pontban érintő körök metszéspontjai a $P$ és $Q$ pontok. Bizonyítsa, hogy a $C$ pontnak az $AB$ szakasz felezőmerőlegesére vonatkozó tükörképe illeszkedik a $PQ$ egyenesre!

\ki{Bíró Bálint}{Magyarország}\medskip

{\bf 6. feladat: } Jelöljük az $ABC$ háromszög szögeit $\alpha$, $\beta$, $\gamma$-val, az említett szögekkel szemközti oldalakat pedig rendre $a$, $b$, $c$-vel. Bizonyítsa be, hogy $b<\frac{1}{2}(a+c)$ esetén $\beta<\frac{1}{2}(\alpha+\gamma)$.

\ki{Fonyó Lajos}{Magyarország}
\end{document}