\documentclass[a4paper,10pt]{article} 
\usepackage[utf8]{inputenc}
\usepackage{t1enc}
\usepackage{graphicx}
\usepackage{amssymb}
\voffset - 20pt
\hoffset - 35pt
\textwidth 450pt
\textheight 650pt 
\frenchspacing 

\pagestyle{empty}
\def\ki#1#2{\hfill {\it #1 (#2)}\medskip}

\def\tg{\, \hbox{tg} \,}
\def\ctg{\, \hbox{ctg} \,}
\def\arctg{\, \hbox{arctg} \,}
\def\arcctg{\, \hbox{arcctg} \,}

\begin{document}
\begin{center} \Large {\em XVIII. Nemzetközi Magyar Matematika Verseny} \end{center}
\begin{center} \large{\em Gyula, 2009. március 12-16.} \end{center}
\smallskip
\begin{center} \large{\bf 10. osztály} \end{center}
\bigskip 

{\bf 1. feladat: }
Egy háromszög belsejében felvett tetszőleges ponton át a háromszög
oldalaival pár\-hu\-za\-mo\-san egyeneseket húzunk. Ezek az egyenesek a háromszög területét hat
részre osztják. A keletkezett háromszögek területeit jelöljük $t_1, t_2$ és $t_3$-mal és az eredeti
háromszög területét pedig $T$-vel.

Bizonyítsuk be, hogy $ \sqrt{T}=\sqrt{t_1}+\sqrt{t_2}+\sqrt{t_3}$

\ki{Oláh György }{Komárom}\medskip

{\bf 2. feladat: } 
Az $f$ függvény értelmezési tartománya a 0-tól különböző valós számok
halmaza. Az értelmezési tartomány minden $x$ elemére teljesül az $ \displaystyle{f(x)+2f \Big(\frac 1x\Big)=3}x $
összefüggés. Mely $x$ valós számokra áll fenn az $f(x)=f(-x)$ egyenlőség?


\ki{Kántor Sándorné }{Debrecen}\medskip

{\bf 3. feladat: } 
Legyen $p\leq 3$ egy adott prímszám. Oldjuk meg az egész számok halmazán az
$$ x^3+y^3=x^2y+xy^2+p^{2009} $$
egyenletet!

\ki{Bencze Mihály }{Brassó}\medskip

{\bf 4. feladat: }
Oldjuk meg a pozitív valós számok halmazán a következő egyenletrendszert!
\begin{eqnarray}
x+y+z & = & 9 \\
\frac 1x+\frac 1{y+1}+\frac 1{z+3} & = & \frac 9{13} 
\end{eqnarray}

\ki{Kovács Béla}{Szatmárnémeti}\medskip

{\bf 5. feladat: }
Jelölje $H$ az $ABC$ háromszög magasságpontját, $O$ pedig a köré írt körének
középpontját. Az $A$ csúcsból a $BC$ egyenesre bocsájtott merőleges talppontja rajta van az $AC$
oldal felező merőlegesén. Határozzuk meg a $\displaystyle{\frac {CH}{BO}}$
arányt!
 
\ki{Sipos Elvira}{Zenta}\medskip

{\bf 6. feladat: } 
Legalább hány számot kell kihúznunk az $1, 2, 3,\dots , 2009 $ számok közül
ahhoz, hogy a megmaradó számok egyike se legyen két másik, tőle különböző megmaradó
szám szorzata?

\ki{Katz Sándor}{Bonyhád}\medskip

\vfill
\end{document}
