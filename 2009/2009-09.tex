\documentclass[a4paper,10pt]{article} 
\usepackage[utf8]{inputenc}
\usepackage{t1enc}
\usepackage{graphicx}
\usepackage{amssymb}
\usepackage{pstricks,pstricks-add}
\voffset - 20pt
\hoffset - 35pt
\textwidth 450pt
\textheight 650pt 
\frenchspacing 

\pagestyle{empty}
\def\ki#1#2{\hfill {\it #1 (#2)}\medskip}

\def\tg{\, \hbox{tg} \,}
\def\ctg{\, \hbox{ctg} \,}
\def\arctg{\, \hbox{arctg} \,}
\def\arcctg{\, \hbox{arcctg} \,}

\begin{document}
\begin{center} \Large {\em XVIII. Nemzetközi Magyar Matematika Verseny} \end{center}
\begin{center} \large{\em Gyula, 2009. március 12-16.} \end{center}
\smallskip
\begin{center} \large{\bf 9. osztály} \end{center}
\bigskip 

{\bf 1. feladat: } Oldjuk meg a természetes számok halmazán az $\displaystyle{\frac 1x+\frac 1y=\frac 1{2009}}$
egyenletet!

\smallskip
\ki{Kántor Sándor }{Debrecen}\medskip

{\bf 2. feladat: } 
Az $ABCD$ deltoidban az $A$ és $C$ csúcsnál derékszög van, és a $BD$ átló 12
cm. Az ábra szerint a deltoidba három azonos oldalhosszúságú rombusz
írható. Mekkora a deltoid $B$ és $D$ csúcsánál levő szöge és az $AC$ átló
hossza?

\centerline{
\psset{xunit=0.6cm,yunit=0.6cm,algebraic=true,dotstyle=o,dotsize=3pt 0,linewidth=0.8pt,arrowsize=3pt 2,arrowinset=0.25}
\begin{pspicture*}(2.73,-5.14)(7.95,4.83)
\psline(5.18,4.1)(5.28,-4.2)
\psline(5.18,4.1)(3.2,-3.67)
\psline[linewidth=2pt](3.2,-3.67)(5.28,-4.2)
\psline[linewidth=2pt](5.28,-4.2)(7.35,-3.62)
\psline(7.35,-3.62)(5.18,4.1)
\psline[linewidth=2pt](5.27,-3.09)(3.2,-3.67)
\psline[linewidth=2pt](3.73,-1.59)(5.27,-3.09)
\psline[linewidth=2pt](5.27,-3.09)(7.35,-3.62)
\psline[linewidth=2pt](5.27,-3.09)(6.77,-1.55)
\psline[linewidth=2pt](3.73,-1.59)(5.23,-0.05)
\psline[linewidth=2pt](5.23,-0.05)(6.77,-1.55)
\psline[linewidth=2pt](4.65,2.02)(5.23,-0.05)
\psline[linewidth=2pt](5.23,-0.05)(5.76,2.03)
\psline[linewidth=2pt](5.18,4.1)(4.65,2.02)
\psline[linewidth=2pt](5.18,4.1)(5.76,2.03)
\psline(3.2,-3.67)(7.35,-3.62)
\psdots[dotstyle=*](5.18,4.1)
\rput[bl](5.28,4.24){$B$}
\psdots[dotstyle=*](5.28,-4.2)
\rput[bl](5.35,-4.68){$D$}
\psdots[dotstyle=*,linecolor=darkgray](3.2,-3.67)
\psdots[dotstyle=*,linecolor=darkgray](7.35,-3.62)
\psdots[dotstyle=*,linecolor=darkgray](5.28,-4.2)
\psdots[dotstyle=*](3.2,-3.67)
\rput[bl](2.83,-3.53){$C$}
\psdots[dotstyle=*](7.35,-3.62)
\rput[bl](7.44,-3.48){$A$}
\end{pspicture*}
}

\ki{Katz Sándor}{Bonyhád}\medskip

{\bf 3. feladat: } 
Adjuk meg az összes olyan $n$ természetes számot, amelyre $ 2^8+2^{11}+2^n$ négyzetszám!


\ki{Eigel Ernő}{Gyula}\medskip

{\bf 4. feladat: }
Oldjuk meg az
$$ \frac {x}{x+1}+\frac{2x}{(x+1)(2x+1)}+\frac{3x}{(x+1)(2x+1)(3x+1)}+\dots+\frac{2009x}{(x+1)(2x+1)\dots(2009x+1)}>1$$
egyenlőtlenséget a valós számok halmazán!

\ki{Balázsi Borbála}{Beregszász}\medskip

{\bf 5. feladat: }
Húsz személy mindegyike a húszból tíz másiknak küld levelet. Van-e két
olyan személy, akik között volt levélváltás?
 
\ki{Szabó Magdolna}{Szabadka}\medskip

{\bf 6. feladat: } 
Az $ABCD$ téglalap $DC$ oldala, mint átmérő fölé ( átmérőre ) kört rajzolunk.
Húzzunk a körhöz a téglalap $A$ csúcsából az $AD$ egyenesétől különböző érintőt, az érintési
pont legyen $E$. A téglalap $BC$ oldalegyenesét az $AE$ egyenes a $G$ pontban, a $DE$ egyenes a $H$
pontban metszi.

a.) Bizonyítsuk be, hogy az $EGH$ háromszög egyenlő szárú!

b.) Mekkora a téglalap oldalainak aránya, ha az $EGH$ háromszög szabályos?

c.) Bizonyítsuk be, hogy ha az $EGH$ háromszög szabályos, akkor a kör $F$ középpontja, az $E$
érintési pont és a téglalap $B$ csúcsa egy egyenesen van!


\ki{Nemecskó István}{Budapest}\medskip

\vfill
\end{document}
