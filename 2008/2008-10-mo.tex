\documentclass[a4paper,10pt]{article} 
\usepackage[utf8]{inputenc}
\usepackage{t1enc}
\usepackage{graphicx}
\usepackage{amssymb}
\usepackage{pstricks, pstricks-add}
\voffset - 20pt
\hoffset - 35pt
\textwidth 450pt
\textheight 650pt 
\frenchspacing 

\pagestyle{empty}
\def\ki#1#2{\hfill {\it #1 (#2)}\medskip}

\def\tg{\, \hbox{tg} \,}
\def\ctg{\, \hbox{ctg} \,}
\def\arctg{\, \hbox{arctg} \,}
\def\arcctg{\, \hbox{arcctg} \,}

\begin{document}
\begin{center} \Large {\em XVII. Nemzetközi Magyar Matematika Verseny} \end{center}
\begin{center} \large{\em Kassa, 2008. március 6-9.} \end{center}
\smallskip
\begin{center} \large{\bf 10. osztály} \end{center}
\bigskip 

{\bf 1. feladat: } Osztható-e $20^{2008}+16^{2008}-3^{2008}-1$  \  $323$-al? Az állításodat igazold!

\ki{Oláh György}{Komárom}

\medskip
{\bf 1. feladat megoldása: } Megmutatjuk, hogy az adott szám osztható 323-al. A bizonyítást két
részben végezzük el. A 323 felbontható két prímszám, a 17 és a 19
szorzatára. Ha bebizonyítjuk, hogy a kifejezés mindkét számmal
osztható, akkor megoldottuk a feladatot.
Sőt ennél többet is: a 2008 helyett bármely páros hatványra is
be tudjuk látni az oszthatóságot.

Első lépésben nézzük meg a 17-el való oszthatóságot. 

Ehhez
csoportosítsuk át a tagokat így: $\left(20^{2k}-3^{2k}\right)+\left(16^{2k}-1\right)$.
Ekkor ismert azonosság alapján mindkét zárójelből ki tudjuk emelni a
17-et. Az első zárójelből kiemelhető a $20-3= 17$, a másikból pedig a
$16^2-1= (16-1)\cdot(16+1)$. Ezzel a 17-el való oszthatóságot beláttuk.

Most ismét átcsoportosítjuk a tagokat a 19
kiemeléséhez: $\left(20^{2k}-1\right)+\left(16^{2k}-3^{2k}\right)$. Az előbbiek szerint az első tagból
kiemelhető $20-1=19$, a másodikból pedig a $16+3=19$. Ezzel a 19-el
való oszthatóságot beláttuk.



\medskip
{\bf 2. feladat: } Az $ABC$ háromszögben $|AB|=20e$, $|AC|=16e$  és 
$|BC|=12e$. Egy $P$ középpontú és $2e$ sugarú kör végig gurul az $ABC$ háromszög belsejében úgy, hogy mindig érinti a háromszögnek legalább az egyik oldalát. Mekkora utat tesz meg $P$ addig, amíg először tér vissza a kiindulási helyzetébe? 

\ki{Dr. Kántor Sándorné}{Debrecen}\medskip

{\bf 2. feladat megoldása: } Az $ABC$ háromszög derékszögű, mert oldalainak hosszai pitagoraszi
számhármast alkotnak és a derékszög a $C$ csúcsban van.
Legyen a $BC$ befogó függőleges helyzetű (csak az egyszerűbb
kifejezés lehetősége miatt). Tekintsük a guruló körnek azt a helyzetét,
amikor a legfelső helyzetben érinti az $AB$ átfogót és $BC$ befogót.
Jelöljük ekkor $O$-val a kör középpontját és $B’$-vel az $O$-ból a $BC$-re
emelt merőleges talppontját, vagyis az érintési pontot.

Nyilván $OB’ = 2$, $ABC\sphericalangle=2\cdot OBB\sphericalangle = 2\beta$.
$$
BB'= 2 \cdot \ctg \beta = 
2 \cdot \frac{\cos \beta}{\sin \beta}=
2 \cdot \frac{\sqrt{(1+\cos 2\beta)/2}}{\sqrt{(1-\cos 2\beta)/2}}=
2 \cdot \frac{\sqrt{1+12/20}}{\sqrt{1-12/20}}=
2 \cdot \frac{\sqrt{32}}{\sqrt{8}}=
4.$$

Így a $P$ pont által befutott függőleges távolság: $12 - 4 - 2 = 6$, ami
éppen a fele az $ABC$ háromszög $BC$ oldala hosszának. Az $ABC$
háromszög és a $P$ pont által befutott háromszög alakú pálya
hasonlósága miatt, mivel a hasonlóság aránya $1:2$-höz, $P$ pályájának a
hossza az $ABC$ háromszög kerületének a fele, vagyis: 24.



{\bf 3. feladat: } Egy nagy táblázat ,,közepére'' beírjuk az 1-et, majd az ábrán látható módon ,,csiga\-vo\-nal\-ban'' folytonosan beírjuk az egymást követő egész számokat. Mivel egyenlő a közvetlenül 2008 felett, illetve alatt álló két szám összege?

\centerline{
\begin{tabular}{|c|c|c|c|c|c|}
\hline
17 & 16 & 15 & 14 & 13 &  \\ \hline
$\downarrow $ & 5 & 4 & 3 & 12 &  \\ \hline
 & 6 & 1 & 2 & 11 &  \\ \hline
 & 7 & 8 & 9 & 10 & $\uparrow$ \\ \hline
 &  &  &  &  &  \\ \hline
\end{tabular}}


\ki{Kiss Sándor}{Nyíregyháza}\medskip


{\bf 3. feladat megoldása: } Folytassuk a táblázat kitöltését és keressünk szabályszerűséget!
Látható, hogy a páratlan négyzetszámok egy ,,félátlóban''
helyezkednek el. 

A 2008 a $43^2=1849$ és $45^2=2025$ között helyezkedik el, ezért azt kell
megvizsgálnunk, hogy melyik két szám helyezkedik el a 2008 alatt és
felett a $43^2$ és a $47^2=2209$ sorában.

17-tel kisebb a 2008 a 2025-nél, tehát a ,,felette'' lévő sorban az 1849-
nél 16-tal kisebb szám van, azaz az 1833.
Az ,,alatta'' lévő sorban a 2209-nél 18-cal kisebb szám, azaz a 2191
látható.

A keresett összeg tehát $1833 + 2191 = 4024$.

A gondolatmenet lényegét az alábbi séma teszi szemléletessé.

\begin{tabular}{cccccc}


1833& \dots & 16-tal kisebb& $43^2$&& \cr
2008& \dots & 17-tel kisebb& \dots & $45^2$ &\cr
2191& \dots & 18-cal kisebb& \dots & \dots & $47^2$
\end{tabular}

\medskip
{\bf 4. feladat: } Legyen $n$ tetszőleges pozitív egész szám. Oldjuk meg a pozitív egész számok halmazán a  következő egyenletet:

$$\frac{n}{x_1+x_2+\dots+x_n} + \frac{1}{x_1\cdot x_2 \cdot \dots x_n} = 2. $$

\ki{Bencze Mihály}{Brassó}\medskip



{\bf 4. feladat megoldása: } Igazoljuk, hogy $x_k =1$, ( $k = 1,2,\dots, n$ ) az egyetlen megoldás. 

Ha $\max\{x_1, x_2,\dots, x_n\}>1$, akkor $x_1+x_2+\dots+x_n>n \land x_1\cdot x_2\cdot \dots \cdot x_n>1$,

így $2=\frac{n}{x_1+x_2+\dots+x_n}+\frac{1}{x_1\cdot x_2\cdot \dots \cdot x_n}<2$
ellentmondás.


\medskip
{\bf 5. feladat: } Legyen az  $a$  adott valós szám és az  $f$ olyan valós függvény, amelyre teljesül a következő egyenlet tetszőleges valós $x, y$-ra:
$$f(x+y)=f(x)f(a-y)+f(y)f(a-x).$$

Mennyi az $f(2008)$ értéke, ha $f(0)=\frac{1}{2}$?

\ki{Szabó Magdi}{Szabadka}\medskip


{\bf 5. feladat I. megoldása: } Az $y = 0$ majd az $x = a$ helyettesítés után

(1) $f(x)=f(x)f(a)+f(0)f(a-x)$

majd $f(a)=[f(a)]^2 +1/4$,
amiből következik, hogy $(f(a)-1/2)^2 =0$, azaz $f(a) =1/2$.

Most az (1) és az $f(a) = 1/2 = f(0)$ alapján következik a

(2) $f(x) = f(a-x)$.

Így a feladatban leírt összefüggés a következőre alakul: $f(x+y) = 2f(x)f(y)$.

Továbbá $1/2 = f(a) =2f(x)f(a-x) = 2[f(x)]^2$, amiből az következik, hogy
$f(x)= \pm 1/2$.

Az $f(b)= -1/2$ ellentmondásra vezet; ugyanis $-1/2 = f(b) = f(b/2 +b/2)= 2[f(b/2)]^2$, 
ami lehetetlen, tehát $f(x)= 1/2$ minden valós $x$-re, így az $f(2008)=1/2$.


\medskip

{\bf 5. feladat II. megoldása: } Az $x = y = 0$ helyettesítésből megkapjuk, hogy $f(a) =1/2$.
Azután $y$ helyére $0$-t ill. $a$-t helyettesítünk és azt kapjuk, hogy
$f(x) = f(a-x)$ ill. $f(x) = f(a+x)$.

Minden valós $x$-re teljesül, hogy $f(-x) = f(a-(-x)) = f(a+x) = f(x)$.
A feladat feltétele alapján minden $x,y$-ra következik, hogy:
$f(x-y) = f(x) f(a+y) + f(-y)f(a-x) = f(x+y) = f(x) f(a-y) + f(y)f(a-x) =
f(x+y)$.


Az $y=x$ helyettesítéssel $f(2x) = f(0) = 1/2$, azaz $f(x) = 1/2$, így az
$f(2008)=1/2$.


\medskip
{\bf 6. feladat: } Az  $ABC$ háromszög  $A$ pontból induló belső szögfelezője a háromszög köré írható kört $E$ pontban metszi. A kör $E$-beli érintője $AC$ egyenest $D$-ben, $AB$ egyenest $F$-ben metszi. Bizonyítsátok be, hogy 
$\displaystyle{\frac{|AD|+|AF|}{|AE|}=\frac{|FD|}{|BE|}}$.

\ki{Egyed László}{Baja}


\eject
{\bf 6. feladat megoldása: } Készítsünk ábrát!

\begin{center}
\psset{xunit=1.0cm,yunit=1.0cm,algebraic=true,dotstyle=o,dotsize=3pt 0,linewidth=0.8pt,arrowsize=3pt 2,arrowinset=0.25}
\begin{pspicture*}(-2.88,-0.74)(7.5,5.94)
\pscircle(3.04,2.74){2.84}
\psline(0.34,1.86)(5.64,1.6)
\psline(4.36,5.26)(2.9,-0.09)
\psline(-1.7,0.13)(6.29,-0.26)
\psline(4.36,5.26)(-1.7,0.13)
\psline(4.36,5.26)(6.29,-0.26)
\rput[tl](0.84,2.25){$ \beta $}
\rput[tl](3.78,4.75){$ \frac{\alpha}{2} $}
\rput[tl](4.28,4.8){$ \frac{\alpha}{2} $}
\begin{scriptsize}
\psdots[dotstyle=*](4.36,5.26)
\rput[bl](4.44,5.38){$A$}
\psdots[dotstyle=*](0.34,1.86)
\rput[bl](-0.04,1.8){$B$}
\psdots[dotstyle=*](5.64,1.6)
\rput[bl](5.98,1.5){$C$}
\psdots[dotstyle=*](2.9,-0.09)
\rput[bl](2.64,-0.48){$E$}
\psdots[dotstyle=*](-1.7,0.13)
\rput[bl](-1.94,-0.2){$F$}
\psdots[dotstyle=*](6.29,-0.26)
\rput[bl](6.54,-0.3){$D$}
\end{scriptsize}
\end{pspicture*}
\end{center}

A $CBE$ és $CAE$ kerületi szögek azonos ívhez tartoznak, így egyenlőek: 
$CBE\sphericalangle = CAE\sphericalangle =\frac{\alpha}{2}$.

Az $AED$ érintő szárú kerületi szög, valamint az $ABE$ kerületi szög azonos ívekhez tartoznak,
így egyenlőek. $AED\sphericalangle = ABE\sphericalangle = \beta+\frac{\alpha}{2}$.

Ebből viszont következik, hogy $AED$ és $ABE$
háromszögek hasonlóak. Felírhatjuk tehát a hasonlóság arányát:
$\frac{ED}{AD}=\frac{BE}{AE}$, ahonnan 
$ED=\frac{AD\cdot BE}{AE}$.


A szögfelező tételből:
$\frac{FE}{ED}=\frac{AF}{AD}$, ahonnan $FE=\frac{AF\cdot ED}{AD}$.

$FD=FE+ED=\frac{AF\cdot ED}{AD}+\frac{AD\cdot BE}{AE}$

Osszuk el az egyenlet mindkét oldalát $BE$-vel.

$\frac{FD}{BE}=\frac{AF\cdot ED}{AD\cdot BE}+\frac{AD}{AE} $, felhasználva hogy 

$ED = \frac{AD\cdot BE}{AE}$, kapjuk:

$\frac{FD}{BE}=\frac{AF}{AD}\cdot\frac{AD}{DE}+\frac{AD}{DE} $, amiből

$\frac{FD}{BE}=\frac{AF+AD}{BE}$.

Ezzel állításunkat bizonyítottuk.


\end{document}
