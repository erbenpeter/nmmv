\documentclass[a4paper,10pt]{article} 
\usepackage[utf8]{inputenc}
\usepackage{t1enc}
\usepackage[magyar]{babel}
\usepackage{pstricks,pstricks-add,pst-math,pst-xkey}
\usepackage{graphicx}
\usepackage{amssymb}
\voffset - 20pt
\hoffset - 35pt
\textwidth 450pt
\textheight 650pt 
\frenchspacing 

\pagestyle{empty}
\def\ki#1#2{\hfill {\it #1 (#2)}\medskip}

\def\tg{\, \hbox{tg} \,}
\def\ctg{\, \hbox{ctg} \,}
\def\arctg{\, \hbox{arctg} \,}
\def\arcctg{\, \hbox{arcctg} \,}

\begin{document}
\begin{center} \Large {\em XVII. Nemzetközi Magyar Matematika Verseny} \end{center}
\begin{center} \large{\em Kassa, 2008. március 6-9.} \end{center}
\smallskip
\begin{center} \large{\bf 12. osztály} \end{center}
\bigskip 

{\bf 1. feladat: } $n$ darab ceruzát egyenként két részre törünk. Az így kapott $2n$ darab ceruzát kettesével találomra párosítjuk. Mi a valószínűsége annak, hogy minden így kapott párból összeragasztható az eredeti ceruza?

\ki{Bencze Mihály}{Brassó}\medskip

{\bf 2. feladat: } Az  háromszögben $B\angle=50^\circ$, $C\angle=70^\circ$, $H$ a magasságok metszéspontja és $I$ a háromszögbe írt kör középpontja. Számítsátok ki az $IHC$ háromszög belső szögeit.

\ki{Neubauer Ferenc}{Munkács}\medskip

{\bf 3. feladat: } Egy függvény minden valós  $x, y$ számpárra eleget tesz az $f(x)+f(y)=f(x+y)-xy-1$ függvényegyenletnek. Ha $f(1)=1$, akkor van-e olyan 1-től különböző $n$ egész szám, amelyre $f(n)=n$?

\ki{Oláh György}{Komárom}\medskip

{\bf 4. feladat: } Az $x_1, x_2,\dots, x_n$  $(n\ge 2)$ egész számokra érvényes 
     $$|x_1|+|x_2|+\dots+|x_n| - |x_1+ x_2+\dots+ x_n|\in
     \{-2, 2\}. $$
Igazoljátok, hogy létezik legalább egy $x_k$ ezek közül, amelyre $x_k\in \{-1, 1\}$.  

\ki{Bencze Mihály}{Brassó}\medskip

{\bf 5. feladat: } Bizonyítsátok be, hogy ha egy háromszögben az egyik csúcsból induló magasságvonal, súlyvonal és szögfelező négy egyenlő részre osztja a szöget, akkor a háromszög derékszögű. 

\ki{Egyed László}{Baja}\medskip

{\bf 6. feladat: } Határozzátok meg azt az $(x, y)$  számpárt, amelyre az 
$$f(x,y)=\sqrt{x^2+y^2-8x+16}+\sqrt{x^2+y^2-16x-18y+145}+$$ 
$$+\sqrt{x^2+y^2+8x-24y+160}+\sqrt{x^2+y^2+4x+2y+5}$$
		
értéke minimális lesz.

\ki{Olosz Ferenc}{Szatmárnémeti}\medskip

\vfill
\end{document}
