\documentclass[a4paper,10pt]{article} 
\usepackage[utf8]{inputenc}
\usepackage{t1enc}
\usepackage[magyar]{babel}
\usepackage{pstricks,pstricks-add,pst-math,pst-xkey}
\usepackage{graphicx}
\usepackage{amssymb}
\voffset - 20pt
\hoffset - 35pt
\textwidth 450pt
\textheight 650pt 
\frenchspacing 

\pagestyle{empty}
\def\ki#1#2{\hfill {\it #1 (#2)}\medskip}

\def\tg{\, \hbox{tg} \,}
\def\ctg{\, \hbox{ctg} \,}
\def\arctg{\, \hbox{arctg} \,}
\def\arcctg{\, \hbox{arcctg} \,}

\begin{document}
\begin{center} \Large {\em XVII. Nemzetközi Magyar Matematika Verseny} \end{center}
\begin{center} \large{\em Kassa, 2008. március 6-9.} \end{center}
\smallskip
\begin{center} \large{\bf 11. osztály} \end{center}
\bigskip 

{\bf 1. feladat: } Fejeződhet-e a $3^n$ valamely természetes $n$ számra $0001$-re?

\ki{Zolnai Irén}{Újvidék}\medskip

{\bf 2. feladat: }  A  $K, L, M, N$ az $ABCDE$ ötszög  $BC$, $CD$, $DE$, $EA$  oldalainak felezőpontjai, a $Q$ és $P$ pontok pedig az $LN$, $KM$ szakaszok felezőpontjai. Bizonyítsátok be, hogy $PQ||AB$-vel és határozzátok meg a szakaszok hosszának arányát.

\begin{center}
\psset{xunit=0.6cm,yunit=0.6cm,algebraic=true,dotstyle=o,dotsize=3pt 0,linewidth=0.8pt,arrowsize=3pt 2,arrowinset=0.25}
\begin{pspicture*}(-3.68,-2.8)(8.46,5.32)
\psline(-0.38,-0.46)(4.78,-1.4)
\psline(4.78,-1.4)(7.02,1.38)
\psline(7.02,1.38)(0.86,4.16)
\psline(0.86,4.16)(-2.04,1.9)
\psline(-2.04,1.9)(-0.38,-0.46)
\psline(-1.21,0.72)(3.94,2.77)
\psline(-0.59,3.03)(5.9,-0.01)
\psline(1.37,1.75)(2.66,1.51)
%\psdots[dotstyle=*](-0.38,-0.46)
\rput[bl](-0.76,-0.99){$A$}
%\psdots[dotstyle=*](4.78,-1.4)
\rput[bl](4.92,-1.9){$B$}
%\psdots[dotstyle=*](7.02,1.38)
\rput[bl](7.1,1.5){$C$}
%\psdots[dotstyle=*](0.86,4.16)
\rput[bl](0.94,4.28){$D$}
%\psdots[dotstyle=*](-2.04,1.9)
\rput[bl](-2.54,1.96){$E$}
%\psdots[dotstyle=*](5.9,-0.01)
\rput[bl](6.2,-0.22){$K$}
%\psdots[dotstyle=*](3.94,2.77)
\rput[bl](4.1,3){$L$}
%\psdots[dotstyle=*](-0.59,3.03)
\rput[bl](-0.86,3.24){$M$}
%\psdots[dotstyle=*](-1.21,0.72)
\rput[bl](-1.78,0.5){$N$}
%\psdots[dotstyle=*](1.37,1.75)
\rput[bl](1.22,1.04){$Q$}
%\psdots[dotstyle=*](2.66,1.51)
\rput[bl](2.74,1.62){$P$}
\end{pspicture*}
\end{center}

\ki{Mészáros József}{Galánta}\medskip

{\bf 3. feladat: } A természetes számok halmazán értelmezett $f$ függvényre teljesül a következő egyenlőség: $f(1)+2^2f(2)+\dots+n^2f(n)=n^3f(n)$  tetszőleges $n\ge 1$  esetén.
Ha $f(1) = 2008$, határozzátok meg $f(2008)$ értéket.

\ki{Kovács Béla}{Szatmárnémeti}\medskip

{\bf 4. feladat: } Oldjátok meg a $60p^2+57q=2007$ egyenletet, ha $p$ és $q$ pozitív prímszámok.

\ki{Egyed László}{Baja}\medskip

{\bf 5. feladat: }  Oldjátok meg a  
$\log_2\left(x^2+4\right)-\log_2 x=7x^2+4x-x^4-18$
egyenletet.

\ki{Olosz Ferenc}{Szatmárnémeti}\medskip

{\bf 6. feladat: } Az  $ABC$ szabályos háromszög oldalai $\sqrt{p}$ hosszúságúak. A háromszög egy  belső pontja az $A, B,C$ pontoktól rendre 1, $\sqrt{r}$, $\sqrt{r+1}$ egység távolságra van, ahol  $p$ és $r$  prímszámok. Mekkora az $ABC$ háromszög kerülete?

\ki{Bíró Bálint}{Eger}\medskip

\vfill
\end{document}
