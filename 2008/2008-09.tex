\documentclass[a4paper,10pt]{article} 
\usepackage[utf8]{inputenc}
\usepackage{t1enc}
\usepackage{graphicx}
\usepackage{amssymb}
\voffset - 20pt
\hoffset - 35pt
\textwidth 450pt
\textheight 650pt 
\frenchspacing 

\pagestyle{empty}
\def\ki#1#2{\hfill {\it #1 (#2)}\medskip}

\def\tg{\, \hbox{tg} \,}
\def\ctg{\, \hbox{ctg} \,}
\def\arctg{\, \hbox{arctg} \,}
\def\arcctg{\, \hbox{arcctg} \,}

\begin{document}
\begin{center} \Large {\em XVII. Nemzetközi Magyar Matematika Verseny} \end{center}
\begin{center} \large{\em Kassa, 2008. március 6-9.} \end{center}
\smallskip
\begin{center} \large{\bf 9. osztály} \end{center}
\bigskip 

{\bf 1. feladat: } Határozzuk meg az összes $p, q$ prímszám kettősöket, amelyekre érvényes:
$$ 145p^2-p=q^2-q$$

\ki{Oláh György}{Komárom}\medskip

{\bf 2. feladat: } 
Jancsi unalmában 21 darab egybevágó négyzetre számokat írt, 
mégpedig a következőképpen
4 darabra 1-est, 
2 darabra 2-est, 
7 darabra 3-ast és 8 darabra 4-est írt. 
Egy más alkalommal ezekből a négyzetekből 20 darabot felhasználva kirakott egy $4 \times 5$-ös téglalapot. 
Amikor végzett a kirakással, nézegette a számokat és észrevette, hogy függőlegesen minden oszlopban egyezik az összeg. 
Tovább nézegetve észrevette, hogy vízszintesen a sorokban is egyenlők az összegek. 
Melyik kártyát nem használta fel Jancsi a kirakáshoz?

\ki{Pintér Ferenc}{Nagykanizsa}\medskip

{\bf 3. feladat: } Harry Potter a Roxfort Boszorkány- és Varázslóképző Szakiskolában öt év alatt összesen 31 vizsgát tett le. Mindegyik évben több vizsgája volt, mint az előző évben, az ötödik évben pedig háromszor annyi tárgyból vizsgázott le, mint az első évben. Hány vizsgát kellett teljesítenie Harry Potternek a negyedik évben?

\ki{Péics Hajnalka}{Szabadka}\medskip

{\bf 4. feladat: } Egy konferenciának 2008 résztvevője van. Bármely három háromtagú csoportban van két olyan személy,  akik azonos nyelven beszélnek. Bizonyítsátok be, hogy ha minden résztvevő legfeljebb öt nyelvet beszél, akkor van legalább 202 olyan személy akik azonos nyelvet beszélnek.

\ki{Szabó Magdi}{Szabadka}\medskip

{\bf 5. feladat: } Az $ABCD$ rombusz $AC$ átlóján tetszőlegesen választott $E$  pont különbözik az $A$ és  $C$ csúcstól. Legyenek  az $AB$ és $BC$  egyenesen rendre az $N$ és $M$ pontok olyanok, hogy $|AE|=|NE|$ és $|CE|=|ME|$. Jelölje  $K$ az $AM$ és $CN$  egyenesek metszéspontját. Igazoljuk, hogy a  $K, D$ és $E$  pontok kollineárisak (egy egyeneshez illeszkednek).

\ki{Sipos Elvíra}{Zenta}\medskip

{\bf 6. feladat: } Bizonyítsátok be, hogy ha $x, y, z\ge 0$, akkor 
$x+y+z+xyz(xy+yz+zx)\ge 6xyz$.\\ 
Mikor áll fenn egyenlőség? 

\ki{Bencze Mihály}{Brassó}\medskip

\vfill
\end{document}
