\documentclass[a4paper,10pt]{article} 
\usepackage[utf8]{inputenc}
\usepackage{t1enc}
\usepackage[magyar]{babel}
\usepackage{graphicx}
\usepackage{amssymb}
\voffset - 20pt
\hoffset - 35pt
\textwidth 450pt
\textheight 650pt 
\frenchspacing 

\pagestyle{empty}
\def\ki#1#2{\hfill {\it #1 (#2)}\medskip}

\def\tg{\, \hbox{tg} \,}
\def\ctg{\, \hbox{ctg} \,}
\def\arctg{\, \hbox{arctg} \,}
\def\arcctg{\, \hbox{arcctg} \,}

\begin{document}
\begin{center} \Large {\em XVII. Nemzetközi Magyar Matematika Verseny} \end{center}
\begin{center} \large{\em Kassa, 2008. március 6-9.} \end{center}
\smallskip
\begin{center} \large{\bf 10. osztály} \end{center}
\bigskip 

{\bf 1. feladat: } Osztható-e $20^{2008}+16^{2008}-3^{2008}-1$  \  $323$-al? Az állításodat igazold!

\ki{Oláh György}{Komárom}\medskip

{\bf 2. feladat: }  Az $ABC$ háromszögben $|AB|=20e$, $|AC|=16e$  és 
$|BC|=12e$. Egy $P$ középpontú és $2e$ sugarú kör végig gurul az $ABC$ háromszög belsejében úgy, hogy mindig érinti a háromszögnek legalább az egyik oldalát. Mekkora utat tesz meg $P$ addig, amíg először tér vissza a kiindulási helyzetébe? 

\ki{Dr. Kántor Sándorné}{Debrecen}\medskip

{\bf 3. feladat: } Egy nagy táblázat ,,közepére'' beírjuk az 1-et, majd az ábrán látható módon ,,csiga\-vo\-nal\-ban'' folytonosan beírjuk az egymást követő egész számokat. Mivel egyenlő a közvetlenül 2008 felett, illetve alatt álló két szám összege?

\centerline{
\begin{tabular}{|c|c|c|c|c|c|}
\hline
17 & 16 & 15 & 14 & 13 &  \\ \hline
$\downarrow $ & 5 & 4 & 3 & 12 &  \\ \hline
 & 6 & 1 & 2 & 11 &  \\ \hline
 & 7 & 8 & 9 & 10 & $\uparrow$ \\ \hline
 &  &  &  &  &  \\ \hline
\end{tabular}}


\ki{Kiss Sándor}{Nyíregyháza}\medskip

{\bf 4. feladat: } Legyen $n$ tetszőleges pozitív egész szám. Oldjuk meg a pozitív egész számok halmazán a  következő egyenletet:

$$\frac{n}{x_1+x_2+\dots+x_n} + \frac{1}{x_1\cdot x_2 \cdot \dots x_n} = 2. $$

\ki{Bencze Mihály}{Brassó}\medskip

{\bf 5. feladat: }  Legyen az  $a$  adott valós szám és az  $f$ olyan valós függvény, amelyre teljesül a következő egyenlet tetszőleges valós $x, y$-ra:
$$f(x+y)=f(x)f(a-y)+f(y)f(a-x).$$

Mennyi az $f(2008)$ értéke, ha $f(0)=\frac{1}{2}$?

\ki{Szabó Magdi}{Szabadka}\medskip

{\bf 6. feladat: } Az  $ABC$ háromszög  $A$ pontból induló belső szögfelezője a háromszög köré írható kört $E$ pontban metszi. A kör $E$-beli érintője $AC$ egyenest $D$-ben, $AB$ egyenest $F$-ben metszi. Bizonyítsátok be, hogy 
$\displaystyle{\frac{|AD|+|AF|}{|AE|}=\frac{|FD|}{|BE|}}$.

\ki{Egyed László}{Baja}\medskip

\vfill
\end{document}
