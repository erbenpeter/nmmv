\documentclass[a4paper,10pt]{article} 
\usepackage[latin2]{inputenc}
\usepackage{graphicx}
\usepackage{amssymb}
\voffset - 20pt
\hoffset - 35pt
\textwidth 450pt
\textheight 650pt 
\frenchspacing 

\pagestyle{empty}
\def\ki#1#2{\hfill {\it #1 (#2)}\medskip}

\def\tg{\, \hbox{tg} \,}
\def\ctg{\, \hbox{ctg} \,}
\def\arctg{\, \hbox{arctg} \,}
\def\arcctg{\, \hbox{arcctg} \,}

\begin{document}
\begin{center} \Large {\em XIII. Nemzetk�zi Magyar Matematika Verseny} \end{center}
\begin{center} \large{\em Nagydobrony, 2004. m�rc. 15-20.} \end{center}
\smallskip
\begin{center} \large{\bf 9. oszt�ly} \end{center}
\bigskip 

{\bf 1. feladat: } Egy kis erdei tavat egy forr\'as t\'apl\'al friss v\'izzel. Egyszer megjelent egy $183$ tag\'u elef\'antcsorda \'es egy nap alatt kiitta a t\'o viz\'et. K\'es\H obb, mikor \'ujra megtelt a t\'o, egy $37$ tag\'u csorda $5$ nap alatt itta ki a vizet. Egy elef\'ant h\'any nap alatt inn\'a ki a t\'o viz\'et?



\ki{Dr. Katz S�ndor}{Bonyh�d}\medskip

{\bf 2. feladat: } Az  $1$, $2$, $3$, $\ldots$, $2000$, $2001$, $2002$, $2003$, $2004$ sz\'amokat valamilyen sorrendben egym\'as mell\'e \'irjuk. Lehet-e az \'igy kapott \'uj sz\'am n\'egyzetsz\'am?


\ki{Dr. K�ntor S�ndorn�}{Debrecen}\medskip

{\bf 3. feladat: } Az ABCD t\'eglalapban AD = 3AB. Az E \'es F pontok AD-t h\'arom egyenl\H o r\'eszre osztj\'ak. Mennyi a BEA, BFA \'es BDA sz\"ogek \"osszege?


\ki{Bal�zsi Borb�la}{Beregsz�sz}\medskip

{\bf 4. feladat: } Az $a$, $b$ \'es $c$ pozit\'iv sz\'amok egy h\'aromsz\"og oldalainak hossz\'at jel\"olik, \'es \'erv\'enyes r\'ajuk a k\"ovetkez\H o \"osszef\"ugg\'es: $3b^2=2(c^2-a^2)$. Mekkora lehet a $\frac{b}{a}$ t\"ort \'ert\'eke?


\ki{Bogd�n Zolt�n}{Cegl�d}\medskip

{\bf 5. feladat: } Igazolja, hogy a h\'aromsz\"og sz\"ogfelez\H oinek metsz\'espontja \'es a h\'aromsz\"og cs\'ucsai k\"oz\"otti t\'avols\'agok n\'egyzeteinek \"osszege nem kevesebb a h\'aromsz\"og k\'etszeres ter\"ulet\'en\'el!


\ki{Bencze Mih�ly}{Brass�}\medskip

{\bf 6. feladat: } Bizony\'itsa be, hogy ha $p$ \'es $q$ h\'aromn\'al nagyobb pr\'imsz\'am, akkor $7p^2+11q^2-39$ nem pr\'imsz\'am.


\ki{Ol�h Gy�rgy}{Kom�rom}\medskip

{\bf 7. feladat: } Oldja meg a $\left(p-x\right)^2+\frac{2}{x}+4p=\left(p+\frac{1}{x}\right)^2+2x$  egyenletet az eg\'esz sz\'amok halmaz\'an, ha a $p$ param\'eter eg\'esz sz\'am!


\ki{B�r� B�lint}{Eger}\medskip

\vfill
\end{document}