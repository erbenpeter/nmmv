\documentclass[a4paper,10pt]{article} 
\usepackage[latin2]{inputenc}
\usepackage{graphicx}
\usepackage{amssymb}
\voffset - 20pt
\hoffset - 35pt
\textwidth 450pt
\textheight 650pt 
\frenchspacing 

\pagestyle{empty}
\def\ki#1#2{\hfill {\it #1 (#2)}\medskip}

\def\tg{\, \hbox{tg} \,}
\def\ctg{\, \hbox{ctg} \,}
\def\arctg{\, \hbox{arctg} \,}
\def\arcctg{\, \hbox{arcctg} \,}

\begin{document}
\begin{center} \Large {\em XIII. Nemzetk�zi Magyar Matematika Verseny} \end{center}
\begin{center} \large{\em Nagydobrony, 2004. m�rc. 15-20.} \end{center}
\smallskip
\begin{center} \large{\bf 10. oszt�ly} \end{center}
\bigskip 

{\bf 1. feladat: } K\'et g\'epkocsi halad az aut\'op\'aly\'an egy ir\'anyban. A k\"ozt\"uk l\'ev\H o t\'avols\'ag jelenleg $2$~km, \'es minden $n$-edik percben $\frac{1}{n^2}$~km-rel cs\"okken. Utol\'eri-e a m\'asodik j\'arm\H u az el\H otte halad\'ot?


\ki{Gecse Frigyes}{Ungv�r}\medskip

{\bf 2. feladat: } A $k_{1}$ \'es $k_{2}$ k\"or\"ok az $A$ \'es $B$ pontokban metszik egym\'ast.

\centerline{\includegraphics{figures/2004-10-2}}

A $k_{1}$ k\"or tetsz\H oleges ($A$-t\'ol \'es $B$-t\H ol k\"ul\"onb\"oz\H o) $K$ pontj\'at \"osszek\"otj\"uk az $A$ \'es $B$ pontokkal. A $KA$ \'es $KB$ egyenesek a $k_{2}$ k\"ort m\'asodszor a $P$ \'es $Q$ pontokban metszi. Bizony\'itsa be, hogy a $PQ$ h\'ur mer\H oleges a $k_{1}$ k\"or $KM$ \'atm\'er\H oj\'ere.


\ki{Dr. Pint�r Ferenc}{Nagykanizsa}\medskip

{\bf 3. feladat: } Melyek azok a term\'eszetes $n$ sz\'amok, melyekre $n^2-440$ teljes n\'egyzet!


\ki{Dr. K�ntor S�ndorn�}{Debrecen}\medskip

{\bf 4. feladat: } L\'etezik-e olyan $x$ \'es $y$ term\'eszetes sz\'am, melyekre $7^x-5^y=2004$?


\ki{Kacs� Ferenc}{Marosv�s�rhely}\medskip

{\bf 5. feladat: } Adott a t\'erben hat tetsz\H oleges pont. Ezeket \"osszek\"otj\"uk az \"osszes lehets\'eges m\'odon. Igazolja, hogy azon h\'arom szakasz felez\H opontjai \'altal alkotott h\'aromsz\"ogek s\'ulypontjai, amely szakaszoknak p\'aronk\'ent nincs egy k\"oz\"os v\'egpontja, egybeesnek.



\ki{Bencze Mih�ly}{Brass�}\medskip

{\bf 6. feladat: } Egy trap\'ezba, melynek egyik sz\'ara $40$~cm, ter\"ulete $1280$~\mbox{cm${}^2$}, k\"or \'irhat\'o. A trap\'ez magass\'aga az alapok m\'ertani k\"ozepe. Bizony\'itsa be, hogy a trap\'ez k\"or\'e is \'irhat\'o k\"or, \'es sz\'am\'itsa ki a be\'irt \'es k\"or\'e \'irt k\"or\"ok k\"oz\'eppontjai k\"oz\"otti t\'avols\'agot!


\ki{Gecse Frigyes}{Ungv�r}\medskip

{\bf 7. feladat: } Az $\frac{a}{b}$ k\"oz\"ons\'eges t\"ort tizedes t\"ort alakja olyan v\'egtelen szakaszos tizedes t\"ort, amelynek szakasza $\left(b-1\right)$ sz\'amjegyb\H ol \'all.
Az $a$ \'es $b$ sz\'amok pozit\'iv eg\'eszek.
Fejezz�k ki az egy szakaszban l\'ev\H o jegyek \"osszeg\'et $b$-vel!


\ki{Bogd�n Zolt�n}{Cegl�d}\medskip

\vfill
\end{document}