\documentclass[a4paper,10pt]{article} 
\usepackage[latin2]{inputenc}
\usepackage{graphicx}
\usepackage{amssymb}
\voffset - 20pt
\hoffset - 35pt
\textwidth 450pt
\textheight 650pt 
\frenchspacing 

\pagestyle{empty}
\def\ki#1#2{\hfill {\it #1 (#2)}\medskip}

\def\tg{\, \hbox{tg} \,}
\def\ctg{\, \hbox{ctg} \,}
\def\arctg{\, \hbox{arctg} \,}
\def\arcctg{\, \hbox{arcctg} \,}

\begin{document}
\begin{center} \Large {\em XIII. Nemzetk�zi Magyar Matematika Verseny} \end{center}
\begin{center} \large{\em Nagydobrony, 2004. m�rc. 15-20.} \end{center}
\smallskip
\begin{center} \large{\bf 10. oszt�ly} \end{center}
\bigskip 

{\bf 1. feladat: } K\'et g\'epkocsi halad az aut\'op\'aly\'an egy ir\'anyban. A k\"ozt\"uk l\'ev\H o t\'avols\'ag jelenleg $2$~km, \'es minden $n$-edik percben $\frac{1}{n^2}$~km-rel cs\"okken. Utol\'eri-e a m\'asodik j\'arm\H u az el\H otte halad\'ot?


\ki{Gecse Frigyes}{Ungv�r}\medskip

{\bf 1. feladat I. megold�sa: } A k\'et j\'arm\H u k\"oz\"otti t\'avols\'ag $n$ perc alatt $\left(\frac{1}{1^2} + \frac{1}{2^2} + \frac{1}{3^2} + \ldots + \frac{1}{n^2}\right)$-tel cs\"okken, ez\'ert ez a t\'avols\'ag $$D_{n} = 2-\left(\frac{1}{1^2} + \frac{1}{2^2} + \frac{1}{3^2} + \ldots + \frac{1}{n^2}\right)=1-\left(\frac{1}{2^2} + \frac{1}{3^2} + \ldots + \frac{1}{n^2}\right) \ \hbox{km}.$$ De 
{\setlength\arraycolsep{2pt}\begin{eqnarray*}
\frac{1}{2^2} + \frac{1}{3^2} + \ldots + \frac{1}{n^2}&<&\frac{1}{1\cdot 2} + \frac{1}{2\cdot 3} + \ldots + \frac{1}{\left(n-1\right)\cdot n} = \\
&=&1 - \frac{1}{2} + \frac{1}{2} - \frac{1}{3} + \ldots + \frac{1}{n-1} -\frac{1}{n} = 1-\frac{1}{n}.\end{eqnarray*}} Ez\'ert $D_{n}>1-\left(1-\frac{1}{n}\right) = \frac{1}{n}>0$ $n$ minden term\'eszetes \'ert\'ek\'evel. Teh\'at a m\'asik j\'arm\H u \textbf{soha nem \'eri utol az el\H otte halad\'ot}.

\medskip


\hbox to \hsize{\hskip2truecm\hrulefill\hskip2truecm}
{\bf 2. feladat: } A $k_{1}$ \'es $k_{2}$ k\"or\"ok az $A$ \'es $B$ pontokban metszik egym\'ast.

\centerline{\includegraphics{figures/2004-10-2}}

A $k_{1}$ k\"or tetsz\H oleges ($A$-t\'ol \'es $B$-t\H ol k\"ul\"onb\"oz\H o) $K$ pontj\'at \"osszek\"otj\"uk az $A$ \'es $B$ pontokkal. A $KA$ \'es $KB$ egyenesek a $k_{2}$ k\"ort m\'asodszor a $P$ \'es $Q$ pontokban metszi. Bizony\'itsa be, hogy a $PQ$ h\'ur mer\H oleges a $k_{1}$ k\"or $KM$ \'atm\'er\H oj\'ere.


\ki{Dr. Pint�r Ferenc}{Nagykanizsa}\medskip

{\bf 2. feladat I. megold�sa: } Legyen $E$ a $PQ$ egyenes \'es $MK$ egyenes metsz\'espontja. Akkor elegend\H o bizony\'itani, hogy $QEK$ h\'aromsz\"ogben az $E$ sz\"og der\'eksz\"og, vagy $EKQ\angle+EQK\angle=90^{\circ}$.

\centerline{\includegraphics{figures/2004-10-2}}

Fel\'irhatjuk: $EKQ\angle + EQK\angle = MKA\angle + AKB\angle + BAK\angle$ Ugyanis $EQK$ \'es $BAK$ sz\"ogek a $k_{2}$ k\"or $PB$ \'iv\'ere t\'amaszkod\'o ker\"uleti sz\"ogek, ez\'ert egyenl\H ok. De a fel\'irt h\'arom sz\"og \"osszege a $k_{1}$ k\"or\"on sorban az $MA$, $AB$ \'es $BK$ \'ivekre t\'amaszkodnak, melyek \"osszesen egy f\'elk\"ort tesznek ki, teh\'at fokm\'ert\'ek\"uk \"osszege $180^{\circ}$. Ez\'ert a h\'arom sz\"og \"osszege ennek a fele, teh\'at $90^{\circ}$.
\medskip


\hbox to \hsize{\hskip2truecm\hrulefill\hskip2truecm}
{\bf 3. feladat: } Melyek azok a term\'eszetes $n$ sz\'amok, melyekre $n^2-440$ teljes n\'egyzet!


\ki{Dr. K�ntor S�ndorn�}{Debrecen}\medskip

{\bf 3. feladat I. megold�sa: } Legyen $n^{2}-440=m^{2}$, $m$ eg\'esz.
Ebb\H ol $n^{2}-m^{2}=440\iff \left(n+m\right)\left(n-m\right)=2^{3}\cdot 5\cdot 11$.
Az $n + m$ valamint $n - m$ sz\'amokra fenn\'all\'o lehet\H os\'egek kiv\'alaszt\'asakor most azt vessz\"uk figyelembe, hogy $n+m>n-m$ tov\'abb\'a mindkett\H o p\'aros, mert szorzatuk p\'aros. Ez\'ert a k\"ovetkez\H o lehet\H os\'egeket kapjuk:
$$\begin{array}{llll}
\left\{\begin{array}{l}
n+m=4\cdot 5\cdot 11,\\
n-m=2;
\end{array}
\right.&\left\{
\begin{array}{l}
n+m=4\cdot 11,\\
n-m=2\cdot 5;
\end{array}
\right.&\left\{
\begin{array}{l}
n+m=2\cdot 11,\\
n-m=4\cdot 5;
\end{array}
\right.&\left\{
\begin{array}{l}
n+m=2\cdot 5\cdot 11,\\
n-m=4.
\end{array}
\right.\\
\end{array}$$

Megoldva a n\'egy egyenletrendszert, n\'egy $n$-\'ert\'eket kapunk: $\textbf{111}$, $\textbf{27}$, $\textbf{21}$ \'es $\textbf{57}$. Mindegyikre teljes\"ul a k\"ovetelm\'eny: $111^{2}-440=11881=109^{2}$; $27^{2}-440=289=17^{2}$; $21^{2}-440=1=1^{2}$; $57^{2}-440=2809=53^{2}$.
\medskip


\hbox to \hsize{\hskip2truecm\hrulefill\hskip2truecm}
{\bf 4. feladat: } L\'etezik-e olyan $x$ \'es $y$ term\'eszetes sz\'am, melyekre $7^x-5^y=2004$?


\ki{Kacs� Ferenc}{Marosv�s�rhely}\medskip

{\bf 4. feladat I. megold�sa: } Felhaszn\'aljuk a sz\'amelm\'elet ismert \'all\'it\'asait: b\'armely term\'eszetes $n$-re $a^{n} - b^{n}$ oszthat\'o $\left(a - b\right)$-vel, p\'aratlan $n$-re pedig  oszthat\'o $\left(a + b\right)$-vel, ahol $a$ \'es $b$ eg\'esz sz\'amok.
Fel\'irjuk az egyenletet m\'ask\'eppen: $\left(7^{x} + 1\right)-\left(5^{y}-1\right)=2006$.
Itt p\'aratlan $x$-re \'es b\'armilyen $y$-ra a baloldal mindk\'et tagja oszthat\'o $4$-gyel, a jobb oldal $2006$ viszont nem.
Teh\'at $x$ nem lehet p\'aratlan.
Ha az egyenletet $\left(7^{x}-1\right)-\left(5^{y}+1\right)=2002$ alakban \'irjuk fel, akkor a bal oldal mindk\'et tagja oszthat\'o $6$-tal, a jobb oldal viszont $2002$ nem oszthat\'o $6$-tal.
Teh\'at $y$ sem lehet p\'aratlan.
Ha $x$ \'es $y$ p\'arosak, akkor $x = 2k$, $y = 2m$, \'es az egyenletet ilyen alakban \'irhatjuk: $\left(49^{k}-1\right)-\left(25^{m}-1\right)=2004$.
A bal oldalon $49^{k}-1$ oszthat\'o $48$-cal, $25^{m}-1$ pedig $24$-gyel, vagyis a bal oldal oszthat\'o $8$-cal a jobb oldal $2004$ pedig nem.
Teh\'at, a \textbf{keresett $x$ \'es $y$ nem l\'etezik}.

\medskip


\hbox to \hsize{\hskip2truecm\hrulefill\hskip2truecm}
{\bf 5. feladat: } Adott a t\'erben hat tetsz\H oleges pont. Ezeket \"osszek\"otj\"uk az \"osszes lehets\'eges m\'odon. Igazolja, hogy azon h\'arom szakasz felez\H opontjai \'altal alkotott h\'aromsz\"ogek s\'ulypontjai, amely szakaszoknak p\'aronk\'ent nincs egy k\"oz\"os v\'egpontja, egybeesnek.



\ki{Bencze Mih�ly}{Brass�}\medskip

{\bf 5. feladat I. megold�sa: } Legyen $A$, $B$, $C$, $D$, $E$, $F$ az adott hat pont, $AB$, $CD$ \'es $EF$ a h\'arom kiv\'alasztott, a felt\'etelnek megfelel\H o szakasz.
$M$, $K$ \'es $P$ sorban e szakaszok felez\H opontjai, az $MKP$ h\'aromsz\"og s\'ulypontja pedig $S$. Akkor a t\'er tetsz\H oleges $O$ pontj\'ara a h\'aromsz\"og s\'ulypontj\'anak valamint a szakasz felez\H opontj\'anak ismert vektork\'epleteit alkalmazva fel\'irhatjuk:
{\setlength\arraycolsep{2pt}\begin{eqnarray*}
\overrightarrow{OS}&=&\frac{1}{3}\left(\overrightarrow{OM} + \overrightarrow{OK} + \overrightarrow{OP}\right)=\\[6pt]
&=& \frac{1}{3}\left(\frac{1}{2}\left(\overrightarrow{OA} + \overrightarrow{OB}\right) + \frac{1}{2}\left(\overrightarrow{OC}+\overrightarrow{OD}\right) + \frac{1}{2}\left(\overrightarrow{OE} + \overrightarrow{OF}\right)\right)=\\[6pt]
&=&\frac{1}{6}\left(\overrightarrow{OA} + \overrightarrow{OB} + \overrightarrow{OC}+\overrightarrow{OD} + \overrightarrow{OE} + \overrightarrow{OF}\right).
\end{eqnarray*}}
K\"onny\H u bel\'atni, hogy ugyanezt az eredm\'enyt kapjuk, ha az $AB$, $CD$ \'es $EF$ szakaszok helyett b\'armely m\'asik h\'arom alkalmas szakaszt v\'alasztjuk. Ebb\H ol k\"ovetkezik, hogy az $S$ pont minden esetben ugyanaz.

\medskip


\hbox to \hsize{\hskip2truecm\hrulefill\hskip2truecm}
{\bf 6. feladat: } Egy trap\'ezba, melynek egyik sz\'ara $40$~cm, ter\"ulete $1280$~\mbox{cm${}^2$}, k\"or \'irhat\'o. A trap\'ez magass\'aga az alapok m\'ertani k\"ozepe. Bizony\'itsa be, hogy a trap\'ez k\"or\'e is \'irhat\'o k\"or, \'es sz\'am\'itsa ki a be\'irt \'es k\"or\'e \'irt k\"or\"ok k\"oz\'eppontjai k\"oz\"otti t\'avols\'agot!


\ki{Gecse Frigyes}{Ungv�r}\medskip

{\bf 6. feladat I. megold�sa: } Az \'abra jel\"ol\'eseivel: $PC=\sqrt{40^{2}-ab}$; $BM=a-b-\sqrt{40^{2}-ab}$.
Figyelembe vessz\"uk, hogy a trap\'ezba k\"or \'irhat\'o, vagyis a szemben fekv\H o oldalak \"osszege egyenl\H o, az $AMB$ der\'eksz\"og\H u h\'aromsz\"ogben alkalmazzuk a Pitagorasz-t\'etelt, fel\'irjuk a trap\'ez ter\"ulet\'et, kapjuk az al\'abbi egyenletrendszert:
$$\left\{
\begin{array}{l}
a+b=40 + c;\\
\left(\sqrt{ab}\right)^{2} + \left(a - b - \sqrt{40^{2} - ab}\right)^2 = c^2;\\
\left(a + b\right)\sqrt{ab}=2560.
\end{array}
\right.$$

\centerline{\includegraphics{figures/2004-10-6A}}

Az els\H o egyenletb\H ol $c=a+b-40$, ebb\H ol $c^{2}=a^{2}+b^{2}+1600 + 2ab-80a-80b$.
Ezt az eredm\'enyt a m\'asodik egyenletbe helyettes\'itve, \'es elv\'egezve az \'atalak\'it\'asokat, a $\sqrt{1600 - ab}\left(a-b\right)=40\left(a + b\right) - 2ab$ egyenlethez jutunk, amit n\'egyzetre emelve csak $\left(a + b\right)$ \'es $ab$  alak\'u kifejez\'eseket tartalmaz\'o egyenlet ad\'odik:
$$\left(1600 - ab\right)\left(\left(a+b\right)^{2}-4ab\right)=1600\left(a+b\right)^{2}-160ab\left(a+b\right) + 4a^{2}b^{2}.$$

Ebb\H ol egyszer\H us\'it\'esekkel az $\left(a+b\right)^{2}-160\left(a+b\right)+6400=0$ egyenletet kapjuk, ahonnan $a+b=80$.
Ezut\'an a rendszer harmadik egyenlet\'eb\H ol $\sqrt{ab}=32\iff ab=1024$.
Felt\'etelezve, hogy $a>b$, az alapokra $a = 64$cm \'es $b = 16$cm \'ert\'ekeket kapjuk.
A $c$ sz\'arra $c = 80-40=40$(cm) \'ert\'eket kapjuk, ami annyit jelent, hogy a trap\'ez egyenl\H o sz\'ar\'u.
Teh\'at, k\"or\'e is \'irhat\'o k\"or.
Mind a be\'irt, mind a k\"or\'e \'irt k\"or k\"oz\'eppontja a szimmetriatengelyen fekszik.
Legyen $GK = y$, akkor $KE = 32 - y$.
Az $AEK$ \'es $BGK$ der\'eksz\"og\H u h\'aromsz\"ogekb\H ol a k\"or\'e \'irt k\"or sugar\'anak n\'egyzet\'et kifejezve az $y^{2}+32^{2}=\left(32-y\right)^{2}+8^{2}$ egyenlet ad\'odik, melyet megoldva $y = 1$ \'ert\'eket kapjuk.
Ez azt jelenti, hogy a $K$ pont val\'oban a trap\'ez belsej\'eben van.
V\'eg\"ul
$\textbf{OK = 16-1 = 15}$\textbf{~(cm)}.
\medskip

{\bf 6. feladat II. megold�sa: } El\H osz\"or igazoljuk, hogy a trap\'ez egyenl\H o sz\'ar\'u.
Ehhez elegend\H o bizony\'itani, hogy a be\'irt k\"or az alapokat a felez\H opontokban \'erinti.
A trap\'ez alapon fekv\H o sz\"ogeinek \"oszege $180^{\circ}$, a be\'irt k\"or k\"oz\'eppontja pedig a sz\"ogfelez\H ok metsz\'espontja, ez\'ert $AOB\angle=COD\angle=90^{\circ}$.
Az $AOB$ \'es $COD$ der\'eksz\"og\H u h\'aromsz\"ogekb\H ol $r^{2}=xu=yv$, ahonnan $\frac{x}{y}=\frac{v}{u}$ (*).

\centerline{\includegraphics{figures/2004-10-6B}}

A felt\'etel szerint a magass\'ag az alapok m\'ertani k\"ozepe: $\left(2r\right)^2=\left(x + y\right)\left(u + v\right)$, vagyis $xu + xv + yu + yv = 4xu$.
De $xu=yv$ miatt $xv-2xu+yu=0$, melyet $\left(yu\right)$-val osztva (*) figyelembev\'etel\'evel az ${\left(\frac{x}{y}\right)}^{2} - 2\left(\frac{x}{y}\right) + 1 = 0 \iff \left(\frac{x}{y} - 1\right)^2 = 0$ egyenlethez jutunk.
Ebb\H ol $\frac{x}{y} = 1$, azaz $x=y$ \'es akkor $u=v$.
A tov\'abbi sz\'am\'it\'asokat az olvas\'ora b\'izzuk.
\medskip


\hbox to \hsize{\hskip2truecm\hrulefill\hskip2truecm}
{\bf 7. feladat: } Az $\frac{a}{b}$ k\"oz\"ons\'eges t\"ort tizedes t\"ort alakja olyan v\'egtelen szakaszos tizedes t\"ort, amelynek szakasza $\left(b-1\right)$ sz\'amjegyb\H ol \'all.
Az $a$ \'es $b$ sz\'amok pozit\'iv eg\'eszek.
Fejezz�k ki az egy szakaszban l\'ev\H o jegyek \"osszeg\'et $b$-vel!


\ki{Bogd�n Zolt�n}{Cegl�d}\medskip

{\bf 7. feladat I. megold�sa: } V\'egezz\"uk el n\'eh\'any t\"ort tizedess\'e alak\'it\'as\'at! Pl. $\frac{1}{7}=0,142857142857\ldots= 0,\left(142857\right)$, $\frac{1}{6}=0,1666\ldots=0,1\left(6\right)$, $\frac{1}{13}=0,076923076923\ldots=0,\left(076923\right)$.
Azt tal\'aljuk, hogy csak az els\H o p\'eld\'ara teljes\"ul a feladat felt\'etele: a nevez\H o $7$, \'es $\left(7-1\right)$ sz\'amjegyb\H ol \'all a szakasz.
Ebben az esetben az oszt\'as folyamatos elv\'egz\'esekor minden lehets\'eges marad\'ekot megkapunk (term\'eszetesen a $0$ kiv\'etel\'evel).
Ezek a marad\'ekok: $1$, $2$, $3$, $4$, $5$, $6$ (m\'as sorrendben).
Az oszt\'askor mindig egy $0$-t \'irunk jobbr\'ol a marad\'ekhoz, vagyis a tizedes t\"ort sz\'amjegyei \'ugy \'allnak el\H o, hogy 10-et, 20-at, 30-at, stb. osztjuk 7-tel, mik\"ozben a  szakasz sz\'amjegyei a nem teljes h\'anyadosok lesznek, \'ugy is mondhatjuk, a $\left[\frac{10}{7}\right]$;$\left[\frac{20}{7}\right]$ stb. sz\'amjegyek.
\'Altal\'anos alakban ha $b$-vel osztunk, a marad\'ekok az $1$, $2$, $3$, \ldots , $\left(b - 1\right)$ sz\'amok lesznek, a szakasz sz\'amjegyei pedig $\left[\frac{10}{b}\right]$;$\left[\frac{20}{b}\right]$;\ldots;$\left[\frac{10\left(b-1\right)}{b}\right]$ sz\'amok.
De $\frac{10k}{b} = \left[\frac{10k}{b}\right]+\frac{r_k}{b}\qquad \left(k\leq b-1\right)$, ahol az $r_k$ sz\'am a $10k$ sz\'am  $b$-vel val\'o oszt\'as\'anak marad\'eka.

\'Igy $\frac{10}{b} + \frac{20}{b} + \ldots + \frac{10\left(b-1\right)}{b} = \left[\frac{10}{b}\right] + \frac{r_1}{b} + \left[\frac{20}{b}\right] + \frac{r_2}{b} + \ldots + \left[\frac{10\left(b-1\right)}{b}\right] + \frac{r_{b-1}}{b}.$

De {\setlength\arraycolsep{2pt}\begin{eqnarray*}
r_1 + r_2 + \ldots + r_{b-1} &=& 1 + 2 + \ldots + \left(b - 1\right)=\frac{\left(1+b-1\right)\left(b-1\right)}{2}=\frac{b\left(b-1\right)}{2} \\
\frac{10}{b} + \frac{20}{b} + \ldots + \frac{10\left(b-1\right)}{b} &=& \frac{10\left(1+2+\ldots + \left(b-1\right)\right)}{b} = \frac{10b\left(b-1\right)}{2b} = 5\left(b-1\right) \\
\frac{r_1}{b} + \frac{r_2}{b} + \ldots + \frac{r_{b-1}}{b} &=& \frac{1}{b}\cdot \frac{b\left(b-1\right)}{2} = 0,5\left(b-1\right).\end{eqnarray*}}

V\'eg\"ul $\left[\frac{10}{b}\right] + \left[\frac{20}{b}\right] + \ldots + \left[\frac{10\left(b-1\right)}{b}\right] = 5\left(b-1\right)-0,5\left(b-1\right)=4,5\left(b-1\right)$.
A megold\'ast nem befoly\'asolja a szakasz \'es a tizedes vessz\H o k\"oz\"ott esetleg el\H ofordul\'o sz\'amcsoport.

\medskip

\vfill
\end{document}