\documentclass[a4paper,10pt]{article} 
\usepackage[utf8]{inputenc}
\usepackage{t1enc}
\usepackage{graphicx}
\usepackage{amssymb}
\voffset - 20pt
\hoffset - 35pt
\textwidth 450pt
\textheight 650pt 
\frenchspacing 

\pagestyle{empty}
\def\ki#1#2{\hfill {\it #1 (#2)}\medskip}

\def\tg{\, \hbox{tg} \,}
\def\ctg{\, \hbox{ctg} \,}
\def\arctg{\, \hbox{arctg} \,}
\def\arcctg{\, \hbox{arcctg} \,}

\begin{document}
\begin{center} \Large {\em XXI. Nemzetközi Magyar Matematika Verseny} \end{center}
\begin{center} \large{\em Kecskemét, 2012. március 14-18.} \end{center}
\smallskip
\begin{center} \large{\bf 12. osztály} \end{center}
\bigskip 

{\bf 1. feladat: } A tízes számrendszerben háromjegyű pozitív egész számok közül véletlenszerűen választunk egyet. Mennyi annak a valószínűsége, hogy olyan néggyel osztható számot választunk, melynek jegyei páronként különbözőek?

\ki{Tarcsay Tamás}{Szeged}\medskip

{\bf 2. feladat: } Határozzuk meg a
$$ \sqrt{2012}\cdot x^{\log_{2012}x}=x^2$$
egyenlet megoldásai szorzata egészrészének utolsó öt számjegyét!

\ki{Kántor Sándorné}{Debrecen}\medskip

{\bf 3. feladat: } Mutassuk meg, hogy
$$\sin^{2010}x+\cos^{2011}x+\sin^{2012}x\le 2$$
bármely valós $x$ esetén!

\ki{Katz Sándor}{Bonyhád}\medskip

{\bf 4. feladat: } Az $ABC$ egyenlő szárú háromszögben $AC = BC$, 
az $AB$ alap felezőpontja $D$, az $A$ és a $D$ pontból
a $BC$ szakaszra bocsátott merőlegesek talppontja rendre a $BC$ szakasz $E$, illetve $F$ belső pontja. A $DF$ szakasz $G$ felezőpontját a $C$ ponttal összekötő szakasz és az $AF$ szakasz metszéspontja $H$.
Bocsássunk merőlegeseket a $D$ pontból az $AE$ és az $AF$ egyenesekre, a merőlegesek talppontjai
legyenek rendre $K$ és $L$! Bizonyítsuk be, hogy az $AF$, $EH$ és $KL$ egyenesek az $ABC$ háromszöghöz hasonló háromszöget zárnak közre!

\ki{Bíró Bálint}{Eger}\medskip

{\bf 5. feladat: } $A$, $B$, $C$ véges halmazok, amelyekre teljesül, 
hogy $|A|=|B|=|C|=a$ és $|A \cap B \cap C|=b$, ahol $a$ és
$b$ nemnegatív egészek. Adjuk meg $a$ és $b$ függvényeként az $|A \cup B \cup C|$ minimumát és maximumát! ( $|X|$ az $X$ halmaz elemeinek számát jelöli.)

\ki{Gecse Frigyes}{Kisvárda}\medskip

{\bf 6. feladat: } Legyen $a_1=1, a_2=2$ és
$\displaystyle{\frac{1}{a_{n+2}}
=\frac{1}{2}-\sum_{k=1}^{n}\frac{a_{k+2}}{a_{k+1}\cdot(a_k+a_{k+1}+a_{k+2})}}$
($n \ge 1$ egész). 
Adjuk meg $a_n$-t zárt formában, azaz $n$ függvényeként!

\ki{Bencze Mihály}{Brassó}\medskip




\vfill
\end{document}
