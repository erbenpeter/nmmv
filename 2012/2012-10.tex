\documentclass[a4paper,10pt]{article} 
\usepackage[utf8]{inputenc}
\usepackage[a4paper]{geometry}
\usepackage[magyar]{babel}
\usepackage{amsmath}
\usepackage{amssymb}
\frenchspacing 
\pagestyle{empty}
\newcommand{\ki}[2]{\hfill {\it #1 (#2)}\medskip}
\newcommand{\vonal}{\hbox to \hsize{\hskip2truecm\hrulefill\hskip2truecm}}
\newcommand{\degre}{\ensuremath{^\circ}}
\newcommand{\tg}{\mathop{\mathrm{tg}}\nolimits}
\newcommand{\ctg}{\mathop{\mathrm{ctg}}\nolimits}
\newcommand{\arc}{\mathop{\mathrm{arc}}\nolimits}
\begin{document}
\begin{center} \Large {\em XXI. Nemzetközi Magyar Matematika Verseny} \end{center}
\begin{center} \large{\em Kecskemét, 2012. március 14--18.} \end{center}
\smallskip
\begin{center} \large{\bf 10. osztály} \end{center}
\bigskip 

{\bf 1. feladat: } Van-e olyan egész együtthatós $P(x)$ polinom, amelyre 
$P(0)=12, P(1)=20$ és $P(2)=2012$?

\ki{Pintér Ferenc}{Nagykanizsa}\medskip

{\bf 2. feladat: } Határozzuk meg mindazokat a $p$, $q$, $r$ prímszámokat, amelyekre
\[pqr < pq + qr + rp\, !\]

\ki{Oláh György}{Révkomárom}\medskip

{\bf 3. feladat: } Mely $n$ pozitív egész számok esetén lesz az $n^2+n+19$ kifejezés értéke négyzetszám?

\ki{Kacsó Ferenc}{Marosvásárhely}\medskip

{\bf 4. feladat: } Határozzuk meg az
\[E=\frac{2x}{3y+4z}+\frac{3y}{4z+2x}+\frac{4z}{2x+3y}\]
kifejezés legkisebb értékét, ha $x$, $y$ és $z$ pozitív valós számok!

\ki{Kovács Béla}{Szatmárnémeti}\medskip

{\bf 5. feladat: } Az $ABC$ egyenlő szárú háromszögben $AC = BC$, 
az $AB$ alap felezőpontja $D$, az $A$ és a $D$ pontból
a $BC$ szakaszra bocsátott merőlegesek talppontja rendre a $BC$ szakasz $E$, illetve $F$ belső pontja. A $DF$ szakasz $G$ felezőpontját a $C$ ponttal összekötő szakasz és az $AF$ szakasz metszéspontja $H$.
Igazoljuk, hogy a $H$ pont az $AC$ szakasz mint átmérő fölé írt Thalész-körön van!

\ki{Bíró Bálint}{Eger}\medskip

{\bf 6. feladat: } Az első 2012 darab pozitív egész szám mindegyikét átírjuk hármas számrendszerbe. Hány palindrom szám van a kapott 2012 darab hármas számrendszerbeli szám között? (Palindrom számon
olyan pozitív egész számot értünk, amelynek számjegyeit fordított sorrendben írva az eredeti
számot kapjuk vissza.)

\ki{Kosztolányi József}{Szeged}\medskip
\end{document}