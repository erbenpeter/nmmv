\documentclass[a4paper,10pt]{article} 
\usepackage[utf8]{inputenc}
\usepackage{t1enc}
\usepackage{graphicx}
\usepackage{amssymb}
\voffset - 20pt
\hoffset - 35pt
\textwidth 450pt
\textheight 650pt 
\frenchspacing 

\pagestyle{empty}
\def\ki#1#2{\hfill {\it #1 (#2)}\medskip}

\def\tg{\, \hbox{tg} \,}
\def\ctg{\, \hbox{ctg} \,}
\def\arctg{\, \hbox{arctg} \,}
\def\arcctg{\, \hbox{arcctg} \,}

\begin{document}
\begin{center} \Large {\em XXI. Nemzetközi Magyar Matematika Verseny} \end{center}
\begin{center} \large{\em Kecskemét, 2012. március 14-18.} \end{center}
\smallskip
\begin{center} \large{\bf 11. osztály} \end{center}
\bigskip 

{\bf 1. feladat: } Határozzuk meg azokat a pozitív egész számpárokat, amelyek számtani közepe 
1-gyel nagyobb a mértani közepüknél!

\ki{Kallós Béla}{Nyíregyháza}\medskip

{\bf 2. feladat: } Az $ABC$ háromszögben $H$ a $BC$ oldal $C$-hez közelebbi harmadoló pontja, $N$ pedig az $AB$ oldal $B$-hez közelebbi negyedelő pontja. Az $AH$ és $CN$ szakaszok metszéspontja $M$.

a) Milyen arányban osztja az $M$ pont az $AH$ és $CN$ szakaszokat?

b) Hányad része az $ABC$ háromszög területének a $HMNB$ négyszög területe?

\ki{Katz Sándor}{Bonyhád}\medskip

{\bf 3. feladat: } Oldjuk meg a valós számok halmazán a következő egyenletet!
$$3^{2x+1}-(x-1)\cdot 3^x=10x^2+13x+4$$

\ki{Bencze Mihály}{Brassó}\medskip

{\bf 4. feladat: } Az $ABC$ háromszög $AB$ oldalán vegyük fel a $D$ pontot, $AC$ oldalán pedig az $E$ és $F$ pontokat úgy, hogy
$\displaystyle{\frac{AE}{AC}=\frac{CF}{AC}=\frac{AD}{AB}}$
teljesüljön! Az $F$ ponton keresztül húzzunk párhuzamost az $AB$ oldallal,
messe ez a párhuzamos a $BC$ oldalt a $G$ pontban! Mely $D$, $E$, $F$ pontok esetén lesz a 
$DEFG$ négyszög területe a lehető legnagyobb?

\ki{Nemecskó István}{Budapest}\medskip

{\bf 5. feladat: } Mely $n$ pozitív egész számok esetén osztható az 
$1^n+2^n+3^n+4^n+5^n+6^n+7^n+8^n$ összeg 5-tel?

\ki{Oláh György}{Révkomárom}\medskip

{\bf 6. feladat: } Aladár és Béla a következő játékot játsszák: a táblára felírják az 
$1, 2,\dots, 2012$ számokat, melyek
közül felváltva törölnek le egy-egy számot. Aladár kezd. A játék akkor ér véget, amikor két szám
marad a táblán. Ha ezek különbségének abszolútértéke egy előre megadott rögzített pozitív egész $k$ számnál nagyobb prímszám, akkor Béla nyer, egyébként pedig Aladár nyer. Döntsük el, hogy $k$
értékétől függően melyik játékosnak van nyerő stratégiája!

\ki{Borbély József}{Tata}\medskip



\vfill
\end{document}
