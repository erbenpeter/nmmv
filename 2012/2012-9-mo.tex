\documentclass[a4paper,10pt]{article} 
\usepackage[utf8]{inputenc}
\usepackage{graphicx}
\usepackage{amssymb}

\usepackage{wrapfig}
\usepackage{pstricks, pstricks-add}
\voffset - 20pt
\hoffset - 35pt
\textwidth 450pt
\textheight 650pt 
\frenchspacing

\pagestyle{empty}
\def\ki#1#2{\hfill {\it #1 (#2)}\medskip}

\def\tg{\, \hbox{tg} \,}
\def\ctg{\, \hbox{ctg} \,}
\def\arctg{\, \hbox{arctg} \,}
\def\arcctg{\, \hbox{arcctg} \,}

\begin{document}
\begin{center} \Large {\em XXI. Nemzetközi Magyar Matematikaverseny} \end{center}
\begin{center} \large{\em Kecskemét, 2012. március 14--18.} \end{center}
\smallskip
\begin{center} \large{\bf 9. osztály} \end{center}
\bigskip 

{\bf 1. feladat: } A Gumimacik megszervezték a Nemzetközi Gumibogyó Szüreti Fesztivált, ahol minden
résztvevő Gumimaci ugyanannyi üveg idei termésből készült gumibogyó szörpöt kapott ajándékba.
Ha a Szüreti Fesztiválon tízzel kevesebb Gumimaci lett volna jelen, akkor az elkészített
mennyiségből minden résztvevő két üveggel több gumibogyó szörpöt kaphatott volna. Amennyiben
a Szüreti Fesztiválon nyolc Gumimacival többen vettek volna részt, akkor az idén sajtolt
gumibogyó szörp mennyiségből mindannyian egy üveggel kevesebbet kaptak volna. Valójában
hány Gumimaci vett részt a Nemzetközi Gumibogyó Szüreti Fesztiválon, és fejenként hány üveg
gumibogyó szörpöt kapott ajándékba?

\ki{Péics Hajnalka}{Szabadka}\medskip

{\bf 1. feladat megoldása: }
Jelölje $x$ a Nemzetközi Gumibogyó Szüreti Fesztiválon résztvevő Gumimacik
számát, $y$ pedig azoknak a gumibogyó szörppel teli üvegeknek a számát, amennyit minden
Gumimaci ajándékba kapott! A feladat feltételei alapján ekkor a következő egyenletrendszert állíthatjuk fel:

\begin{center}
$$(x-10)(y+2)=xy$$
$$(x+8)(y-1)=xy$$
\rule{3cm}{0.01cm}
$$xy+2x-10y-20=xy$$
$$xy-x+8y-8=xy$$
\rule{4cm}{0.01cm}
$$x-5y=10$$
$$-x+8y=8$$
\rule{2cm}{0.01cm}
\end{center}

Ebből következik, hogy $x=40$ és $y=6$.

A Nemzetközi Gumibogyó Szüreti Fesztiválon tehát $40$ Gumimaci vett részt, és fejenként $6$
üveg gumibogyó szörpöt kaptak ajándékba.
\medskip

\hbox to \hsize{\hskip2truecm\hrulefill\hskip2truecm}
{\bf 2. feladat: } Az $ABCDE$ szabályos ötszög $AD$ és $EB$ átlóinak metszéspontja legyen $S$, az $AC$ és $EB$ szakaszok
metszéspontja $P$, az $AD$ és $EC$ átlók metszéspontja $R$, a $DB$ és $EC$ szakaszok metszéspontja pedig
$Q$! Határozzuk meg az $APQD$ négyszög területét, ha az átlók által meghatározott $ABCDE$
csillagötszög (ötágú csillag) területe 2 egység!


\ki{Nemecskó István}{Budapest}\medskip

{\bf 2. feladat megoldása: }
A $PQDR$ négyszög paralelogramma, tehát átlója felezi a területét:
$$T_{PQD}=\frac{T_{PQDR}}2=T_{QDR}.$$
\begin{wrapfigure}{r}{0.4\textwidth}
\psset{xunit=0.7cm,yunit=0.7cm,algebraic=true,dotstyle=o,dotsize=3pt 0,linewidth=0.8pt,arrowsize=3pt 2,arrowinset=0.25}
\begin{pspicture*}(-3.55,-0.39)(3.57,6.75)
\psline(-2,0)(2,0)
\psdots[dotstyle=*](-2,0)
\rput(-2.4,-0.1){$A$}
\psdots[dotstyle=*](2,0)
\rput(2.3, -0.2){$B$}
\psline(2,0)(3.24,3.8)
\psdot[dotstyle=*](3.24, 3.8)
\rput(3.4, 4.1){$C$}
\psline(3.24,3.8)(0,6.16)
\psdot[dotstyle=*](0,6.16)
\rput(0, 6.5){$D$}
\psline(0,6.16)(-3.24,3.8)
\psdot[dotstyle=*](-3.24, 3.8)
\rput(-3.34, 4.1){$E$}
\psline(-3.24,3.8)(-2,0)
\psline(-2,0)(3.24,3.8)
\psline(2,0)(0,6.16)
\psline(0,6.16)(-2,0)
\psline(3.24,3.8)(-3.24,3.8)
\psline(2,0)(-3.24,3.8)
\psline(0,1.45)(0,6.16)
\psline(0,1.45)(0.76,3.8)
\psline(0,1.45)(-0.76,3.8)
\psdot[dotstyle=*](0,1.45)
\rput(0,1.05){$P$}
\psdot[dotstyle=*](0.76,3.8)
\rput(1, 4.1){$Q$}
\psdot[dotstyle=*](-0.76, 3.8)
\rput(-1, 4.1){$R$}
\psdot[dotstyle=*](-1.24,2.35)
\rput(-1.55, 2.15){$S$}
\end{pspicture*}
\end{wrapfigure}

Az $RES$ háromszög egybevágó a $QDR$ háromszöggel, tehát területük egyenlő:
$$T_{QDR}=T_{RES}.$$
A keresett terület:
$$T_{APQD}=T_{APQD}-T_{PQD}+T_{RES}.$$

Az ábráról látszik, hogy ez a csillagötszög (ötágú csillag) fele, tehát a keresett terület 1
egység.
\bigskip

\hbox to \hsize{\hskip2truecm\hrulefill\hskip2truecm}
{\bf 3. feladat: } Az asztalon egy egyenes mentén 50 zsetont helyeztek el. Aladár és Bea a következő játékot
játssza: felváltva vesznek el a zsetonok közül alkalmanként 3-3 darabot addig, amíg 2 zseton nem
marad. Ha ezek nem szomszédosak, akkor a kezdő játékos győz, ha pedig szomszédosak, akkor a
második játékos a győztes. Kinek van nyerő stratégiája, ha a játékot Bea kezdi?


\ki{Szabó Magda}{Szabadka}\medskip

{\bf 3. feladat megoldása: }
Aladárnak van nyerő stratégiája. Ez a következő:

Valamelyik irányból megszámozza és párba állítja a zsetonokat, az $(1; 2)$, $(3; 4)$, \dots , $(49; 50)$
párokba, majd Bea választása után ő az alábbiak szerint kontrázik:
\begin{itemize}
\item[--] ha Bea választ 3 zsetont 3 különböző párból, akkor kiveszi mindhárom kiválasztott párból a
másodikat;
\item[--] ha Bea egy pár két zsetonját és egy tetszőleges harmadikat választ, akkor Aladár a harmadik
zseton párját és egy tetszőleges pár két zsetonját veszi el.
\end{itemize}

Mivel $50=6\cdot 8+2$, ezért a nyolcadik lépés után két szomszédos zseton marad.

Általában is igaz, hogy a másodikként elvevő játékosnak van nyerő stratégiája, ha a zsetonok
száma $6k+2$.

\medskip


\hbox to \hsize{\hskip2truecm\hrulefill\hskip2truecm}
{\bf 4. feladat: } Határozzuk meg azokat a pozitív egész $n$ számokat, amelyekre a $2^n-1$ és a $2^n+1$ számok közül
legalább az egyik osztható 7-tel!


\ki{Kántor Sándor}{Debrecen}\medskip

{\bf 4. feladat megoldása: }
Keressük a megfelelő $n$-eket a $3$-as maradékaik szerint megkülönböztetve, azaz
$n=3k+r$ alakban, ahol $r\in\{0;1;2\}$!

$2^n-1=2^r\cdot\left(8^k-1\right)+2^r-1$. Mivel $8^k-1$ osztható $7$-tel, ezért $2^n-1$ $7$-es maradéka egyenlő
$2^r-1$ $7$-es maradékával, ami lehet $0$, $1$ vagy $3$. $2^n-1$ tehát pontosan akkor osztható $7$-tel, ha
$n$ osztható $3$-mal.

$2^n+1=2^r\cdot\left(8^k-1\right)+2^r+1$. Mivel $8^k-1$ osztható $7$-tel, ezért $2^n+1$ $7$-es maradéka egyenlő $2^r+1$ $7$-es maradékával, ami lehet $2$, $3$ vagy $5$. $2^n+1$ tehát egyetlen pozitív egész $n$ esetén sem osztható $7$-tel.

Kaptuk, hogy a $2^n-1$ és $2^n+1$ számok közül legalább az egyik (ez mindig a $2^n-1$)
pontosan a $3$-mal osztható pozitív egész $n$-ek esetén osztható $7$-tel.

\medskip


\hbox to \hsize{\hskip2truecm\hrulefill\hskip2truecm}
{\bf 5. feladat: } Az $ABC$ háromszög $AB$ és $AC$ oldalainak belsejében úgy vesszük fel rendre a $D$ és $E$ pontokat,
hogy $BD=CE$ teljesüljön. Legyen $F$ és $G$ rendre a $BC$ és $DE$ szakaszok felezőpontja, valamint
legyen $M$ az $FG$ egyenesnek az $AC$ oldallal vett metszéspontja! Határozzuk meg az $AM$ szakasz
hosszát az $AB$ és $AC$ oldalak hosszának függvényében!

\ki{Olosz Ferenc}{Szatmárnémeti}\medskip

{\bf 5. feladat I. megoldása: }
A feladat lényegén nem változtat, ha feltételezzük, hogy $AB<AC$.

\begin{wrapfigure}{r}{0.35\textwidth}
\psset{xunit=0.72cm,yunit=0.72cm,algebraic=true,dotstyle=o,dotsize=3pt 0,linewidth=0.8pt,arrowsize=3pt 2,arrowinset=0.25}
\begin{pspicture*}(-1.66,-0.35)(6.35,5.64)
\psline[linewidth=1.5pt](1.66,3.17)(0,0)
\psline[linewidth=1.5pt](0,0)(6,0)
\psline[linewidth=1.5pt](6,0)(1.66,3.17)
\psline(0.55,1.04)(5.05,0.69)
\psline(3,0)(2.39,2.65)
\rput[tl](0.4,0.75){$d$}
\rput[tl](5,0.45){$d$}
\rput[tl](-0.67,5.26){$d$}
\psline[linewidth=1pt, linestyle=dashed,dash=1pt 1pt](1.66,3.17)(-1.23,5.29)
\psline[linewidth=1pt, linestyle=dashed,dash=1pt 1pt](0,0)(-1.23,5.29)
\psline[linewidth=1pt, linestyle=dashed,dash=1pt 1pt](0.55,1.04)(-0.28,4.6)
\begin{scriptsize}
\psdots[dotstyle=*](1.66,3.17)
\rput[bl](1.75,3.26){$A$}
\psdots[dotstyle=*](0,0)
\rput[bl](-0.45,-0.15){$B$}
\psdots[dotstyle=*](6,0)
\rput[bl](5.75,0.25){$C$}
\psdots[dotstyle=*](0.55,1.04)
\rput[bl](0.1,1){$D$}
\psdots[dotstyle=*](3,0)
\rput[bl](3.07,0.12){$F$}
\psdots[dotstyle=*](5.05,0.69)
\rput[bl](5.14,0.81){$E$}
\psdots[dotstyle=*](2.8,0.87)
\rput[bl](2.88,0.98){$G$}
\psdots[dotstyle=*](2.39,2.65)
\rput[bl](2.47,2.76){$M$}
\psdots[dotstyle=*](-0.28,4.6)
\rput[bl](-0.2,4.71){$D'$}
\psdots[dotstyle=*](-1.23,5.29)
\rput[bl](-1.16,5.41){$B'$}
\end{scriptsize}
\end{pspicture*}
\end{wrapfigure}

Az $AC$ oldal $A$-n túli meghosszabbításán felvesszük a $B'$ és $D'$ pontokat úgy, hogy $AB'=AB$ és
$AD'=AD$ teljesüljön. Könnyen látható, hogy $B'D'=BD=CE=d$ és $BB'$ párhuzamos $DD'$-vel.

Ha $M'$ az $ED'$ szakasz felezőpontja, akkor $M'$ a $CB'$ szakasz felezőpontja is.

Az $EDD'$ háromszögben $GM'$ középvonal, tehát $GM'$ párhuzamos $DD'$-vel.

A $CBB'$ háromszögben $FM'$ középvonal, tehát $FM'$ párhuzamos $BB'$-vel.

Mivel $GM'$ párhuzamos $DD'$-vel, $FM'$ párhuzamos $BB'$-vel és $BB'$ párhuzamos $DD'$-vel,
ezért $F$, $G$, $M'$ egy egyenesen elhelyezkedő pontok, tehát $M'$ egybeesik az $M$ ponttal.
$$ME=MD'=MA+AD'=AM+AD=AM+AB-BD=AM+AB-d$$
és
$$AC=AM+ME+EC=AM+(AM+AB-d)+d,$$
ahonnan $AC=2AM+AB$ és $AM=\frac{AC-AB}2.$ Tehát $AM=\frac{|AB-AC|}2.$

\medskip
{\bf 5. feladat II. megoldása: }
A feladat lényegén nem változtat, ha feltételezzük, hogy $AB<AC$.

\begin{wrapfigure}{r}{0.35\textwidth}
\psset{xunit=0.72cm,yunit=0.72cm,algebraic=true,dotstyle=o,dotsize=3pt 0,linewidth=0.8pt,arrowsize=3pt 2,arrowinset=0.25}
\begin{pspicture*}(-0.59,-0.4)(6.52,4.45)
\psline[linewidth=1.5pt](1.66,3.17)(0,0)
\psline[linewidth=1.5pt](0,0)(6,0)
\psline[linewidth=1.5pt](6,0)(1.66,3.17)
\psline(0.55,1.04)(5.05,0.69)
\psline(3,0)(2.39,2.65)
\psline[linewidth=1pt, linestyle=dashed,dash=1pt 1pt](1.66,3.17)(2.08,3.97)
\psline[linewidth=1pt, linestyle=dashed,dash=1pt 1pt](2.08,3.97)(2.39,2.65)
\begin{scriptsize}
\psdots[dotstyle=*](1.66,3.17)
\rput[bl](1.4,3.26){$A$}
\psdots[dotstyle=*](0,0)
\rput[bl](-0.41,-0.15){$B$}
\psdots[dotstyle=*](6,0)
\rput[bl](5.75,0.25){$C$}
\psdots[dotstyle=*](0.55,1.04)
\rput[bl](0.1,1){$D$}
\psdots[dotstyle=*](3,0)
\rput[bl](3.07,0.12){$F$}
\psdots[dotstyle=*](5.05,0.69)
\rput[bl](5.14,0.81){$E$}
\psdots[dotstyle=*](2.8,0.87)
\rput[bl](2.88,0.98){$G$}
\psdots[dotstyle=*](2.39,2.65)
\rput[bl](2.47,2.76){$M$}
\psdots[dotstyle=*](2.08,3.97)
\rput[bl](2.08, 4.1){$L$}
\end{scriptsize}
\end{pspicture*}
\end{wrapfigure}

Legyen $L$ az $FG$ egyenes és az $AB$ egyenes metszéspontja.

Az $ABC$ és $ADE$ háromszögekben alkalmazzuk Menelaosz tételét az $FG$ szelőre nézve:
$$\frac{LB}{LA}\cdot\frac{MA}{MC}\cdot\frac{FC}{FB}=1 \quad \textrm{és} \quad \frac{LD}{LA}\cdot\frac{MA}{ME}\cdot\frac{GE}{GD}=1.$$

Figyelembe véve az $FB=FC$ és $GD=GE$ egyenlőségeket, kapjuk, hogy
$$\frac{LB}{LA}\cdot\frac{MA}{MC}=1 \quad \textrm{és} \quad \frac{LD}{LA}\cdot\frac{MA}{ME}=1.\qquad (*)$$

$(*)$-ból következik, hogy $\frac{LB}{LA}=\frac{MC}{MA}$ és $\frac{LD}{LA}=\frac{ME}{MA}$, ahonnan az aránypárok megfelelő oldalait egymásból kivonva kapjuk,
hogy $\frac{LB-LD}{LA}=\frac{MC-ME}{MA}$, vagyis $\frac{BD}{LA}=\frac{CE}{MA}$.
Mivel $BD=CE$, ezért $LA=MA$.

A $(*)$ egyenlőségek bal oldalainak egyenlőségéből következik $\frac{LB}{MC}=\frac{LD}{ME}$, vagyis $\frac{LB}{LD}=\frac{MC}{ME}$, amelyből származtatjuk az $\frac{LB}{LB-LD}=\frac{MC}{MC-ME}\Leftrightarrow\frac{LB}{BD}=\frac{MC}{CE}$ aránypárt.

Mivel $BD=CE$, ezért $LB=MC$, amely felírható $LA+AB=AC-AM$ alakban.
Figyelembe véve az $LA=MA$ egyenlőséget, kapjuk, hogy $AM=\frac{AC-AB}2$.
Tehát $AM=\frac{|AB-AC|}2$.
\medskip


\hbox to \hsize{\hskip2truecm\hrulefill\hskip2truecm}
{\bf 6. feladat: } Egy valós számokból álló $a_1, a_2, a_3,\dots, a_n$ véges sorozat tagjaira teljesül, hogy bármely 5 egymást
követő tagjának összege negatív, és bármely 8 egymást követő tagjának összege pozitív. Legfeljebb
hány tagja lehet egy ilyen sorozatnak?

\ki{Kallós Béla}{Nyíregyháza}\medskip

{\bf 6. feladat megoldása: }
Megmutatjuk, hogy 12 tagja már nem lehet a sorozatnak.
Vizsgáljuk meg a következő táblázatot:
\begin{center}
\begin{tabular}{|c|c|c|c|c|c|c|c|}
\hline
$a_1$ & $a_2$ & $a_3$ & $a_4$ & $a_5$ & $a_6$ & $a_7$ & $a_8$ \\
\hline
$a_2$ & $a_3$ & $a_4$ & $a_5$ & $a_6$ & $a_7$ & $a_8$ & $a_9$ \\
\hline
$a_3$ & $a_4$ & $a_5$ & $a_6$ & $a_7$ & $a_8$ & $a_9$ & $a_{10}$ \\
\hline
$a_4$ & $a_5$ & $a_6$ & $a_7$ & $a_8$ & $a_9$ & $a_{10}$ & $a_{11}$ \\
\hline
$a_5$ & $a_6$ & $a_7$ & $a_8$ & $a_9$ & $a_{10}$ & $a_{11}$ & $a_{12}$ \\
\hline
\end{tabular}
\end{center}

A feladat egyik feltétele miatt minden oszlop összege negatív, azaz a táblázatban szereplő
összes szám összege negatív.

A másik feltétel miatt azonban minden sor összege pozitív, azaz az összes szám összege
pozitív.

Ez az ellentmondás azt jelenti, hogy 12 tagja nem lehet a sorozatnak.
11 tagja viszont már lehet a sorozatnak, amint ezt a következő példa is mutatja:
$$5; –8; 5; 5; –8; 5; –8; 5; 5; –8; 5.$$
\medskip

\vfill
\end{document}


