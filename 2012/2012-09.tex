\documentclass[a4paper,10pt]{article} 
\usepackage[utf8]{inputenc}
\usepackage{t1enc}
\usepackage{graphicx}
\usepackage{amssymb}
\voffset - 20pt
\hoffset - 35pt
\textwidth 450pt
\textheight 650pt 
\frenchspacing 

\pagestyle{empty}
\def\ki#1#2{\hfill {\it #1 (#2)}\medskip}

\def\tg{\, \hbox{tg} \,}
\def\ctg{\, \hbox{ctg} \,}
\def\arctg{\, \hbox{arctg} \,}
\def\arcctg{\, \hbox{arcctg} \,}

\begin{document}
\begin{center} \Large {\em XXI. Nemzetközi Magyar Matematika Verseny} \end{center}
\begin{center} \large{\em Kecskemét, 2012. március 14-18.} \end{center}
\smallskip
\begin{center} \large{\bf 9. osztály} \end{center}
\bigskip 

{\bf 1. feladat: } A Gumimacik megszervezték a Nemzetközi Gumibogyó Szüreti Fesztivált, ahol minden
résztvevő Gumimaci ugyanannyi üveg idei termésből készült gumibogyó szörpöt kapott ajándékba.
Ha a Szüreti Fesztiválon tízzel kevesebb Gumimaci lett volna jelen, akkor az elkészített
mennyiségből minden résztvevő két üveggel több gumibogyó szörpöt kaphatott volna. Amennyiben
a Szüreti Fesztiválon nyolc Gumimacival többen vettek volna részt, akkor az idén sajtolt
gumibogyó szörp mennyiségből mindannyian egy üveggel kevesebbet kaptak volna. Valójában
hány Gumimaci vett részt a Nemzetközi Gumibogyó Szüreti Fesztiválon, és fejenként hány üveg
gumibogyó szörpöt kapott ajándékba?


\ki{Péics Hajnalka}{Szabadka}\medskip

{\bf 2. feladat: } Az $ABCDE$ szabályos ötszög $AD$ és $EB$ átlóinak metszéspontja legyen $S$, az $AC$ és $EB$ szakaszok
metszéspontja $P$, az $AD$ és $EC$ átlók metszéspontja $R$, a $DB$ és $EC$ szakaszok metszéspontja pedig
$Q$! Határozzuk meg az $APQD$ négyszög területét, ha az átlók által meghatározott $ABCDE$
csillagötszög (ötágú csillag) területe 2 egység!

\ki{Nemecskó István}{Budapest}\medskip

{\bf 3. feladat: } Az asztalon egy egyenes mentén 50 zsetont helyeztek el. Aladár és Bea a következő játékot
játssza: felváltva vesznek el a zsetonok közül alkalmanként 3-3 darabot addig, amíg 2 zseton nem
marad. Ha ezek nem szomszédosak, akkor a kezdő játékos győz, ha pedig szomszédosak, akkor a
második játékos a győztes. Kinek van nyerő stratégiája, ha a játékot Bea kezdi?

\ki{Szabó Magda}{Szabadka}\medskip

{\bf 4. feladat: } Határozzuk meg azokat a pozitív egész $n$ számokat, amelyekre a $2^n-1$ és a $2^n+1$ számok közül
legalább az egyik osztható 7-tel!

\ki{Kántor Sándor}{Debrecen}\medskip

{\bf 5. feladat: } Az $ABC$ háromszög $AB$ és $AC$ oldalainak belsejében úgy vesszük fel rendre a $D$ és $E$ pontokat,
hogy $BD=CE$ teljesüljön. Legyen $F$ és $G$ rendre a $BC$ és $DE$ szakaszok felezőpontja, valamint
legyen $M$ az $FG$ egyenesnek az $AC$ oldallal vett metszéspontja! Határozzuk meg az $AM$ szakasz
hosszát az $AB$ és $AC$ oldalak hosszának függvényében!


\ki{Olosz Ferenc}{Szatmárnémeti}\medskip

{\bf 6. feladat: } Egy valós számokból álló $a_1, a_2, a_3,\dots, a_n$ véges sorozat tagjaira teljesül, hogy bármely 5 egymást
követő tagjának összege negatív, és bármely 8 egymást követő tagjának összege pozitív. Legfeljebb
hány tagja lehet egy ilyen sorozatnak?


\ki{Kallós Béla}{Nyíregyháza}\medskip

\vfill
\end{document}
