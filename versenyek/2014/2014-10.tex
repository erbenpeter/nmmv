\documentclass[a4paper,10pt]{article} 
\usepackage[utf8]{inputenc}
\usepackage[a4paper]{geometry}
\usepackage[magyar]{babel}
\usepackage{t1enc}
\usepackage{amsmath}
\usepackage{amssymb}
\frenchspacing 
\pagestyle{empty}
\newcommand{\ki}[2]{\hfill {\it #1 (#2)}\medskip}
\newcommand{\vonal}{\hbox to \hsize{\hskip2truecm\hrulefill\hskip2truecm}}
\newcommand{\degre}{\ensuremath{^\circ}}
\newcommand{\tg}{\mathop{\mathrm{tg}}\nolimits}
\newcommand{\ctg}{\mathop{\mathrm{ctg}}\nolimits}
\newcommand{\arc}{\mathop{\mathrm{arc}}\nolimits}
\begin{document}
\begin{center} \Large {\em 23. Nemzetközi Magyar Matematika Verseny} \end{center}
\begin{center} \large{\em Csíkszereda, 2014. március 12-16.} \end{center}
\smallskip
\begin{center} \large{\bf 10. osztály} \end{center}
\bigskip

{\bf 1. feladat: } Oldd meg a prímszámok halmazán a $$
3x^{2}-y^{2}=22y-12x$$ egyenletet!


\ki{Olosz Ferenc}{Szatmárnémeti}\medskip

{\bf 2. feladat: } Négy Tudós Matematikus egy egyenlő szárú trapéz alakú birtokon
él, házaik a trapéz csúcsaira épültek. A trapéz hosszabb alapjának
hossza $a$, az alapon fekvő szögek nagysága $50^\circ $, az átlók
által bezárt szög pedig $76^\circ$. A Tudósok szeretik a szabályos
dolgokat, így elhatározták, hogy olyan kutat építenek, amely
mindannyiuk házától ugyanolyan távolságra helyezkedik el. Milyen
távolságra kell építeniük házaiktól a kutat? Vajon a kút a
birtokukon lesz-e?


\ki{dr. Péics Hajnalka}{Szabadka}\medskip

{\bf 3. feladat: } Oldd meg a pozitív valós számok halmazán a $$
\displaystyle 2^{4x+1}+2^{\frac{1}{2x^{2}}}=12 $$
 egyenletet!

\ki{Koczinger Éva és Kovács Béla}{Szatmárnémeti}\medskip

{\bf 4. feladat: } Adott az $ABC$ háromszög, amelyben feltételezzük, hogy $AB < BC <
AC$. A $BC$ oldalon felvesszük a $B'$ pontot úgy, hogy $CB'=AB.$
Hasonlóan felvesszük az $AC$ oldalon az $A'$ és a $C'$ pontot
úgy, hogy $CA'=AB$ és $AC'=BC.$ Jelöljük az $AA',$ $BB',$
illetve $CC'$ szakaszok felezőpontját rendre $D$-vel, $E$-vel
és $F$-fel. Bizonyítsd be, hogy ha ${A_1}$ a $BC$ szakasz,
${B_1}$ az $AC$ szakasz és ${C_1}$ az $AB$ szakasz felezőpontja,
valamint $\{G\}=A_1D\cap AB,$ $\{H\}=B_1E\cap AB$ és
$\{I\}=C_1F\cap BC,$ akkor:
\begin{itemize}
\item[a)] $BI=GH$; \item[b)] az ${A_1}D,$ ${C_1}F$ és ${B_1}E$
egyeneseknek van közös pontja; \item[c)] ha $J$ az $ABC$
háromszögbe, $K$ az ${A_1}{B_1}{C_1}$ háromszögbe írt kör
középpontja, $L$ pedig az $ABC$ háromszög súlypontja, akkor a $J,$
$K$ és $L$ pontok egy egyenesen helyezkednek el és $JL = 2KL.$
\end{itemize}

\ki{Pálhegyi Farkas László}{Nagyvárad}\medskip

{\bf 5. feladat: } Bizonyítsd be, hogy az összes $\frac{1}{m\cdot n}$
alakú szám összege nem egész szám, ahol $1\leq
m<n\leq2014$, illetve $m$ és $n$ természetes számok.


\ki{dr. Kántor Sándor}{Debrecen}\medskip

{\bf 6. feladat: } a) Határozd meg a síknak egységoldalú szabályos
hatszögekkel, egy\-ség\-oldalú négyzetekkel és
egységoldalú szabályos tizenkétszögekkel való összes
szabályos lefödését! Egy lefödés azt jelenti, hogy a
sokszögek hézag és átfödés nélkül (egyrétűen)
lefödik a síkot. A lefödés szabályos, ha léteznek olyan
$a,b,c$ nullától különböző természetes számok,
amelyekre minden keletkező csúcs körül pontosan $a$ darab
hatszög, $b$ darab négyzet és $c$ darab tizenkétszög van,
valamilyen rögzített sorrendben.

b) Bizonyítsd be, hogy az előbbi hatszögekkel,
négyzetekkel, tizenkétszögekkel, valamint egységoldalú
szabályos háromszögek\-kel létre lehet hozni olyan, nem
feltétlenül sza\-bá\-lyos lefödést, amelyben mind a négy
típusú alakzatot végtelen sokszor hasz\-náljuk, és
amelyben létezik végtelen sok páronként különböző
mintázat, amely véges sokszor jelenik meg! (Mintázat alatt a
lefödés véges sok sokszöge által meghatározott
összefüggő alakzatot értünk.)

\ki{Zsombori Gabriella}{Csíkszereda}

\ki{dr. András Szilárd, dr. Lukács Andor}{Kolozsvár}\medskip
\end{document}