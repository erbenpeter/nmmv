\documentclass[a4paper,10pt]{article} 
\usepackage[utf8]{inputenc}
\usepackage[a4paper]{geometry}
\usepackage[magyar]{babel}
\usepackage{t1enc}
\usepackage{amsmath}
\usepackage{amssymb}
\usepackage{caption}
\usepackage{pgf,tikz}
\usepackage{pstricks-add}
\frenchspacing 
\pagestyle{empty}
\allowdisplaybreaks
\newcommand{\ki}[2]{\hfill {\it #1 (#2)}\medskip}
\newcommand{\vonal}{\hbox to \hsize{\hskip2truecm\hrulefill\hskip2truecm}}
\newcommand{\degre}{\ensuremath{^\circ}}
\newcommand{\tg}{\mathop{\mathrm{tg}}\nolimits}
\newcommand{\ctg}{\mathop{\mathrm{ctg}}\nolimits}
\newcommand{\arc}{\mathop{\mathrm{arc}}\nolimits}
\renewcommand{\vec}[1]{\mathbf{#1}}
\begin{document}
\begin{center} \Large {\em XXI. Nemzetközi Magyar Matematikaverseny} \end{center}
\begin{center} \large{\em Kecskemét, 2012. március 14-18.} \end{center}
\smallskip
\begin{center} \large{\bf 12. osztály} \end{center}
\bigskip

{\bf 1. feladat: } A tízes számrendszerben háromjegyű pozitív egész számok közül véletlenszerűen választunk egyet. Mennyi annak a valószínűsége, hogy olyan néggyel osztható számot választunk, melynek jegyei páronként különbözőek?

\ki{Tarcsay Tamás}{Szeged}\medskip

{\bf Megoldás: } A tízes számrendszerben háromjegyű számok száma 900.
Egy pozitív egész szám akkor és csak akkor osztható 4-gyel, ha az utolsó két jegyéből képzett
kétjegyű szám osztható 4-gyel.

25 különböző 4-gyel osztható végződés van, ezek közül a 00, 44 és 88 nem felel meg, mert
ismétlődő számjegy van benne, így marad 22 darab ,,jó'' végződés.

Ezek között 6 tartalmaz 0 számjegyet (04, 08, 20, 40, 60, 80). Ezek elé még 8 különböző
számjegyet írhatunk, így 48 darab ,,kedvező'' számot kaphatunk.

A fennmaradó 16 végződés elé 7 különböző számot írhatunk (a 0-t nem), így 112 darab
,,kedvező'' számot kaphatunk.
A ,,kedvező'' számok száma: $48 + 112 = 160$.

A keresett valószínűség: $\displaystyle{P=
\frac{160}{900}=\frac{8}{45}\approx 0{,}18}$.

\medskip

\vonal

{\bf 2. feladat: } Határozzuk meg a
$$ \sqrt{2012}\cdot x^{\log_{2012}x}=x^2$$
egyenlet megoldásai szorzata egészrészének utolsó öt számjegyét!

\ki{Kántor Sándorné}{Debrecen}\medskip

{\bf Megoldás: } A logaritmus definíciója miatt $x > 0$. Vegyük az egyenlet mindkét oldalának 2012-es alapú logaritmusát, és használjuk fel a logaritmus azonosságait:
$$
\frac{1}{2}\log_{2012} 2012 + \log^2_{2012} x = 2 \log_{2012} x.
$$

A $\log_{2012} x = a$ jelölés bevezetésével kapjuk, hogy 
$a^2-2a+\frac{1}{2}=0$. 

Innen $a_1= \log_{2012} x_1 = 1+\sqrt{\frac{1}{2}}$
és 
$a_2= \log_{2012} x_2 = 1-\sqrt{\frac{1}{2}}$.

Mindkét megoldás pozitív és összegük 2.
$\log_{2012}\left(x_1\cdot x_2\right)=
\log_{2012} x_1 + \log_{2012} x_2 = 2$ miatt
$$x_1 \cdot x_2 = 2012^2 = (2000+12)^2 = 2000^2+48000+144= 4~048~144,$$
vagyis a szorzat utolsó öt számjegye: $48~144$.

\medskip

\vonal


\newpage
{\bf 3. feladat: } Mutassuk meg, hogy
$$\sin^{2010}x+\cos^{2011}x+\sin^{2012}x\le 2$$
bármely valós $x$ esetén!

\ki{Katz Sándor}{Bonyhád}\medskip

{\bf Megoldás: } Bármely 1-nél nem nagyobb abszolútértékű valós $z$ és bármely $n \ge 2$ esetén
teljesül a $z^n \le z^2$ összefüggés.

Mivel $\left|\sin x\right|\le 1$ és $\left|\cos x \right|\le 1$, ezért bármely valós $x$ esetén
$$\sin^{2010} x + \cos^{2011} x + \sin^{2012} x \le 
\sin^2 x + \cos^2 x + \sin^{2012} x \le 1 + 1 = 2.$$
Egyenlőség akkor és csak akkor teljesül, ha 
$x=\frac{\pi}{2}+k\pi$, ($k$ egész szám).

\medskip

\vonal


{\bf 4. feladat: } Az $ABC$ egyenlő szárú háromszögben $AC = BC$, 
az $AB$ alap felezőpontja $D$, az $A$ és a $D$ pontból
a $BC$ szakaszra bocsátott merőlegesek talppontja rendre a $BC$ szakasz $E$, illetve $F$ belső pontja. A $DF$ szakasz $G$ felezőpontját a $C$ ponttal összekötő szakasz és az $AF$ szakasz metszéspontja $H$.
Bocsássunk merőlegeseket a $D$ pontból az $AE$ és az $AF$ egyenesekre, a merőlegesek talppontjai
legyenek rendre $K$ és $L$! Bizonyítsuk be, hogy az $AF$, $EH$ és $KL$ egyenesek az $ABC$ háromszöghöz hasonló háromszöget zárnak közre!

\ki{Bíró Bálint}{Eger}\medskip

{\bf Megoldás: } Az állítás bizonyításához készített ábrán az $AC$ szakasz Thalész-körét $k$, a
$CBA \sphericalangle = CAB \sphericalangle$ szögeket pedig $\beta$ jelöli.

\begin{center}
\newrgbcolor{qqwuqq}{0 0.39 0}
\psset{xunit=2.0cm,yunit=2.0cm,algebraic=true,dimen=middle,dotstyle=o,dotsize=3pt 0,linewidth=0.8pt,arrowsize=3pt 2,arrowinset=0.25}
\begin{pspicture*}(-1.38,-0.4)(4.26,4.35)
\pspolygon[hatchcolor=black,fillstyle=hlines,hatchangle=95.0,hatchsep=0.09](1.9,0.43)(2.41,-0.17)(2.74,0.62)
\psline(0,0)(4,0)
\psline(4,0)(2,3.9)
\psline(2,3.9)(0,0)
\psline(2,0)(3.58,0.81)
\psline(3.17,1.62)(0,0)
\psline(2,3.9)(2,0)
\pscircle(1,1.95){4.38}
\psline(0,0)(3.58,0.81)
\psline(2,0)(1.9,0.43)
\psline(2,0)(1.58,0.81)
\psline(2.74,0.62)(2,0)
\psline(2,3.9)(2.79,0.41)
\psline(1.58,0.81)(2.41,-0.17)
\psline(3.17,1.62)(2.41,-0.17)
\parametricplot{2.0446476632896573}{3.615443990084554}{0.13*cos(t)+3.17|0.13*sin(t)+1.62}
\psellipse*[linewidth=0.4pt](3.09,1.65)(0.01,0.01)
\pscustom[linewidth=0.4pt,linecolor=darkgray]{
\parametricplot{-2.667741317095033}{-1.0969449903001363}{0.13*cos(t)+3.58|0.13*sin(t)+0.81}
\lineto(3.58,0.81)\closepath}
\psellipse*[linewidth=0.4pt,linecolor=darkgray](3.56,0.74)(0.01,0.01)
\pscustom[linewidth=0.4pt,linecolor=darkgray]{
\parametricplot{-2.667741317095033}{-1.0969449903001363}{0.13*cos(t)+1.58|0.13*sin(t)+0.81}
\lineto(1.58,0.81)\closepath}
\psellipse*[linewidth=0.4pt,linecolor=darkgray](1.56,0.74)(0.01,0.01)
\pscustom[linewidth=0.4pt,linecolor=darkgray]{
\parametricplot{-2.9187440037380674}{-1.347947676943171}{0.13*cos(t)+1.9|0.13*sin(t)+0.43}
\lineto(1.9,0.43)\closepath}
\psellipse*[linewidth=0.4pt,linecolor=darkgray](1.86,0.37)(0.01,0.01)
\pscustom[linewidth=0.4pt,linecolor=darkgray]{
\parametricplot{1.5707963267948966}{3.141592653589793}{0.13*cos(t)+2|0.13*sin(t)+0}
\lineto(2,0)\closepath}
\psellipse*[linewidth=0.4pt,linecolor=darkgray](1.94,0.06)(0.01,0.01)
\pscustom[linecolor=qqwuqq,fillcolor=qqwuqq,fillstyle=solid,opacity=0.1]{
\parametricplot{2.0446476632896573}{3.1415926535897936}{0.27*cos(t)+4|0.27*sin(t)+0}
\lineto(4,0)\closepath}
\psline(1.9,0.43)(2.41,-0.17)
\psline(2.41,-0.17)(2.74,0.62)
\psline(2.74,0.62)(1.9,0.43)
\begin{scriptsize}
\psdots[dotstyle=*](0,0)
\rput[bl](-0.13,-0.17){$A$}
\psdots[dotstyle=*](4,0)
\rput[bl](4.08,-0.17){$B$}
\psdots[dotstyle=*](2,3.9)
\rput[bl](2.03,3.95){$C$}
\psdots[dotstyle=*](2,0)
\rput[bl](1.97,-0.19){$D$}
\psdots[dotstyle=*](3.17,1.62)
\rput[bl](3.2,1.68){$E$}
\psdots[dotstyle=*](3.58,0.81)
\rput[bl](3.62,0.86){$F$}
\rput[bl](-0.1,3.64){$k$}
\psdots[dotstyle=*](1.58,0.81)
\rput[bl](1.62,0.9){$K$}
\psdots[dotstyle=*](1.9,0.43)
\rput[bl](1.87,0.52){$L$}
\psdots[dotstyle=*](2.74,0.62)
\rput[bl](2.57,0.7){$H$}
\psdots[dotstyle=*](2.79,0.41)
\rput[bl](2.87,0.3){$G$}
\psdots[dotstyle=*](2.41,-0.17)
\rput[bl](2.51,-0.21){$M$}
\rput[bl](3.8,0.04){\qqwuqq{$\beta$}}
\end{scriptsize}
\end{pspicture*}
\end{center}

Mivel $ABE \sphericalangle = \beta$, ezért az $ABE$ derékszögű háromszögben $BAE \sphericalangle = 90 ^\circ  - \beta$, így a $DAK$
derékszögű háromszögben $ADK \sphericalangle = \beta$, továbbá az egybevágó $ACD$ és $BCD$ derékszögű
háromszögekben $ACD \sphericalangle = BCD \sphericalangle = 90 ^\circ- \beta$.

Az $A$, $D$, $L$, $K$ pontok az $AD$ szakasz fölé rajzolt Thalész-körre illeszkednek, ezek a pontok
tehát egy húrnégyszög csúcsai, ebből a kerületi szögek tétele miatt $ADK \sphericalangle = ALK \sphericalangle = \beta$ 
következik.

Az $AF$, $EH$ és $KL$ egyenesek által közrezárt $LMH$ háromszögben azonban $HLM \sphericalangle = \beta$, hiszen
$ALK \sphericalangle$ és $HLM \sphericalangle$ csúcsszögek.

A továbbiakban belátjuk, hogy a $H$ pont illeszkedik $k$-ra. Ehhez először igazoljuk, hogy az $AF$
egyenes merőleges a $CG$ egyenesre.

A $DF$ egyenes párhuzamos $AE$-vel, hiszen mindkettő merőleges $BC$-re, és mivel $D$ az $AB$
felezőpontja, ezért $DF$ középvonala az $ABE$ háromszögnek, és így $F$ felezőpontja az $EF$
szakasznak. A megfelelő szögek egyenlősége miatt az $ABE$ és $DBF$ háromszögek hasonlók (a
hasonlóság aránya $2:1$).

A $DF$ szakasz a $BCD$ derékszögű háromszög átfogójához tartozó magassága, és mint
ismeretes, ez a magasság két hasonló háromszögre bontja az eredeti háromszöget, eszerint a
$DBF$ és $CDF$ háromszögek hasonlók.
A hasonlóság tranzitív tulajdonsága miatt ezért az $ABE$ és a $CDF$ háromszögek is hasonlók, a
megfelelő oldalak aránya tehát egyenlő, azaz $\frac{AE}{EB}
=\frac{CF}{FD}$.

Mivel azonban $FD = 2 \cdot  FG$ és $EB = 2 \cdot  EF$, ezért
\begin{equation} \label{eq:geo}
\frac{AE}{EF} = \frac{CF}{FG}
\end{equation}

Az (\ref{eq:geo}) egyenlőség szerint az $AEF$ és $CFG$ háromszögekben két-két oldal aránya megegyezik,
ugyanakkor ezekben a háromszögekben az $AE$; $EF$ illetve $CF$; $FG$ oldalak derékszöget zárnak
be, így tehát az $AEF$ és $CFG$ háromszögek hasonlók.

Hasonló háromszögekben a megfelelő szögek egyenlők, így $FAE \sphericalangle = GCF \sphericalangle = HCF \sphericalangle$ (az
ábrán két ívvel jelölt szögek).

Az $A$ és $C$ pontok az $EH$ egyenes ugyanazon oldalán vannak, az előzőek szerint pedig az $A$ és
$C$ pontokból az $EH$ szakasz egyenlő nagyságú szögben látszik, így az $A$, $H$, $E$, $C$ pontok egy
körre, a $k$ körre illeszkednek.

Az $AHEC$ húrnégyszögben $ACE \sphericalangle = 180 ^\circ-2\beta$, ezért a vele szemközti szögre $AHE \sphericalangle = 2 \beta$ 
teljesül. Az $AHE \sphericalangle$ mellékszöge az $LHM \sphericalangle$, és így $LHM \sphericalangle = 180 ^\circ-2\beta$.

Az
$LMH$
háromszögben
tehát
$HLM \sphericalangle + LHM \sphericalangle + HML \sphericalangle = 180 ^\circ  $,
azaz
$\beta  + 180 ^\circ - 2 \beta  + HML \sphericalangle = 180 ^\circ$, innen pedig azonnal adódik, hogy $HML \sphericalangle = \beta$.

Az $LMH$ háromszög szögei ezért $HLM \sphericalangle = HML \sphericalangle = \beta$  és $LHM \sphericalangle = 180 ^\circ- 2 \beta$, ezért az $ABC$
háromszög, és az $AF$, $EH$ és $KL$ egyenesek által közrezárt $LMH$ háromszög valóban hasonlók.

\textit{Megjegyzés}: A $KL$ egyenes az $AEH$ háromszögnek a $D$ ponthoz tartozó Simson-Wallace-egyenese.
\medskip

\vonal


{\bf 5. feladat: } $A$, $B$, $C$ véges halmazok, amelyekre teljesül, 
hogy $|A|=|B|=|C|=a$ és $|A \cap B \cap C|=b$, ahol $a$ és
$b$ nemnegatív egészek. Adjuk meg $a$ és $b$ függvényeként az $|A \cup B \cup C|$ minimumát és maximumát! ( $|X|$ az $X$ halmaz elemeinek számát jelöli.)

\ki{Gecse Frigyes}{Kisvárda}\medskip

{\bf Megoldás: } Előbb a maximumot (jelölje $x$) határozzuk meg. Legyen 
$M = A \cap  B \cap  C$,
$A_1 = A \setminus  M$, $B_1 = B \setminus  M$, $C_1 = C \setminus  M$. 
Mivel $A$ az $A_1$ és $M$ diszjunkt halmazok uniója, ezért
$a = |A| = |A_1|+|M|$ és $b = |M|$ miatt $A_1 = a-b$. 
Hasonlóan $B_1 = a-b$, $C_1 = a-b$.

Nyilvánvaló, hogy
\begin{equation} \label{eq:e51}
A \cup  B \cup  C = M \cup  A_1 \cup  B_1 \cup  C_1.
\end{equation}

Az $|A \cup  B \cup  C|$ akkor a legnagyobb, 
ha az $A_1, B_1, C_1$ halmazok páronként diszjunktak.

Ekkor
$$
x = \max |A \cup  B \cup  C| = 
|M|+|A_1|+|B_1|+|C_1| = b+3(a-b)= 3a-2b.$$

Térjünk át a minimum (jelölje $y$) meghatározására! Az $M$ és az 
$A_1 \cup  B_1 \cup  C_1$ halmazok
diszjunktak, ezért (\ref{eq:e51}) alapján a 
$z = |A_1 \cup  B_1 \cup  C_1|$ jelöléssel

\begin{equation} \label{eq:e52}
|A \cup  B \cup  C| = |M|+|A_1 \cup  B_1 \cup  C_1| = b+z.
\end{equation}

Nézzük, hogyan választható $z$ a legkisebbnek! Vezessük be 
az $A_1 \cap  B_1 = D$, $A_2 = A_1 \setminus  D$,
$B_2 = B_1 \setminus  D$, $D = u$, $v = a-b-u$ jelöléseket! Két esetet vizsgálunk.

\begin{enumerate}
\item Ha $u \ge  v$, akkor 
$u \ge \frac 12(a-b)$. Ekkor a $C$ halmaz úgy választható, hogy 
$A_2 \subset  C$, $B_2 \subset C$ legyen. Ekkor
\begin{equation}\label{eq:e53}
z = |A_1 \cup  B_1 \cup  C_1| = 
|D|+|A_2|+|B_2|+|C_1 \setminus \left(A_2 \cup  B_2\right) 
= u+2v+|C_1 \setminus  \left(A_2 \cup  B_2\right)|.
\end{equation}

De $|C_1 \setminus \left(A_2 \cup  B_2\right) 
= a-b-2v$, mert $A_2$ és $B_2$ diszjunktak, 
és így $A_2 \cap  B_2 \cap  C_1 = \emptyset$.

Ennélfogva
$$
z = u+2v+a-b-2v = u+a-b \ge \frac 12 (a-b)+a-b 
= \frac{3(a-b)}{2}.
$$

Ha $\frac 12(a-b)$ egész szám, 
akkor $A_1$-et és $B_1$-et úgy választjuk ki, hogy 
$u=\frac 12(a-b)$ legyen.

Ekkor $z = \frac 12(a-b)$ és 
$y = b+\frac 32(a-b)=
\left[\frac{3a-b+1}{2}\right]
$, 
ahol [ ] az egészrész jele. Ha az
$\frac 12(a-b)$ $\frac 12$-del különbözik egy egész számtól, akkor a halmazokat úgy kell megválasztani,
hogy 
$
z = \frac{3a-b}{2}+\frac{1}{2}$
legyen, és ismét 
$y=b+\frac{3a-b}{2}+\frac{1}{2}
=
\left[
\frac{3a-b+1}{2}
\right]$.

\item Ha $u > v$, akkor a halmazok úgy választhatók, 
hogy $z = u+2v = a-b+v \ge \frac{3(a-b)}{2}$, és
ismét $y = \left[\frac{3a-b+1}{2}\right]$
\end{enumerate}



Tehát a feladat megoldása: $x = 3a-2b$, $y =\left[\frac{3a-b+1}{2}\right]$.

\medskip

\vonal


{\bf 6. feladat: } Legyen $a_1=1, a_2=2$ és
$\displaystyle{\frac{1}{a_{n+2}}
=\frac{1}{2}-\sum_{k=1}^{n}\frac{a_{k+2}}{a_{k+1}\cdot(a_k+a_{k+1}+a_{k+2})}}$
($n \ge 1$ egész). 
Adjuk meg $a_n$-t zárt formában, azaz $n$ függvényeként!

\ki{Bencze Mihály}{Brassó}\medskip

{\bf Megoldás: } A rekurzív definíció szerint:


\begin{eqnarray*}
\frac{1}{a_{1+2}}
&=&
\frac{1}{2}-\sum_{k=1}^1 \frac{a_{k+2}}{a_{k+1}\left(a_k+a_{k+1}+a_{k+2}\right)} =
\frac{1}{2}-\frac{a_{1+2}}{a_{1+1}\left(a_1+a_{1+1}+a_{1+2}\right)}\\
\frac{1}{a_{3}}
&=&
\frac{1}{2}-\frac{a_{3}}{a_{2}\left(a_1+a_{2}+a_{3}\right)}\\
\frac{1}{a_{3}}
&=&
\frac{1}{2}-\frac{a_{3}}{2\left(1+2+a_{3}\right)}\\ 
\frac{1}{a_{3}}
&=&
\frac{1}{2}-\frac{a_{3}}{2\left(3+a_{3}\right)}\\
6+2a_3
&=&
a_3(3+a_3)-a_3^2\\
6+2a_3
&=&
3a_3\\
6
&=&
a_3
\end{eqnarray*}


Az előzmények alapján megfogalmazható a sejtés, 
hogy $a_n = n!$. Bizonyítsuk az állítást teljes
indukcióval!

Láttuk, hogy $n = 1$, $n = 2$, $n = 3$ esetén igaz az állítás! Tegyük fel, hogy $a_t = t!$, ha $t$ $n$-nél
nem nagyobb pozitív egész szám ($n \ge 4$)!
A rekurzív definíció szerint:

$$\frac{1}{a_{t+1}}=\frac{1}{a_{(t-1)+2}}=
\frac{1}{2}-\sum_{k=1}^{t-1} \frac{a_{k+2}}{a_{k+1}(a_k+a_{k+1}+a_{k+2})}=
$$
$$
=\frac{1}{2}-\sum_{k=1}^{t-2} \frac{a_{k+2}}{a_{k+1}(a_k+a_{k+1}+a_{k+2})}
-\frac{a_{t+1}}{a_{t}(a_{t-1}+a_t+a_{t+1})}
$$

Alkalmazva az indukciós feltevést:

\begin{eqnarray*}
\frac{1}{a_{t+1}}&=& 
\frac{1}{2}-\sum_{k=1}^{t-2} \frac{(k+2)!}{(k+1)!(k!+(k+1)!+(k+2)!)}
-\frac{a_{t+1}}{t!\cdot((t-1)!+t!+a_{t+1})}\\
\frac{1}{a_{t+1}}&=& 
\frac{1}{2}-\sum_{k=1}^{t-2} \frac{k+2}{k!\cdot(1+k+1+(k+1)(k+2))}
-\frac{a_{t+1}}{t!\cdot((t-1)!+t!+a_{t+1})}\\
\frac{1}{a_{t+1}}&=& 
\frac{1}{2}-\sum_{k=1}^{t-2} \frac{k+2}{k!\cdot(1+k+1+k^2+3k+2)}
-\frac{a_{t+1}}{t!\cdot((t-1)!+t!+a_{t+1})}\\
\frac{1}{a_{t+1}}&=& 
\frac{1}{2}-\sum_{k=1}^{t-2} \frac{k+2}{k!\cdot(k^2+4k+4)}
-\frac{a_{t+1}}{t!\cdot((t-1)!+t!+a_{t+1})}\\
\frac{1}{a_{t+1}}&=& 
\frac{1}{2}-\sum_{k=1}^{t-2} \frac{k+2}{k!\cdot(k+2)^2}
-\frac{a_{t+1}}{t!\cdot((t-1)!+t!+a_{t+1})}\\
\frac{1}{a_{t+1}}&=& 
\frac{1}{2}-\sum_{k=1}^{t-2} \frac{1}{k!\cdot(k+2)}
-\frac{a_{t+1}}{t!\cdot((t-1)!+t!+a_{t+1})}
\end{eqnarray*}

Ugyanakkor

$$\frac{1}{(k+1)!}-\frac{1}{(k+2)!}=
\frac{(k+2)!-(k+1)!}{(k+1)!\cdot(k+2)!}=
\frac{(k+1)!(k+2-1)}{(k+1)!\cdot(k+2)!}=
\frac{k+1}{(k+2)\cdot(k+1)\cdot k!}=
\frac{1}{k!\cdot (k+2)}
$$

Ezt felhasználva

$$\frac{1}{a_{t+1}}=\frac{1}{2}-\sum_{k=1}^{t-2}
\left(\frac{1}{(k+1)!}-\frac{1}{(k+2)!}\right)-\frac{a_{t+1}}{t!\cdot((t-1)!+t!+a_{t+1})}$$

$$\frac{1}{a_{t+1}} =
\frac{1}{2}-\frac{1}{2!}+\frac{1}{3!}-\frac{1}{3!}+\frac{1}{4!}-
\ldots 
-\frac{1}{(t-2)!}+\frac{1}{(t-1)!}-\frac{1}{(t-1)!}+\frac{1}{t!}
-\frac{a_{t+1}}{t!\cdot((t-1)!+t!+a_{t+1})}
$$

Teleszkópikus összeget kaptunk:

\begin{eqnarray*}
\frac{1}{a_{t+1}}&=&\frac{1}{t!}-\frac{a_{t+1}}{t!\cdot((t-1)!+t!+a_{t+1})}\\
t!&=&a_{t+1}-\frac{a_{t+1}^2}{(t-1)!\cdot(t+1)+a_{t+1}}\\
(t-1)!\cdot(t+1)!+t!\cdot a_{t+1} &=&(t-1)!\cdot(t+1)\cdot a_{t+1}\\
(t-1)!\cdot(t+1)! &=&(t-1)!\cdot(t+1)\cdot a_{t+1}-t!\cdot a_{t+1}\\
(t-1)!\cdot(t+1)! &=&(t-1)!\cdot a_{t+1}\cdot (t+1-t)\\
(t+1)! &=& a_{t+1}
\end{eqnarray*}

Az állítást bebizonyítottuk.

\vfill
\end{document}