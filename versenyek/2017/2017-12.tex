\documentclass[a4paper,10pt]{article} 
\usepackage[utf8]{inputenc}
\usepackage[a4paper]{geometry}
\usepackage[magyar]{babel}
\usepackage{amsmath}
\usepackage{amssymb}
\frenchspacing 
\pagestyle{empty}
\newcommand{\ki}[2]{\hfill {\it #1 (#2)}\medskip}
\newcommand{\vonal}{\hbox to \hsize{\hskip2truecm\hrulefill\hskip2truecm}}
\newcommand{\degre}{\ensuremath{^\circ}}
\newcommand{\tg}{\mathop{\mathrm{tg}}\nolimits}
\newcommand{\ctg}{\mathop{\mathrm{ctg}}\nolimits}
\newcommand{\arc}{\mathop{\mathrm{arc}}\nolimits}
\begin{document}
\begin{center} \Large {\em 26. Nemzetközi Magyar Matematika Verseny} \end{center}
\begin{center} \large{\em Somorja, 2017. március 23-27.} \end{center}
\smallskip
\begin{center} \large{\bf 12. osztály} \end{center}
\bigskip 

{\bf 1. feladat: } A valós számok halmazán oldjátok meg a következő egyenletrendszert:
$$x\cdot y=1,\quad x+y-\cos^2z=-2.$$

\ki{Dr. Minda Mihály}{Vác}\medskip

{\bf 2. feladat: } Néhány papírlapra felírtuk a 2017-nél nem nagyobb pozitív egész számokat. Mindegyik számot pontosan egy
lapra írtunk fel. Az egy lapra írt számok között nem volt két olyan szám, amelyek közül a kisebbik osztója a
nagyobbnak. Mennyi volt a papírlapok számának legkisebb lehetséges értéke? Maximum hány darab szám
szerepelhetett egy papírlapon?

\ki{Erdős Gábor}{Nagykanizsa}\medskip

{\bf 3. feladat: } Az $ABCD$ konvex négyszögben a következő szögek ismertek:
$BCA\sphericalangle=40^\circ$, $BAC\sphericalangle=50^\circ$, és $BDC\sphericalangle=25^\circ$. Határozzátok meg a
négyszög átlói által bezárt szöget.

\ki{Dr. Ripcó Sipos Elvira}{Zenta}\medskip

{\bf 4. feladat: } Bizonyítsátok be, hogy
$\displaystyle{\frac{1}{\cos 50^\circ}+
\frac{1}{\cos 70^\circ}=2\cdot \sqrt{3}+\frac{1}{\cos 10^\circ}}$.\smallskip

\ki{Dr. Bencze Mihály}{Bukarest}\medskip

{\bf 5. feladat: } Az $f$ függvény minden valós 
$x, y$ számpárra eleget tesz az 
$$f(x)+f(y)=f(x+y)-xy-1$$
függvényegyenletnek és $f(1)=1$. Határozzátok meg az $f$ függvény következő értékeit: 
$f(17), f(1/2)$ és $f(1/3)$.

\ki{Oláh György}{Révkomárom}

\ki{Bálint Béla}{Zsolna}\medskip

{\bf 6. feladat: } A táblára fel van írva egymás után $n$ darab valós szám. Ezek a számok a következő tulajdonsággal
rendelkeznek: ,,Ha tetszőleges mennyiségű számot letörlünk úgy, hogy a táblán maradt számok csökkenő
sorozatot alkotnak, akkor ez a sorozat legfeljebb $k$ hosszúságú.'' Bizonyítsátok be, hogy a táblán lévő összes
szám befesthető legfeljebb $k$ színnel úgy, hogy az egyforma színű számok növekvő sorozatot alkossanak.

\ki{Keke\v{n}ák Tamás}{Kassa}\medskip


\end{document}