\documentclass[a4paper,10pt]{article} 
\usepackage[utf8]{inputenc}
\usepackage[a4paper]{geometry}
\usepackage[magyar]{babel}
\usepackage{amsmath}
\usepackage{amssymb}
\frenchspacing 
\pagestyle{empty}
\newcommand{\ki}[2]{\hfill {\it #1 (#2)}\medskip}
\newcommand{\vonal}{\hbox to \hsize{\hskip2truecm\hrulefill\hskip2truecm}}
\newcommand{\degre}{\ensuremath{^\circ}}
\newcommand{\tg}{\mathop{\mathrm{tg}}\nolimits}
\newcommand{\ctg}{\mathop{\mathrm{ctg}}\nolimits}
\newcommand{\arc}{\mathop{\mathrm{arc}}\nolimits}
\begin{document}
\begin{center} \Large {\em 26. Nemzetközi Magyar Matematika Verseny} \end{center}
\begin{center} \large{\em Somorja, 2017. március 23-27.} \end{center}
\smallskip
\begin{center} \large{\bf 9. osztály} \end{center}
\bigskip 

{\bf 1. feladat: } Adott egy síkon 2017 (különböző) pont úgy, hogy nem esik mind egy egyenesre. Bizonyítsuk be, hogy meg
lehet adni a síkon olyan körlapot, amelynek határán rajta van az adott pontok közül legalább három, de a
belsejében egy sem.

\ki{Dr. Kántor Sándor}{Debrecen}\medskip

{\bf 2. feladat: } Az $ABC$ derékszögű háromszögben a $CB$ befogó felezőpontja $M$, az $AC$ befogó felezőpontja $N$, ahol
$|BN|=19$ és $|AM|=22$. Mekkora az $AB$ átfogó hossza?

\ki{Dr. Kántor Sándorné}{Debrecen}\medskip

{\bf 3. feladat: } A táblára felírtunk 2015 darab $D$ betűt, 2016 darab $B$ betűt és 2017 darab $C$ betűt. Ketten játszanak. A soron
levő játékos letöröl 2 nem egyforma betűt és a harmadikat írja helyükbe. Pld. letöröl egy $D$-t és egy $B$-t és felír
egy darab $C$-t. A játék befejeződik, ha csak egyfajta betű maradt a táblán. Ha $D$ betű maradt, akkor a kezdő
játékos nyer, ha $B$ maradt, akkor a második nyer, $C$ betű esetén döntetlen. Kinek van nyerő stratégiája?
Indokoljátok!

\ki{Mészáros József}{Jóka}\medskip

{\bf 4. feladat: } A valós számok halmazán oldjuk meg a $\displaystyle{\sqrt{\frac{1}{x^2}-\frac{3}{4}}<\frac{1}{x}-\frac{1}{2}}$ egyenlőtlenséget!

\ki{Bálint Béla}{Zsolna}\medskip

{\bf 5. feladat: } Az $A$ természetes szám $n$ darab egyforma számjegyből áll, a $B$ természetes szám szintén $n$ darab egyforma
számjegyből áll, a $C$ természetes számot pedig $2n$ darab egyforma számjegy alkotja. Ugyanakkor $n\ge 2$ esetén $A^2+B=C$ is teljesül. Hány ilyen számjegy hármas teljesíti a feltételeket?

\ki{Tóth Sándor}{Kisvárda}\medskip

{\bf 6. feladat: } Határozzuk meg az összes olyan $n$ természetes számot, amelyekre $n^2-10n+23$, 
$n^2-9n+31$ és $n^2-12n+46$ prímszámok!

\ki{Kiss Alexandra és Fedorszki Ádám}{Beregszász}\medskip


\end{document}