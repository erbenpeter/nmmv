\documentclass[a4paper,10pt]{article} 
\usepackage[utf8]{inputenc}
\usepackage[a4paper]{geometry}
\usepackage[magyar]{babel}
\usepackage{amsmath}
\usepackage{amssymb}
\frenchspacing 
\pagestyle{empty}
\newcommand{\ki}[2]{\hfill {\it #1 (#2)}\medskip}
\newcommand{\vonal}{\hbox to \hsize{\hskip2truecm\hrulefill\hskip2truecm}}
\newcommand{\degre}{\ensuremath{^\circ}}
\newcommand{\tg}{\mathop{\mathrm{tg}}\nolimits}
\newcommand{\ctg}{\mathop{\mathrm{ctg}}\nolimits}
\newcommand{\arc}{\mathop{\mathrm{arc}}\nolimits}
\begin{document}
\begin{center} \Large {\em 26. Nemzetközi Magyar Matematika Verseny} \end{center}
\begin{center} \large{\em Somorja, 2017. március 23-27.} \end{center}
\smallskip
\begin{center} \large{\bf 11. osztály} \end{center}
\bigskip 

{\bf 1. feladat: } A Nemzetközi Magyar Matematikaverseny résztvevői délután is hasznosan töltik az idejüket: sakkversenyt
rendeznek. A versenyzők két kategóriába oszthatók: kockákra és nagyágyúkra. A versenyen mindenki mindenki
ellen játszott egy mérkőzést. A verseny vége után észrevették, hogy minden versenyző a pontszámai felét éppen
a kockák elleni mérkőzésein szerezte. Mutassátok meg, hogy ha m az összes versenyző számát jelöli és nem
volt egyetlen döntetlen sem, akkor $\sqrt{m}$ egész szám.

\ki{Keke\v{n}ák Szilvia}{Kassa}\medskip

{\bf 2. feladat: } Minden természetes $n\ge 2$-re bizonyítsátok be a $\displaystyle{3^{2n-1}>n^4+10}$ egyenlőtlenséget!

\ki{Dr. Bencze Mihály}{Bukarest}

\ki{Bálint Béla}{Zsolna}\medskip

{\bf 3. feladat: } Legyen 
$f(x)=x^2+6x+1$
és jelölje $M$ a koordináta-sík azon $(x; y)$ pontjainak a halmazát, amelyekre
$f(x)+f(y)\le 0$ és $f(x)-f(y)\le 0$. Mekkora az $M$ ponthalmaz területe?

\ki{Dr. Kántor Sándorné}{Debrecen}\medskip

{\bf 4. feladat: } A természetes számok halmazán oldjátok meg a következő egyenletet: 
$$x-y-\frac{x}{y}-\frac{x^3}{y^3}+\frac{x^4}{y^4}=2017.$$

\ki{Kovács Béla}{Szatmárnémeti}\medskip

{\bf 5. feladat: } Kati papírból háromszögeket vágott ki. Mindegyik háromszögnek volt egy 5 cm és egy 11 cm hosszú oldala, és
a harmadik oldal hossza szintén cm-ben mérve egész szám volt. Pisti észrevette, hogy semelyik két háromszög
nem volt egybevágó, de ha kivágott volna Kati még egy, a szabályoknak megfelelő háromszöget, akkor az
biztosan egybevágó lett volna valamelyik korábban kivágottal. A kivágott háromszögekkel Kati és Pisti játszani
kezdett. Felváltva léptek. Egy lépésben egy hegyesszögű, vagy egy tompaszögű, vagy két tompaszögű
háromszöget lehetett elvenni. Az nyert, aki utoljára lépett. Melyik játékosnak volt nyerő stratégiája, ha elsőként
Kati lépett?

\ki{Erdős Gábor}{Nagykanizsa}\medskip

{\bf 6. feladat: } Adott egy olyan $ABC$ háromszög, melyben $BAC\sphericalangle=60^\circ$
és $ABC\sphericalangle=50^\circ$. 
Legyen $O$ egy olyan pont az $ABC$
háromszög belsejében, melyre 
$AOC\sphericalangle=100^\circ$ és 
$CBO\sphericalangle=30^\circ$. Határozzátok meg az $OCB\sphericalangle$ mértékét!

\ki{Bíró Béla}{Sepsiszentgyörgy}\medskip


\end{document}