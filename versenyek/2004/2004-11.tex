\documentclass[a4paper,10pt]{article} 
\usepackage[latin2]{inputenc}
\usepackage{graphicx}
\usepackage{amssymb}
\voffset - 20pt
\hoffset - 35pt
\textwidth 450pt
\textheight 650pt 
\frenchspacing 

\pagestyle{empty}
\def\ki#1#2{\hfill {\it #1 (#2)}\medskip}

\def\tg{\, \hbox{tg} \,}
\def\ctg{\, \hbox{ctg} \,}
\def\arctg{\, \hbox{arctg} \,}
\def\arcctg{\, \hbox{arcctg} \,}

\begin{document}
\begin{center} \Large {\em XIII. Nemzetk�zi Magyar Matematika Verseny} \end{center}
\begin{center} \large{\em Nagydobrony, 2004. m�rc. 15-20.} \end{center}
\smallskip
\begin{center} \large{\bf 11. oszt�ly} \end{center}
\bigskip 

{\bf 1. feladat: } Bizony\'itsa be, hogy $9^{n}-8n-1$ oszthat\'o $64$-gyel, ahol $n$ nemnegat\'iv eg\'esz sz\'am.


\ki{Ol�h Gy�rgy}{Kom�rom}\medskip

{\bf 2. feladat: } Igazolja, hogy a k\"ul\"onb\"oz\H o oldal\'u h\'aromsz\"ogben a legkisebb sz\"og cs\'ucs\'ab\'ol h\'uzott sz\"ogfelez\H o a leghosszabb!


\ki{Dr. K�ntor S�ndor}{Debrecen}\medskip

{\bf 3. feladat: } Oldja meg a val\'os sz\'amok halmaz\'an az al\'abbi egyenletet:
$$\sqrt{4x^{2}-4x+2} = \frac{\sqrt{2} - \left(\sqrt{2x-1} -1\right)^{2}}{\sqrt{\frac{1}{2x-1}}}$$



\ki{B�r� B�lint}{Eger}\medskip

{\bf 4. feladat: } Az $ABC$ h\'aromsz\"ogben $AB = 4$~cm, $AC = 8$~cm, $A\angle=120^{\circ}$. Az $F$ pont a h\'aromsz\"og k\"or\'e \'irt k\"or $BAC$ \'iv\'enek felez\H opontja. Mekkora t\'avols\'agra van az $F$ pont a h\'aromsz\"og magass\'againak metsz\'espontj\'at\'ol?



\ki{Neubauer Ferenc}{Munk�cs}\medskip

{\bf 5. feladat: } A $P$ \'es $Q$ pontok \'ugy helyezkednek el az $ABC$ h\'aromsz\"og $BC$ oldal\'an, hogy a l\'etrej\"ov\H o szakaszok ar\'anya:  $BP:PQ:QC=1:2:3$. Az $R$ pont a h\'aromsz\"og $AC$ oldal\'at a k\"ovetkez\H ok\'eppen harmadolja: $AR:RC=1:2$. Az $M$ \'es $N$ pontok a $BR$ szakasznak az $AQ$ \'es az $AP$ szakaszokkal val\'o metsz\'espontjait jel\"oli. Hat\'arozza meg a $PQMN$ n\'egysz\"og ter\"ulet\'et, ha az $ABC$ h\'aromsz\"og ter\"ulete \mbox{$24$~cm${}^{2}$}.



\ki{Dr. Pint�r Ferenc}{Nagykanizsa}\medskip

{\bf 6. feladat: } Bizony\'itsa be, hogy ha az $a$, $b$ \'es $c$ val\'os sz\'amokra teljes\"ul a $0<a<b<c$ felt\'etel, akkor az $\left(a+b+c\right)x^{2}-2\left(ab+bc+ac\right)x+3abc=0$ egyenletnek k\'et k\"ul\"onb\"oz\H o val\'os gy\"oke van, \'es az egyik gy\"ok az $a$ \'es $b$ k\"oz\'e, a m\'asik a $b$ \'es $c$ k\"oz\'e esik.


\ki{Dr. Katz S�ndor}{Bonyh�d}\medskip

{\bf 7. feladat: } Egy h\'aromsz\"og egyik cs\'ucs\'ab\'ol kiindul\'o magass\'ag, sz\"ogfelez\H o \'es oldalfelez\H o az illet\H o sz\"oget rendre $x$, $y$, $x$, $y$ sz\"ogekre osztja. Igazolja, hogy $\sin^{3}\left(x+y\right)=\sin x\cos y$.


\ki{Bencze Mih�ly}{Brass�}\medskip

\vfill
\end{document}