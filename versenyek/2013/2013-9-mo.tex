\documentclass[a4paper,10pt]{article} 
\usepackage[utf8]{inputenc}
\usepackage[a4paper]{geometry}
\usepackage[magyar]{babel}
\usepackage{t1enc}
\usepackage{amsmath}
\usepackage{amssymb}
\usepackage{caption}
\usepackage{pgf,tikz}
\usepackage{pstricks,pstricks-add}
\frenchspacing 
\pagestyle{empty}
\newcommand{\ki}[2]{\hfill {\it #1 (#2)}\medskip}
\newcommand{\vonal}{\hbox to \hsize{\hskip2truecm\hrulefill\hskip2truecm}}
\newcommand{\degre}{\ensuremath{^\circ}}
\newcommand{\tg}{\mathop{\mathrm{tg}}\nolimits}
\newcommand{\ctg}{\mathop{\mathrm{ctg}}\nolimits}
\newcommand{\arc}{\mathop{\mathrm{arc}}\nolimits}
\renewcommand{\vec}[1]{\mathbf{#1}}
\begin{document}
\begin{center} \Large {\em XXII. Nemzetközi Magyar Matematikaverseny} \end{center}
\begin{center} \large{\em Győr, 2013. március 14--18.} \end{center}
\smallskip
\begin{center} \large{\bf 9. osztály} \end{center}
\bigskip

{\bf 1. feladat: } Határozza meg azokat az $m > n > g$ pozitív egész számokat, amelyekre
$$m^2-n^2-g^2=2ng+68.$$

\ki{Oláh György}{Felvidék}\medskip

{\bf Megoldás: } Egyenletünket átalakítva $m^2-(n+g)^2=68$,
illetve: $(m+n+g)(m-n-g)=68$ adódik. Mivel 
$m+n+g>m-n-g>0$, ezért elegendő a
$68=68\cdot 1=17\cdot 4=34\cdot 2$
eseteket vizsgálni.
$(m+n+g)+(m-n-g)=2m$, 
így csak $m+n+g=34$
és $m-n-g=2$ lehetséges. 
Ekkor $2m=36$, $m=18$. Továbbá
$18+n+g=34$ és $18-n-g=2$ miatt $n+g=16$. 

Ez akkor teljesül, ha

\begin{center}
\begin{tabular}{rrr}
m&n&g\cr
\hline
18&9 &7\cr
18&10&6\cr
18&11&5\cr
18&12&4\cr
18&13&3\cr
18&14&2\cr
18&15&1\cr
\hline
\end{tabular}
\end{center}

A feladat megoldását a felsorolt 7 számhármas adja, ezek ki is elégítik az eredeti feltételeket.

\medskip

\vonal


{\bf 2. feladat: } Kukori és Kotkoda egy tojással teli kosárral érkezett a piacra. Az első vevőjük 
Kopasznyakú volt, aki megvette a tojások felét és még két tojást. A második vevő 
Kendermagos volt, aki megvette az első vásárlásból megmaradt tojások felét és még két 
tojást. A harmadik vásárló, Hápogi megvette a második vásárlás után megmaradt tojások 
felét és még két tojást. A negyedik vásárló, Csőrike megvette a Hápogi vásárlása után 
megmaradt tojások felét és még két tojást. Csőrike vásárlása után Kotkoda örömmel 
állapította meg, hogy kiürült a kosár. Mennyi tojást vitt el Kukori és Kotkoda a kosárban a 
piacra? 


\ki{Dr. Péics Hajnalka}{Délvidék}\medskip

{\bf Megoldás: } Jelölje $x$ a kosárban levő tojások számát.
Kopasznyakú vásárlása után $\frac x2-2$ tojás maradt a kosárban.

Kendermagos vásárlása után $\frac{1}{2}\left(\frac{x}{2}-2\right)-2=\frac{x}{4}-3$ tojás maradt a kosárban.

Hápogi vásárlása után $\frac{1}{2}\left(\frac{x}{4}-3\right)-2=\frac{x}{8}-\frac{7}{2}$ tojás maradt a kosárban.

Csőrike vásárlása után $\frac{1}{2}\left(\frac{x}{7}-\frac{7}{2}\right)-2=\frac{x}{16}-\frac{15}{4}$ tojás maradt a kosárban, s mivel a kosár
kiürült, így az

$$\frac{x}{16}-\frac{15}{4}=0$$

egyenletnek kell teljesülnie, amelynek megoldása $x=60$.

Tehát Kukori és Kotkoda 60 tojást vitt a kosárban a piacra. 
Valóban, a vásárlások után $60/2 - 2 = 28$, aztán 12, 4, 0 tojás maradt a kosárban. 

\textit{Megjegyzés}: A feladat -- egyenlet nélkül -- visszafelé is megoldható, sőt általánosítható.

\medskip

\vonal


{\bf 3. feladat: } Három kör közül mindegyik átmegy a másik kettő középpontján. Mekkora a három kör 
közös részének a területe? 

\ki{Pintér Ferenc}{Magyarország}\medskip

{\bf Megoldás: } Jelölje a három kör középpontját $A$, $B$ és $C$. A feladat feltételéből adódóan az $ABC$ háromszög
szabályos.

A keresett területet megkaphatjuk, ha az $A$, $B$ és $C$ középpontú $60^\circ$-os körcikkek
területösszegéből (ami együttesen egy félkör területe) levonjuk az $ABC$ szabályos háromszög
területének a kétszeresét.

Ha a körök sugara $r$, akkor a $60^\circ$-os körcikk területe hatodrésze a kör területének, azaz $\frac{r^2\pi}{6}$,
az $r$ oldalú szabályos háromszög területe $\frac{r^2\sqrt 3}{4}$, így a keresett terület:
$$3\cdot \frac{r^2\pi}{6}-2\cdot \frac{r^2\sqrt 3}{4}=\frac{r^2}{2}\left(\pi-\sqrt 3\right).$$
\medskip

\vonal


{\bf 4. feladat: }  Hányféle módon lehet a 2013-as számot olyan természetes számok összegeként előállítani, az összeadandók sorrendjétől eltekintve, amelyeknek a szorzata is 2013? 


\ki{Szabó Magda}{Délvidék}\medskip

{\bf Megoldás: } Mivel $2013 = 3 \cdot  11 \cdot  61$ így először  
$2013 = 3 \cdot  11 \cdot  61 = 3 + 11 + 61 + 1 +\ldots+1$, az 1-esek
száma 1938, az összeadandók sorrendjétől eltekintünk, ezért ez egy eset. 

A második felbontásnál 
$2013 = 33 \cdot  61 = 33 + 61 + 1 + \ldots + 1$, az 1-esek száma 1919, ez a
második eset.

A harmadik felbontásnál 
$2013 = 183 \cdot  11 = 183 + 11 + 1 + \ldots + 1$, az 1-esek száma 1819, ez a harmadik eset.

A negyedik esethez 
$2013 = 3 \cdot  671 = 671 + 3 + 1 + \ldots + 1$, az 1-esek száma 1339. 

A $2013 = 2013$ nyilvánvaló előállítást megoldásnak is tekinthetjük, ki is zárhatjuk (megállapodás kérdése). Ettől függően 4 vagy 5-féle előállítási módunk lesz. Mindkét választ elfogadjuk.


\medskip

\vonal


{\bf 5. feladat: } Tekintsük az 
$$1\cdot 5^0,\quad 1\cdot 5^0+2\cdot 5^1,\quad 1\cdot 5^0+2\cdot 5^1+3\cdot 5^2,\quad\ldots,\quad 1\cdot 5^0+2\cdot 5^1+3\cdot 5^2+\ldots+k\cdot 5^{k-1}$$
számokat, ahol $k$ tetszőleges pozitív egész szám és vegyük ezen számok utolsó számjegyét, 
majd alkossunk ezen számjegyekből egy sorozatot. Mi a sorozat 9024. tagja? 


\ki{Bíró Béla}{Erdély}\medskip

{\bf I. megoldás: } Igazoljuk, hogy az 
$(a_n)$ sorozat 4-es periódusú, azaz $a_{n+4} = {a_n}$, tetszőleges
$n\in \{1, 2, 3,\ldots\}$ pozitív egész számra.

Egyrészt:

$1\cdot 5^0 = 1 \Rightarrow a_1 = 1$

$1+2\cdot 5^1 = 11 \Rightarrow a_2 =1$

$1+2\cdot 5^1 +3\cdot 5^2 = 86 \Rightarrow a_3 =6$

$1+2\cdot 5^1 +3\cdot 5^2 +4\cdot 5^3 = 586 \Rightarrow a_4 =6$

Másrészt, a periodikussághoz elegendő belátni, hogy az 
$a_{n+4} - a_n$ számok 0-ra végződnek.

Valóban:

$$a_{n+4} - a_n = (n+1)5^n + (n+2)5^{n+1} + (n+3)5^{n+2} + (n+4)5^{n+3} = $$
$$= (n+1)5^n + (n+2)\cdot 5\cdot 5^n +
(n+3)\cdot 25\cdot 5^n + (n+4)\cdot 125\cdot 5^n =$$
$$= (156n + 586)5^n =
2\cdot 5^n (78n+293),$$ 

ami osztható 10-el, bármely $n$ pozitív egész szám esetén.

A $9024 = 4 \cdot (2256)$-ik szám tehát 6.


\medskip

{\bf II. megoldás: } Jelöljük $a_n$-nel az 
$1 \cdot 5^0 + 2 \cdot 5^1 + \ldots + n \cdot 5^{n-1}$ számot, 
$j_n$-nel az utolsó jegyét.

Ekkor $a_{n+1}-a_n =(n+1)\cdot 5^n$, ami $n$ páratlan szám esetén egy 0-ra, míg páros esetén egy 5-re végződő szám. Tehát az utolsó jegy a sorozatban vagy megegyezik az előzővel, vagy
pedig 5-tel tér el az előzőtől. Ez pedig ismétlődik rendre (hiszen a páros-páratlan számok is
ismétlődnek), azaz a páratlan utáni páros indexű elemek megegyeznek, páros utáni páratlan
pedig 5-tel tér el.

Mivel $j_1 = 1; j_2 = 1; j_3 = 6; j_4 = 6; j_5 = 1 \ldots$ tehát az utolsó jegyek négyesével ismétlődnek.

\medskip

\vonal


{\bf 6. feladat: }  Oldja meg a következő egyenletet a valós számok halmazán:
$$\left|2x-4\right|-x=\{x\}$$
($\{x\}$ jelöli az $x$ szám törtrészét, azaz $x$-nek és a legnagyobb, $x$-nél nem nagyobb
egésznek a különbségét. Pl.: $\{3{,}71\} = 3{,}71-3= 0{,}71$, vagy $\{-2{,}4\}= -2{,}4 - (-3) = 0{,}6$.)


\ki{Dr. Katz Sándor}{Magyarország}\medskip

{\bf Megoldás: } Az $\{x\}$ grafikonja 1 meredekségű, alul zárt, felül nyílt szakaszokból áll. 

\begin{center}
\psset{xunit=1.0cm,yunit=1.0cm,algebraic=true,dimen=middle,dotstyle=o,dotsize=3pt 0,linewidth=0.8pt,arrowsize=3pt 2,arrowinset=0.25}
\begin{pspicture*}(-0.66,-2.48)(5.78,2.48)
\psaxes[labelFontSize=\scriptstyle,xAxis=true,yAxis=true,Dx=1,Dy=1,ticksize=-2pt 0,subticks=2]{->}(0,0)(-0.66,-2.48)(5.78,2.48)
\psplot[plotpoints=200,linecolor=gray]{-0.6600000000000005}{5.779999999999999}{abs(2*x-4)-x}
\psline[linewidth=0.05](0,0)(1,1)\psdots[dotstyle=*,dotsize=4pt](0,0)\psdots[dotstyle=o,dotsize=4pt](1,1)
\psline[linewidth=0.05](1,0)(2,1)\psdots[dotstyle=*,dotsize=4pt](1,0)\psdots[dotstyle=o,dotsize=4pt](2,1)
\psline[linewidth=0.05](2,0)(3,1)\psdots[dotstyle=*,dotsize=4pt](2,0)\psdots[dotstyle=o,dotsize=4pt](3,1)
\psline[linewidth=0.05](3,0)(4,1)\psdots[dotstyle=*,dotsize=4pt](3,0)\psdots[dotstyle=o,dotsize=4pt](4,1)
\psline[linewidth=0.05](4,0)(5,1)\psdots[dotstyle=*,dotsize=4pt](4,0)\psdots[dotstyle=o,dotsize=4pt](5,1)
\psline[linewidth=0.05](5,0)(6,1)\psdots[dotstyle=*,dotsize=4pt](5,0)\psdots[dotstyle=o,dotsize=4pt](6,1)
\psline[linewidth=0.05](-1,0)(0,1)\psdots[dotstyle=*,dotsize=4pt](-1,0)\psdots[dotstyle=o,dotsize=4pt](0,1)
\begin{scriptsize}
\rput[tl](1,2.24){\gray{$|2x-4|-x$}}
%\rput[bl](-4.2,-0.5){$f$}
%\rput[bl](-0.54,5.98){$g$}
\end{scriptsize}
\end{pspicture*}
\end{center}


$$|2x-4|-x =
\left\{
\begin{array}{rl}
		x-4  & \mbox{ha } x \geq 2 \\
		-3x+4 & \mbox{ha } x < 2
	\end{array}
\right. $$

$0\le \{x\}<1$, ezért vizsgáljuk, hogy a $|2x-4|-x$
(azaz az $x-4$ és $-3x+4$ kifejezések) hol vesznek fel a $[0; 1[$ intervallumba eső értékeket!

Ha $x \le 1$, vagy $x\ge 5$, akkor 
$|2x-4|-x \ge 1$,
és ha $4/3 <x< 4$ akkor $|2x-4|-x <0$,
ezért az egyenlet megoldásait csak az $[1; 4/3]$ és a $[4; 5[$ intervallumokon kereshetjük .

Az $[1 ; 4/3]$ intervallumon az egyenletünk: $-3x+4 = x-1$. Ennek megoldása $x=5/4$. Ez az
adott intervallumba esik, és kielégíti az egyenletet. 

A $[4; 5[$ intervallumon az egyenletünk: $x-4 = x-4$. Ennek megoldása minden olyan $x$, amely
az adott intervallumba esik.

Tehát az egyenlet megoldásai $x=5/4$ és minden olyan $x$, amelyre $4\le x<5$.



\end{document}