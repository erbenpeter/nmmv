\documentclass[a4paper,10pt]{article} 
\usepackage[utf8]{inputenc}
\usepackage{graphicx}
\usepackage{amssymb}
\voffset - 20pt
\hoffset - 35pt
\textwidth 450pt
\textheight 650pt 
\frenchspacing

\pagestyle{empty}
\def\ki#1#2{\hfill {\it #1 (#2)}\medskip}

\def\tg{\, \hbox{tg} \,}
\def\ctg{\, \hbox{ctg} \,}
\def\arctg{\, \hbox{arctg} \,}
\def\arcctg{\, \hbox{arcctg} \,}

\begin{document}
\begin{center} \Large {\em XXII. Nemzetközi Magyar Matematikaverseny} \end{center}
\begin{center} \large{\em Győr, 2013. március 14--18.} \end{center}
\smallskip
\begin{center} \large{\bf 10. osztály} \end{center}
\bigskip 

{\bf 1. feladat: }
Határozza meg azt az $x$ valós számot, amelyre az
$$f(x) = |8 – x| + | 11 – x | + | 13 – x | + | 16 – x | + | 19 – x |$$
függvény értéke a legkisebb. Mennyi ez az érték?

\ki{Kántor Sándorné}{Magyarország}\medskip

{\bf 2. feladat: }
Keresse meg azokat a pozitív egész számpárokat, amelyeknek a számtani közepe eggyel nagyobb a harmonikus közepüknél!
(Emlékeztetőül: két pozitív valós szám harmonikus közepének reciproka egyenlő a számok reciprokainak számtani közepével.)

\ki{Kallós Béla}{Magyarország}\medskip

{\bf 3. feladat: }
Igazolja, hogy bármely $2 \le n$ egész számhoz léteznek olyan pozitív egész $x$, $y$, $z$ számok, melyekre teljesül, hogy
$$x^2+y^2+z^2=25^n.$$

\ki{Bencze Mihály}{Erdély}\medskip

{\bf 4. feladat: }
Ha $x$, $y$, $z$ pozitív egész számok és $3x+668y=671z$, mutassa meg, hogy az $$n=x^2(y-z)+y^2(z-x)+z^2(x-y)$$ szám osztható $2013\cdot 668$-cal!

\ki{Longáver Lajos}{Erdély}\medskip

{\bf 5. feladat: }
Az $A$-ban derékszögű $ABC$ háromszögben a $BAC$ szög belső szögfelezője $BC$-t $D$-ben metszi. Az $ADB$ szög belső szögfelezője $AB$-t az $E$ pontban, míg az $ADC$ szög belső szögfelezője $AC$-t az $F$ pontban metszi.
Igazolja, hogy a szokásos ($AB = c$, $BC = a$, $CA= b$) jelölésekkel: $BE+CF=\frac{a^2}{b+c}$.

\ki{Molnár István}{Magyarország}\medskip

{\bf 6. feladat: }
Adott egy téglalap, amelynek oldalai $6$ és $3$ egység hosszúságúak és a belsejében $19$ egymástól különböző pont található. Igazolja, hogy létezik közöttük három olyan pont, amelyek által alkotott síkidom területe legfeljebb egy területegység.

\ki{Olosz Ferenc}{Erdély}\medskip

\vfill
\end{document}