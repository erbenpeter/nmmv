\documentclass[a4paper,10pt]{article} 
\usepackage[utf8]{inputenc}
\usepackage[a4paper]{geometry}
\usepackage[magyar]{babel}
\usepackage{amsmath}
\usepackage{amssymb}
\frenchspacing 
\pagestyle{empty}
\newcommand{\ki}[2]{\hfill {\it #1 (#2)}\medskip}
\newcommand{\vonal}{\hbox to \hsize{\hskip2truecm\hrulefill\hskip2truecm}}
\newcommand{\degre}{\ensuremath{^\circ}}
\newcommand{\tg}{\mathop{\mathrm{tg}}\nolimits}
\newcommand{\ctg}{\mathop{\mathrm{ctg}}\nolimits}
\newcommand{\arc}{\mathop{\mathrm{arc}}\nolimits}
\begin{document}
\begin{center} \Large {\em XXII. Nemzetközi Magyar Matematika Verseny} \end{center}
\begin{center} \large{\em Győr, 2013. március 14-18.} \end{center}
\smallskip
\begin{center} \large{\bf 9. osztály} \end{center}
\bigskip 

{\bf 1. feladat: } Határozza meg azokat az $m > n > g$ pozitív egész számokat, amelyekre
$$m^2-n^2-g^2=2ng+68.$$

\ki{Oláh György}{Felvidék}\medskip

{\bf 2. feladat: } Kukori és Kotkoda egy tojással teli kosárral érkezett a piacra. Az első vevőjük 
Kopasznyakú volt, aki megvette a tojások felét és még két tojást. A második vevő 
Kendermagos volt, aki megvette az első vásárlásból megmaradt tojások felét és még két 
tojást. A harmadik vásárló, Hápogi megvette a második vásárlás után megmaradt tojások 
felét és még két tojást. A negyedik vásárló, Csőrike megvette a Hápogi vásárlása után 
megmaradt tojások felét és még két tojást. Csőrike vásárlása után Kotkoda örömmel 
állapította meg, hogy kiürült a kosár. Mennyi tojást vitt el Kukori és Kotkoda a kosárban a 
piacra? 


\ki{Dr. Péics Hajnalka}{Délvidék}\medskip

{\bf 3. feladat: } Három kör közül mindegyik átmegy a másik kettő középpontján. Mekkora a három kör 
közös részének a területe? 

\ki{Pintér Ferenc}{Magyarország}\medskip

{\bf 4. feladat: } Hányféle módon lehet a 2013-as számot olyan természetes számok összegeként előállítani, az összeadandók sorrendjétől eltekintve, amelyeknek a szorzata is 2013? 


\ki{Szabó Magda}{Délvidék}\medskip

{\bf 5. feladat: } Tekintsük az 
$$1\cdot 5^0,\quad 1\cdot 5^0+2\cdot 5^1,\quad 1\cdot 5^0+2\cdot 5^1+3\cdot 5^2,\quad\ldots,\quad 1\cdot 5^0+2\cdot 5^1+3\cdot 5^2+\ldots+k\cdot 5^{k-1}$$
számokat, ahol $k$ tetszőleges pozitív egész szám és vegyük ezen számok utolsó számjegyét, 
majd alkossunk ezen számjegyekből egy sorozatot. Mi a sorozat 9024. tagja? 


\ki{Bíró Béla}{Erdély}\medskip

{\bf 6. feladat: } Oldja meg a következő egyenletet a valós számok halmazán:
$$\left|2x-4\right|-x=\{x\}$$
($\{x\}$ jelöli az $x$ szám törtrészét, azaz $x$-nek és a legnagyobb, $x$-nél nem nagyobb
egésznek a különbségét. Pl.: $\{3{,}71\} = 3{,}71-3= 0{,}71$, vagy $\{-2{,}4\}= -2{,}4 - (-3) = 0{,}6$.)


\ki{Dr. Katz Sándor}{Magyarország}\medskip


\end{document}