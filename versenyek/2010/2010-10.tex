\documentclass[a4paper,10pt]{article} 
\usepackage[utf8]{inputenc}
\usepackage{t1enc}
\usepackage{graphicx}
\usepackage{amssymb}
\usepackage{pstricks, pstricks-add}
\voffset - 20pt
\hoffset - 35pt
\textwidth 450pt
\textheight 650pt 
\frenchspacing 

\pagestyle{empty}
\def\ki#1#2{\hfill {\it #1 (#2)}\medskip}

\def\tg{\, \hbox{tg} \,}
\def\ctg{\, \hbox{ctg} \,}
\def\arctg{\, \hbox{arctg} \,}
\def\arcctg{\, \hbox{arcctg} \,}

\begin{document}
\begin{center} \Large {\em XIX. Nemzetközi Magyar Matematika Verseny} \end{center}
\begin{center} \large{\em Szatmárnémeti, 2010. március 19-22.} \end{center}
\smallskip
\begin{center} \large{\bf 10. osztály} \end{center}
\bigskip 

{\bf 1. feladat: }  Legalább hány szeget kell beütni az alábbi
farács rácspontjaiba ahhoz, hogy biztosan legyen köztük $4$
szeg, amely egy téglalapot feszít ki?

\smallskip
\centerline{
\psset{xunit=0.5cm,yunit=0.5cm}
\begin{pspicture}(12,4)
\psgrid[gridlabels=0pt,subgriddiv=0]
\end{pspicture}
}

\ki{Nagy Örs}{Kolozsvár}\medskip

{\bf 2. feladat: } Határozzuk meg azokat
az $x,y\in \mathbb{N}$ számokat, amelyekre
$$xy(x-y)=13x+15y.$$

\ki{Kacsó Ferenc}{Marosvásárhely}\medskip

{\bf 3. feladat: }  Adott a síkon négy pont úgy, hogy közülük semelyik három sincs egy egyenesen.
Kiszíneztük a négy pontot négy színnel: pirossal, kékkel, zölddel,
és sárgával. Ezután kiszíneztük a pontok által meghatá\-rozott
szaka\-szo\-kat is úgy, hogy azok színe megegyezett valamelyik
vég\-pont\-juk színével, és közben mind a négy színt újra
felhasználtuk. Igaz-e, hogy mindig van olyan pont, hogy

(1) vagy a belőle kiinduló szakaszok közül,

(2) vagy a másik három pont közti szakaszok közül

\noindent az egyik piros, a másik kék, a harmadik zöld?

\ki{dr. Kántor Sándorné}{Debrecen}\medskip

{\bf 4. feladat: } Mennyi azoknak a pozitív egészeknek az összege,
amelyek $2010$-nél nem nagyobbak, és számjegyeik összege
páratlan?

\ki{Fejér Szabolcs}{Miskolc}\medskip

{\bf 5. feladat: } Legyen hat, nem feltétlenül egyforma sugarú, kör
egy síkban. Igazoljuk, hogy ha a hat körnek van közös belső pontja, 
akkor az egyik kör középpontja egy másik
belsejében van.

\ki{Mátyás Mátyás}{Brassó}\medskip

{\bf 6. feladat: } Adott az $ABC$ háromszög, amelyben $AB=AC$ és
$BAC\sphericalangle=20^{\circ}.$ Az $AC$ oldalon felvesszük a $D$
és $E$ pontokat úgy, hogy $AD=BC$ és $BE$ az $ABC\sphericalangle$
szögfelezője. Legyen $F$ és $K$ a $BD,$ illetve $DE$ szakasz
felezőpontja. Bizonyítsuk be, hogy az $EFK_{\triangle}$
egyenlő oldalú.

\ki{Olosz Ferenc}{Szatmárnémeti}\medskip

\vfill
\end{document}
