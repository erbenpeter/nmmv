\documentclass[a4paper,10pt]{article} 
\usepackage[utf8]{inputenc}
\usepackage{t1enc}
\usepackage{graphicx}
\usepackage{amssymb}
\voffset - 20pt
\hoffset - 35pt
\textwidth 450pt
\textheight 650pt 
\frenchspacing 

\pagestyle{empty}
\def\ki#1#2{\hfill {\it #1 (#2)}\medskip}

\def\tg{\, \hbox{tg} \,}
\def\ctg{\, \hbox{ctg} \,}
\def\arctg{\, \hbox{arctg} \,}
\def\arcctg{\, \hbox{arcctg} \,}

\begin{document}
\begin{center} \Large {\em XIX. Nemzetközi Magyar Matematika Verseny} \end{center}
\begin{center} \large{\em Szatmárnémeti, 2010. március 19-22.} \end{center}
\smallskip
\begin{center} \large{\bf 9. osztály} \end{center}
\bigskip 

{\bf 1. feladat: } Mennyi a következő tört értéke?

$$\frac{2010201020102011\cdot4020402040204021-2010201020102010}{2010201020102010\cdot4020402040204021+2010201020102011}$$

\ki{dr. Katz Sándor}{Bonyhád}\medskip

{\bf 1. feladat megoldása: } Látható, hogy a törtkifejezésben szereplő számok a 2010201020102010,
az ennél eggyel nagyobb és a kétszeresénél eggyel nagyobb szám.
Jelöljük ezért a 2010201020102010 számot $n$-nel. Ekkor a tört:
$$\frac{(n+1)(2n+1)-n}{n(2n+1)+n+1}=\frac{2n^2+2n+n+1-n}{2n^2+n+n+1}=$$
$$=\frac{2n^2+2n+1}{2n^2+2n+1}=1.$$

\smallskip\textbf{Megjegyzés: } Ha a $2010201020102011$ számot jelöljük $m$-mel,
akkor a $$\frac{m(2m-1)-(m-1)}{(m-1)(2m-1)+m}=1$$ törtet kapjuk
és ha a $4020402040204021$ számot jelöljük $p$-vel, akkor a
$$\frac{\frac{p+1}{2}p-\frac{p-1}{2}}{\frac{p-1}{2}p+\frac{p+1}{2}}=1$$ törthöz jutunk.
Minden esetben a törtben megjelenő számok közti
összefüggések észrevétele a lényeges.

\medskip
{\bf 2. feladat: } Meseországban fityinggel és fabatkával lehet vásá\-rolni, ahol $1$
fitying $2010$ fabatkát ér. Fajankó és Vasgyúró összehasonlítják
megtakarított pénzüket. Mindketten megszámolják fityingjeiket és
fabatkáikat, majd megállapítják, hogy egyiküknek sincs $2010$-nél
több fityingje, és hogy Vasgyúró va\-gyo\-na $1003$ fabatkával több,
mint Fajankó vagyonának kétszerese. Fajankónak annyi fityingje van,
ahány fabatkája van Vasgyúrónak, és annyi fabatkája, ahány
fityingje van Vasgyúrónak. Mennyi megtakarított pénze van
Fajankónak?

\ki{dr. Péics Hajnalka}{Szabadka}\medskip

{\bf 2. feladat megoldása: }Legyen Fajankónak $x$ fityingje és $y$ fabatkája. Ekkor Vasgyúrónak
$y$ fityingje és $x$  fabatkája van. A megadott feltétel szerint
ekkor
\[2010y+x-1003=2(2010x+y),\]
ahonnan rendezés után $2008(y-2x)=3x+1003$ adódik. Két esetet
különböztetünk meg:

1. eset.  Ha  $y-2x=1$, akkor  $3x+1003=2008$, ahonnan $x=335$ és
$y=671$.

2. eset.  Ha  $y-2x\ge2$, akkor  $3x+1003\ge2\cdot2008=4016$, azaz
$x\ge\frac{3013}{3}$. Ekkor
$y\ge2+2x\ge2+2\cdot\frac{3013}{3}\ge2010$, ami ellentmondást jelent
a feltételekkel.

Tehát Fajankónak 335 fityingje és 671 fabatkája van.

\medskip

{\bf 3. feladat: } Oldjuk meg az
$\displaystyle{\frac{1}{x}-\frac{1}{y}+\frac{1}{z}=\frac{1}{xy}+\frac{1}{yz}-1}$
egyenletet, ha $x,\ y,\ z$ egész számok!

\ki{Bíró Bálint}{Eger}\medskip

{\bf 3. feladat I. megoldása: } Nyilvánvaló, hogy az $x,\ y,\ z $ egész számok egyike sem lehet 0.
Szorozzuk be az egyenletet az $xyz\neq0$  kifejezéssel! Ekkor
először az $yz-xz+xy=z+x-xyz$, majd 0-ra rendezve az
\[yz-z+xyz-xz+xy-x=0\]
egyenletet kapjuk. A bal oldalon $(y-1)$ kiemelhető és így az
$$(y-1)(z+xz+x)=0$$ egyenlethez jutunk.

1. eset. Ha $y-1=0$, akkor $y=1$ és a $z+xz+x$ kifejezés értéke
tetszőleges egész szám lehet, tehát $x$ és $z$ értékének is
tetszőleges, $0$-tól különböző egész számot választhatunk.

2. eset. Ha $z+xz+x=0$, akkor $$z+xz+x+1=(x+1)(z+1)=1.$$ Mivel $x$
és $y$ egész számok, ezért $x+1=1$ és $z+1=1$ vagy  $x+1=-1$ és
$z+1=-1.$ Az első esetben $x=z=0$ adódik és ez nem
lehetséges. A második esetben $x=z=-2$ és
visszahelyettesítve az eredeti egyenletbe látható, hogy
tetszőleges $y\neq0$ esetén teljesül az egyenlőség,
tehát a feladat megoldásainak halmaza
$$M=\left \{(x,1,z)|x,z\in \mathbb{Z}^*\right \}\cup \left \{(-2,y,-2)|y\in\mathbb{Z}^*\right \}.$$

\medskip

{\bf 3. feladat II. megoldása: } A tagok csoportosításával az
$$\frac 1x \left(1-\frac 1y \right)+\frac 1z \left(1-\frac 1y \right)+ \left(1-\frac 1y \right)=0$$
egyenlethez jutunk, tehát az $$ \left(1-\frac 1y \right)\left(
\frac 1x+\frac 1z+1\right)=0$$ egyenletet kell megoldanunk. Ha
$y=1,$ akkor $x$ és $z$ tetszőleges lehet, különben az
$\frac 1x+\frac 1z=1$ egyenlőségnek kell teljesülnie. $|x|>2$
és $|z|>2$ esetén $\frac 1x+\frac 1z<1.$ Ugyanakkor ha $|x|=1$
vagy $|z|=1,$ akkor ellentmondáshoz jutunk, tehát csak a $|x|=2$
esetet kell megvizsgálni. Ebben az esetben az $x=z=-2$
megoldáshoz jutunk, tehát
$$M=\left \{(x,1,z)|x,z\in \mathbb{Z}^*\right \}\cup \left \{(-2,y,-2)|y\in\mathbb{Z}^*\right \}.$$

\medskip


{\bf 4. feladat: } A dubai Burj Khalifa felhőkarcoló $160$ emele\-tes.  Induljon el egy
lift a föld\-szint\-ről, és tételezzük fel, hogy útja során minden
emeleten pontosan egyszer áll meg. Mennyi az a leghosszabb út,
amit eközben megtehet, ha két szomszédos emelet közti
távolság $4$ méter?

\ki{Fejér Szabolcs}{Miskolc}\medskip

{\bf 4. feladat I. megoldása: } Vizsgáljuk meg például, hogy legfeljebb hányszor te\-heti meg a
$[20,21]$  emeleti intervallumot! Ha a lift felfelé mozog, akkor a
,,START'' állomás lehet a $0,1,\dots,20$ sorszámú emelet, a ,,CÉL''
pedig a $21,22,\dots,160.$ Mivel minden emeleten pontosan egyszer
állhat meg, ezért legfeljebb $21$-szer mehet át ezen a szakaszon
úgy, hogy felfele halad. Ha lefelé mozog, akkor a ,,START''
állomás a $21,22,\dots,160$ emeletek, a ,,CÉL'' pedig az
$1,2,\dots,20$ emelet, így lefelé legfeljebb $20$-szor haladhat át.
Ez összesen ma\-xi\-mum $41$ eset. A $[79,80]$ intervallumon felfelé
a ,,START'' lehet $0,1,\dots,79$ (ez $80$ eset), a ,,CÉL'' pedig
$80,81,\dots,160$ ($81$ eset), ezért legfeljebb $80$-szor haladhat
át ezen a szakaszon. Lefelé a ,,START'' ugyancsak 81, a ,,CÉL''
pedig 79, így összesen legfeljebb 80+79-szer mehet át. Ha egy
emelettel feljebb megyünk, a $[80,81]$ szakaszra, akkor felfelé
,,START'' $81$ féle, ,,CÉL'' $80$ féle, tehát maximum $80$-szor
mehet át felfelé menetben. Lefelé ,,START'' $80$ féle, ,,CÉL'' $80$
féle, tehát összesen legfeljebb $80+80$ féleképpen mehet át ezen a
szakaszon. Minden e fölötti szintre igaz, hogy az intervallum fölött
kevesebb megálló van, így ez határozza meg a lehetséges maximális
esetek számát.

Összefoglalva tehát az $[i-1,i]$ szintek között a lift legfeljebb
$2i-1$-szer mehet át, ha $1\leq i\le80$, és $2(161-i)$-szer, ha
$81\leq i\leq 161.$ így a lehető leghosszabb út nem haladhatja meg
a
\[4(1+3+\ldots+159+160+158+\ldots+2)\]
métert, vagyis a $\frac{4\cdot160\cdot161}{2}=51520$ métert.

Másrészt ez létre is jöhet, ha a megállók sorban $0, 160, 1,
159,$ $2,158,\ldots, 79, 81, 80.$

\medskip

{\bf 4. feladat II. megoldása: } Jelölje $x_i$ az $i$-edik
megállónak megfelelő eme\-let számát. Két
egymás\-utáni megálló közti távolság $4|x_i-x_{i+1}|.$
Másrészt
$$|x_i-x_{i+1}|+|x_{i+1}-x_{i+2}|=|x_{i}-x_{i+2}|$$ ha az
$(i+1)$-edik megállónál a lift nem változtat irányt. Ez
azt mutatja, hogy $x_i<x_{i+1}<x_{i+2}$ vagy $x_i>x_{i+1}>x_{i+2}$
ese\-tén növelhető az útvonal hossza, ha az $(i+1)$-edik
megállót máshová tennénk a megállók sorozatában.
így a leghosszabb útvonal esetén (ami létezik, mivel véges
sok különböző útvonal van) az
$$S=4\left(|x_1-x_2|+|x_2-x_3|+\ldots +|x_{160}-x_{161}|\right)$$
kifejezésben a moduluszok felbontása után az $x_i,$ $2\leq
i\leq 160$ számok kétszerese jelenik meg pozitív vagy
negatív előjellel és kifejtésben szereplő
együtthatók összege $0$ (pozitív előjelű $80$ da\-rab
van, mindegyik $+2$ együtthatóval, míg negatív $81$ darab van,
ebből $79$ együtthatója $-2$ és $2$ együtthatója $-1$).
Ez akkor a leg\-na\-gyobb, ha a nagy számok jelennek meg pozitív
előjellel és a ki\-csik negatív előjellel. Pontosabban
akkor a lehető leg\-na\-gyobb, ha a $81,82,\ldots ,160$ számok
vannak pozitív előjellel és a $0,1,2,\ldots, 80$ számok
negatív előjellel. Ez el is érhető ha a megállókat
úgy rakjuk sorba, hogy bármely két egymásutáni megálló
közt ha\-ladjon át a $80$-as szinten. Ebben az esetben ugyanis a
modu\-luszok felbontása után a $80$-nál nagyobbak mind
pozitív előjelűek lesznek és a $80$-nál nem nagyobbak
mind negatív előjelűek.

\medskip

\textbf{Megjegyzés: } A második megoldás mutatja, hogy nemcsak az első
megoldásban megadott sorrend esetén jön ki a leghosszabb
útvonal, hanem sok más esetben is.

\medskip
{\bf 5. feladat: } Az $ABC$ háromszög $AC$ és $BC$ oldalán felvesszük a
$C$-hez, illetve $B$-hez közelebb eső $N$ és $M$
harmadolópontot, valamint az $AB$ oldal $P$ felezőpontját,
majd az eredeti háromszöget kitöröljük. Szer\-kesszük
vissza az $M,N$ és $P$ pont alapján az eredeti háromszöget!

\ki{Kovács Lajos}{Székelyudvarhely}\medskip

{\bf 5. feladat I. megoldása: } Képzeljük el, hogy a háromszög a mellékelt ábrán lát\-ható vég\-telen
rács része. így, ha $Q$ az $N$ szimmetrikusa az $M$-re
nézve, akkor a $PQ$ egyenes egybeesik az $AB$ egyenessel, és $B$
a $PQ$-t öt egyenlő részre osztó pontok közül a
második (a $Q$-tól kezdve). Ez viszont megszerkeszthető és
a szerkesztés a következő lépésekre bontható:
\begin{itemize}\item megszerkesztjük az $N$-nek az $M$-re
vonatkozó $Q$ szimmet\-rikusát;
\item megszerkesztjük a $PQ$ szakasznak azt a $B$ pontját, amelyre $QB=\frac 25 PQ;$
\item a $QP$ meghosszabbításán felmérjük a $PA=PB$
szakaszt;
\item a $BM$ és $AN$ egyenesek metszéspontja a $C$ pont.
\end{itemize}

\centerline{\includegraphics[width=0.8\textwidth]{racs2.eps}}

\medskip

{\bf 5. feladat II. megoldása: } Ha a megfelelő kisbetűkkel jelöljük a
csúcsok helyzetvektorait, akkor $3m=2b+c,$ $3n=2c+a$ és
$2p=a+b.$ így $b=\frac 15 (2p-3n+6m),$ vagyis ha $M$-et
választjuk a helyzetvektorok kezdőpontjának, akkor
$b=\frac{1}{5}(2p-3n)$ és ez éppen az első bizonyításbeli
tulajdonság (vagyis $B$ a $PQ$-t $3:2$ arányban osztó pont).
 Ha nem választjuk $M$-et
kezdőpontnak, akkor az $N$-nek az $O$ kezdőpontra vonatkozó
$Q$ szimmetrikusa esetén megszerkesztendő a $PQ$-nak az $R$
pontja, amely a $PQ$ szakaszt $3:2$ arányban osztja, majd az
$r$-hez hozzá kell adni a $\frac 65m$ vektort.

\medskip



\textbf{Megjegyzés: } A második megoldáshoz hasonló
gondolatme\-ne\-tet használhatunk akkor is, ha ana\-litikus
geometriai eszközöket (koordinátákat) használunk (vagy
akár komplex számokat). A megoldás lényege abban áll, hogy
az eredeti szerkesztés alapján kifejezzük az $A,B$ és $C$
koordinátáinak függ\-vényében az $M,N$ és $P$
koordinátáit, majd a kapott összefüggésekből
kifejezzük az eredeti csúcsok koordinátáit az $M,N,P$
koordinátái függvényében (ez egy egyenletrendszer
megoldása), és végül a kapott összefüggések alapján
megszerkesztjük a pontokat.


\medskip
{\bf 6. feladat: } Színezzük ki egy szabályos $2010$-szög csú\-csait $3$ színnel,
mindhárom színt ugyanannyiszor használva.  Igaz-e, hogy bár\-mely
színezés esetén lesz olyan szabályos háromszög, amelynek vagy minden
csúcsa azonos színű,  vagy a három csúcsa három különböző
színű?

\ki{Fejér Szabolcs}{Miskolc}\medskip

{\bf 6. feladat I. megoldása: } Az állítás nem igaz, ugyanis van olyan színezés, mely esetén minden
szabályos háromszög csúcsai kétféle színűek. Egy ilyen színezést
megkaphatunk, ha számozzuk a csúcsokat $1$-től $2010$-ig (az
óramutató járásával ellentétes sorrendben), majd az
$1,2,\ldots,670$ sorszámú csúcsokat az egyik színnel (pl.
pi\-ros\-sal) színezzük, a következő $335$ csúcsot a második
színnel (pl. kékkel), a következő $335$ csúcsot a harmadik
színnel (pl. sárgával), a következő $335$ csúcsot ismét
a második színnel (kékkel) és végül az utolsó $335$
csúcsot a harmadik színnel (sárgával). A szabályos háromszögek
csúcsainak sorszáma mindig $i,i+670,i+1340$ alakú, ahol  $i\leq
670,$ tehát az egyik csúcs mindig piros és a másik kettő
mindig azonos színű, mivel $i+1340=i+670+2 \cdot 335.$

\medskip

{\bf 6. feladat II. megoldása: } Jelöljük a színeket $p,k$ és
$s$-sel. összesen $670$ egyenlő oldalú háromszög van,
amelynek a csúcsai a sokszög csú\-csai közül kerülnek
ki. Ha ezek között nincs olyan, amelynek azonos színűek a
csúcsai, vagy mind különbözőek, akkor a csú\-csok
szí\-ne\-zése szerint a $670$ háromszög a következő hat
osztályba sorolható:

\centerline{$\{p,k,k\},$ $\{p,s,s\},$ $\{s,p,p\},$ $\{k,p,p\},$
$\{k,s,s\}$ és $\{s,k,k\}$}

\noindent (a $\{p,k,k\}$ azt jelenti, hogy két csúcs $k$
színű és egy csúcs $p$ színű stb.). Ha ezeknek az
osztályoknak a számosságát rendre $a,b,c,d,e$ és  $f$
jelöli, akkor a színek összeszámlálása és a feltétel
alapján írhatjuk, hogy
\begin{equation}\left \{ \begin{array}{ccc}
a+b+2c+2d=670\\
d+e+2f+2a=670\\
c+f+2b+2e=670
\end{array}
\right .
\end{equation}
Ha ebből az egyenletrendszerből kifejezzük az $a$-t, a $b$-t
és a $d$-t a többi változó függvényében, akkor az
\begin{equation}\left \{ \begin{array}{l}
a=335-e-\frac 32 f+\frac c2\\
b=335-e-\frac c2 -\frac f2\\
d=-c+e+f
\end{array}
\right .
\end{equation}
 egyenlőségekhez jutunk. Ez mutatja, hogy több olyan
 színezés is van, amelyben nincs sem egyszínű szabályos
 háromszög, sem olyan, amelynek a csúcsai mind különböz\H
 o színűek. Például $c=f=0$ és $e<335$ esetén
 $a=335-e,$ $b=335-e$ és $d=e$ egy ilyen színezést ír le.
 Világos, hogy a csúcsok színezését elvégezhetjük
 úgy, hogy kiválasztjuk a $670$ egyenlő oldalú
 háromszögből (tetszőlegesen) azt az $a$ darabot, amelyet
 az első osztálynak megfelelően színezünk, aztán a
 maradékból azt a $b$ darabot, amelyet a második osztálynak
 megfelelően színezünk és a többit a negyedik
 osztálynak megfelelően színezzük.

\medskip

\textbf{Megjegyzés: } A második megoldás rávilágít arra, hogy nagyon
sok olyan színezés létezik, amelyben nincs sem egyszínű
szabályos
 háromszög, sem olyan, amelynek a csúcsai mind különböz\H
 o színűek, és ugyanakkor megmutatja az összes ilyen színezés
 szer\-kezetét is.
\end{document}
