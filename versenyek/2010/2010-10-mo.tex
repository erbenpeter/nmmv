\documentclass[a4paper,10pt]{article} 
\usepackage[utf8]{inputenc}
\usepackage{t1enc}
\usepackage{graphicx}
\usepackage{amssymb}
\usepackage{pstricks, pstricks-add}
\voffset - 20pt
\hoffset - 35pt
\textwidth 450pt
\textheight 650pt 
\frenchspacing 

\pagestyle{empty}
\def\ki#1#2{\hfill {\it #1 (#2)}\medskip}

\def\tg{\, \hbox{tg} \,}
\def\ctg{\, \hbox{ctg} \,}
\def\arctg{\, \hbox{arctg} \,}
\def\arcctg{\, \hbox{arcctg} \,}

\begin{document}
\begin{center} \Large {\em XIX. Nemzetközi Magyar Matematika Verseny} \end{center}
\begin{center} \large{\em Szatmárnémeti, 2010. március 19-22.} \end{center}
\smallskip
\begin{center} \large{\bf 10. osztály} \end{center}
\bigskip 

{\bf 1. feladat: }  Legalább hány szeget kell beütni az alábbi
farács rácspontjaiba ahhoz, hogy biztosan legyen köztük $4$
szeg, amely egy téglalapot feszít ki?

\smallskip
\centerline{
\psset{xunit=0.5cm,yunit=0.5cm}
\begin{pspicture}(12,4)
\psgrid[gridlabels=0pt,subgriddiv=0]
\end{pspicture}
}

\ki{Nagy Örs}{Kolozsvár}\medskip


\textbf{1. feladat megoldása: } Tételezzük fel, hogy a lehető legtöbb szeget beütöttük a rácspontokba anélkül, hogy azok valamilyen
téglalapot feszíte\-nének ki. Jelöljük $m$-mel az egy
függőleges (lásd a mellékelt ábrát) rácsegyenesre
illeszkedő szegek maximális számát.

Ha $m=5,$ akkor $12$ függőleges rácsegyenes csak egy-egy
szeget tartalmazhat, különben keletkezne olyan téglalap,
amely\-nek olda\-lai párhuzamosak a rácsegyenesekkel és a
csúcsaiban van egy-egy szeg. így $m=5$ esetén $18$ szeg már
biztosítaná a téglalap keletkezését.

Ha $m=4,$ akkor további $4$ függőleges rácsegyenes
tartalmazhat $2-2$ szeget és a többi legfeljebb $1$-et
(különben ismét keletkezne olyan téglalap, amelynek oldalai
a rácsegyenesekkel párhuza\-mo\-sak). így ebben az esetben
$21$ szeg biztosítaná a tég\-la\-lap keletke\-zé\-sét.


Ha $m=3,$ akkor két esetet kell megvizsgálni aszerint, hogy
hány függőleges rács\-egyenes tartalmaz $3$ szeget. Ha
két ilyen rács\-egyenes is van, akkor csak további $4$
függőleges rácsegyenes tartalmazhat $2$ szeget és az
összes többi legfeljebb $1$-et (különben ismét keletkezne
olyan téglalap, amelynek oldalai a rácsegyenesekkel
párhuzamosak). Ebben az esetben $22$ szeg esetén már
megjelenne legalább egy téglalap. Ha csak egy függőleges
rács\-egyenes tartalmaz $3$ szeget, akkor további $7$ függ\H
oleges rácsegyenes tartalmazhat $2-2$ szeget és az összes
többi legfeljebb $1$-et. Ebben az esetben $23$ szeg esetén
megjelenne olyan téglalap, amelynek oldalai párhuzamosak a
rácsegyenesekkel.

Ha $m=2,$ akkor mivel az $5$ vízszintes egyenesből $10$
féleképpen választhatunk ki kettőt, ezért $10$ függ\H
oleges rácsegyenesen lehet $2-2$ pont és a többin $1-1.$ Ez
összesen $23$ pont, tehát itt csak $24$ pont esetén jelenne
meg olyan téglalap, amelynek oldalai párhuzamosak a
rácsegyenesekkel. Ugyanakkor természetes, hogy más
téglalapok is keletkezhetnek, tehát ahhoz, hogy a megoldás
teljes legyen kell egy olyan elhelyezést találni a $23$ szegre,
amikor semmilyen téglalapot nem feszítenek ki (tehát nemcsak
olyat nem, amelynek az oldalai a rácsegyenesekkel párhuzamosak).
Egy ilyen elhelyezés látható a mellékelt ábrán.

\centerline{\includegraphics[width=0.45\textwidth]{szegek.eps}}

Ha $m=1,$ akkor világos, hogy legfeljebb $13$ szeg lenne a
farácson, tehát a feladatban megfogalmazott kérdésre a
válasz $24.$

\medskip

{\bf 2. feladat: } Határozzuk meg azokat
az $x,y\in \mathbb{N}$ számokat, amelyekre
$$xy(x-y)=13x+15y.$$

\ki{Kacsó Ferenc}{Marosvásárhely}\medskip

\textbf{2. feladat I. megoldása: } A $(0,0)$ megoldása az egyenletnek és ha $x,y$ közül az
egyik nulla, akkor a másik is nulla, tehát a továbbiakban
feltételezhetjük, hogy $x\neq 0\neq y.$ Két esetet tárgyalunk
aszerint, hogy $y$ osztható $13$-mal vagy sem.

1. eset. Ha $y$ osztható $13$-mal, akkor létezik
$a\in\mathbb{N}$ úgy, hogy $y=13a,$ tehát $x=a(x^2-13ax-15).$ Ez
csak akkor lehetséges ha $a|x,$ vagyis létezik $b\in\mathbb{N}$
úgy, hogy $x=ab.$ Visszahelyettesítés után a
$$15=b(a^2b-13a^2-1)$$ egyenlőséghez jutunk és innen következik, hogy $b|15,$ ezért
$b\in\{1,\ 3,\ 5,\ 15\}.$ A négy eset kipróbálása után
csak $b=15$ esetén kapunk $a$-nak is természetes értéket,
tehát $x=15$ és $y=13.$

2. eset. Ha $y$ nem osztható $13$-mal, akkor
$$13x=y(x^2-xy-15)$$ alakba írható, és mivel $y$ relatív prím a $13$-mal, az $x$ osztható kell legyen az $y$-nal.
így létezik $k\in\mathbb{N}$ úgy, hogy $x=ky$ és ezt
visszahelyettesítve a $15=k(ky^2-y^2-13)$ egyenlethez jutunk,
ahonnan $k|15,$ tehát $k\in\{1,\ 3,\ 5,\ 15\}.$ Innen a megoldások
$x=9,\ y=3;\ x=10,\ y=2;\ x=15,\ y=1.$

összesítve, az egyenletnek a következő megoldásai
lehetségesek:
$$M=\{(0,0),(15,13),(9,3),(10,2),(15,1)\}.$$

\medskip

\textbf{2. feladat II. megoldása: } Az egyenlet alapján $x|15y$
és $y|13x.$ Ha $d$ az $x$ és $y$ legnagyobb közös osztója
és $d=1,$ akkor $y\in \{1,13\}.$ $y=1$ esetén $x=15$ és $y=13$
esetén szintén az $x=15$ megoldáshoz jutunk. Ha $d>1,$ akkor
létezik olyan $x_1,y_1\in \mathbb{N},$ amelyre $x_1$ és $y_1$
relatív prímek valamint  $x=dx_1$ és $y=dy_1,$ tehát az
egyenlet $$d^2x_1y_1(x_1-y_1)=13x_1+15y_1$$ alakban írható. Ez
alapján $y_1|13,$ tehát $y_1\in \{1,13\}.$

$y_1=1$ esetén az $$x_1\left(d^2(x_1-1)-13 \right )=15$$
egyenlethez jutunk, ahonnan $x_1\in \{1,3,5,15\}.$ Az $x_1=1$ eset
nem felel meg és a többi esetből rendre a $d=3,$ $d=2,$
illetve $d=1$ értékekhez jutunk.

$y_1=13$ esetén az $$x_1\left (d^2(x_1-13)-1 \right )=15$$
egyenlethez jutunk, ahonnan $x_1\geq 14$ és $x_1|15,$ tehát csak
az $x_1=15$ esetet szükséges vizsgálni. Ebben az esetben
$d=1,$ tehát nem jutunk újabb megoldáshoz. összesítve

\centerline{$M=\{(0,0),(15,13),(9,3),(10,2),(15,1)\}.$}
%\vspace*{-1cm}

\medskip


{\bf 3. feladat: }  Adott a síkon négy pont úgy, hogy közülük semelyik három sincs egy egyenesen.
Kiszíneztük a négy pontot négy színnel: pirossal, kékkel, zölddel,
és sárgával. Ezután kiszíneztük a pontok által meghatá\-rozott
szaka\-szo\-kat is úgy, hogy azok színe megegyezett valamelyik
vég\-pont\-juk színével, és közben mind a négy színt újra
felhasználtuk. Igaz-e, hogy mindig van olyan pont, hogy

(1) vagy a belőle kiinduló szakaszok közül,

(2) vagy a másik három pont közti szakaszok közül

\noindent az egyik piros, a másik kék, a harmadik zöld?

\ki{dr. Kántor Sándorné}{Debrecen}\medskip

\textbf{3. feladat megoldása: } Tekintsük a piros, kék és zöld pontokat. Ha az ezek által
meghatározott három szakasz piros, kék, illetve zöld színű, akkor a
feladat második állítása igaz a sárga pontra. Vizsgáljuk azokat az
eseteket, amelyekre nem teljesül a (2) állítás a sárga pontra. Mivel
minden szakasz színe megegyezik valamelyik végpontjának a színével,
ez csak úgy lehet, ha a piros, kék és zöld pontok által
meghatározott 3 szakasz közül 2 azonos színű. Mivel a színek szerepe
azonos, feltehetjük, hogy a 3 szakasz közül 2 piros, 1 pedig kék
színű. Tekintsük ezután a zöld pontot. Mivel minden pontból indul
saját színű szakasz (hisz a színezés során mind a $4$ színt
felhasználjuk), ezért a zöld pontból indul ki zöld szakasz. Ezek
szerint a zöld pontban egy piros, egy kék és egy zöld szakasz
találkozik, tehát a zöld pontra az (1) állítás igaz.

\medskip


{\bf 4. feladat: } Mennyi azoknak a pozitív egészeknek az összege,
amelyek $2010$-nél nem nagyobbak, és számjegyeik összege
páratlan?

\ki{Fejér Szabolcs}{Miskolc}\medskip

\textbf{4. feladat megoldása: } Jelölje $S(n)$ az $n$ szám számjegyeinek összegét! Legyen $n<1000.$
Ekkor két fontos állítást fogalmazhatunk meg.

1) Ha $S(n)$ páros, akkor $S(999-n)$ páratlan (a kivonásnál sehol
sincs átvitel).

2) Ha $S(n)$ páros, akkor $S(1000+n)$ páratlan.

Tehát az első állítás miatt a $H=\{0,\ 1,\dots,999\}$ halmaz elemei
közül pontosan 500 páros összegű, 500 páratlan. A második állítás
miatt a $H$ halmaz páros összegű számaihoz 1000-t adva kapunk
páratlan összegű számot (szintén 500-at), ezek lesznek a $K=\{1000,\
1001,\dots,1999\}$ halmaz páratlan összegű számai. A kívánt összeget
2000-ig, úgy kapjuk, hogy a $H$ halmaz elemeinek összegéhez
hozzáadunk $500\cdot1000$-t. A keresett összeg tehát
\[\frac{1000\cdot999}{2}+500\cdot1000+2001+2003+2005+2007+2009+2010,\] vagyis
$1011535.$

\medskip


{\bf 5. feladat: } Legyen hat, nem feltétlenül egyforma sugarú, kör
egy síkban. Igazoljuk, hogy ha a hat körnek van közös belső pontja, 
akkor az egyik kör középpontja egy másik
belsejében van.

\ki{Mátyás Mátyás}{Brassó}\medskip

\textbf{5. feladat megoldása: } Legyen $P$ egy pont, amelyik mind a hat kör belsejében megtalálható.
Az egyik tetszőlegesen választott körtől elindulva, és az óramutató
járásával ellentétes irányba haladva $P$ körül, jelölje
$\mathcal{C}_i(O_i,R_i),1\leq i\leq 6$ a köröket. Ekkor
\[m(\widehat{O_1PO_2})+m(\widehat{O_2PO_3})+\dots+m(\widehat{O_5PO_6})+m(\widehat{O_6PO_1})=2\pi.\]
Ha mind a hat fenti szög mértéke megegyezik $\frac{\pi}{3}$-mal
akkor az $O_iPO_{i+1},$ $1\leq i\leq 6$ ($O_7=O_1$)
háromszögekben
$$O_iO_{i+1}\leq \max\{PO_i,PO_{i+1}\}<\max \{R_i,R_{i+1}\}.$$  Ha
$R_i\le R_{i+1},$ akkor az $O_i$ pont a
$\mathcal{C}_{i+1}(O_{i+1},R_{i+1})$ kör belsejében található,
különben az $O_{i+1}$ pont található a $\mathcal{C}_i(O_i,R_i)$
kör belsejében.

Ha nem mind a hat szög mértéke $\frac{\pi}{3},$ akkor
létezik $1\leq i\leq 6$ úgy, hogy
$m(\widehat{O_iPO_{i+1}})<\frac{\pi}{3}$. Az $O_iO_{i+1}P$
háromszögben teljesül, hogy
\[m(\widehat{O_iPO_{i+1}})<\max\{m(\widehat{PO_iO_{i+1}}),\ m(\widehat{PO_{i+1}O_i})\}\]

Mivel a háromszögben nagyobb szöggel szemben nagyobb ol\-dal
fekszik, ezért az $O_iO_{i+1}P$ háromszögben teljesül, hogy
$$O_iO_{i+1}<\max\{PO_i,\ PO_{i+1}\}.$$ Ugyanakkor a $P$ pont a
$\mathcal{C}_i(O_i,R_i)$ és a $\mathcal{C}_{i+1}(O_{i+1},R_{i+1})$
körök belsejében található, tehát $PO_i<R_i$ és
$PO_{i+1}<R_{i+1}.$ így
\[O_iO_{i+1}<\max\{R_i, R_{i+1}\}.\]
Ha $R_i\le R_{i+1},$ akkor az $O_i$ pont a
$\mathcal{C}_{i+1}(O_{i+1},R_{i+1})$ kör belsejében található,
különben az $O_{i+1}$ pont található a $\mathcal{C}_i(O_i,R_i)$
kör belsejében.

\medskip


{\bf 6. feladat: } Adott az $ABC$ háromszög, amelyben $AB=AC$ és
$BAC\sphericalangle=20^{\circ}.$ Az $AC$ oldalon felvesszük a $D$
és $E$ pontokat úgy, hogy $AD=BC$ és $BE$ az $ABC\sphericalangle$
szögfelezője. Legyen $F$ és $K$ a $BD,$ illetve $DE$ szakasz
felezőpontja. Bizonyítsuk be, hogy az $EFK_{\triangle}$
egyenlő oldalú.

\ki{Olosz Ferenc}{Szatmárnémeti}\medskip

\textbf{6. feladat I. megoldása: } Segédszerkesztést végzünk:
megszerkesztjük az\linebreak  $ABC$ háromszöggel kongruens
(egybevágó) $LDA$ háromszöget ($L$ és $B$ az $AC$ egyeneshez
viszonyítva ugyanabban a félsíkban helyezkednek el).

Az $ABC$ és $LDA$ egyenlő szárú háromszögekben
$$AB=AC=LA=LD,$$ az alapokon fekvő szögek $80^{\circ}$-osak. $AL=AB$
és $m(\widehat{LAB})=80^{\circ}-20^{\circ}=60^{\circ}$, innen
következik, hogy az $ALB_{\triangle}$ egyenlő oldalú, tehát
$LA=LB=LD$ és $m(\widehat{DLB})=60^{\circ}-20^{\circ}=40^{\circ}$.

\centerline{\includegraphics[width=0.4\textwidth]{olosz1.eps}}


Az $LBD$ egyenlő szárú háromszögben az alapon fekvő szögek
$m(\widehat{LBD})=m(\widehat{LDB})=70^{\circ}$, ahonnan következik,
hogy
$$m(\widehat{BDC})=180^{\circ}-(80^{\circ}+70^{\circ})=30^{\circ}$$ és
$m(\widehat{ABD})=70^{\circ}-60^{\circ}=10^{\circ}$, így
$m(\widehat{DBE})=\frac{80^{\circ}}{2}-10^{\circ}=30^{\circ}.$

Tehát $EBD$ egyenlő szárú háromszög és a $BD$ alap felező\-pontja
$F$, így $EF$ merőleges a $BD$-re, ahonnan következik, hogy $FK$ az
$EFD$ derékszögű háromszögben oldalfelező (súlyvonal), tehát $FK=KE$
és $m(\widehat{FED})=90^{\circ}-30^{\circ}=60^{\circ}$ vagyis $EFK$
egyenlő oldalú háromszög.

\medskip

\textbf{6. feladat II. megoldása: } Gondolkodjunk visszafele: Mire
lenne szükség ahhoz, hogy $EFK$ egyen\-lő oldalú legyen?
Mivel $BE$ felezi az $ABC$ szöget, ezért
$m(\widehat{EBC})=40^{\circ},$ tehát $m(\widehat{BEC})=60^{\circ}$
és így az $\widehat{FEK}$ mértéke pontosan akkor lenne
$60^{\circ},$ amikor $FE$ szögfelező is a $DEB$
háromszögben. Mivel $F$ a $BD$ felezőpontja, ez pontosan
akkor teljesül, ha a $BED$ háromszög egyenlő szárú.
Ehhez elégséges belátni, hogy a $\widehat{DBE}$ mértéke
$30^{\circ}$ vagy az $\widehat{ABD}$ mértéke $10^{\circ}.$
Tehát a feladat visszavezetődik a következő
tulajdonságra:

Tekintjük az $ABC$ háromszöget, amelyben $AB=AC$ és
\linebreak $m(\widehat{BAC})=20^{\circ}.$ Az $AC$ oldalon
felvesszük a $D$ pontot. Igazoljuk, hogy az $AD=BC$ egyenl\H
oség pontosan akkor teljesül, ha $m(\widehat{ABD})=10^{\circ}.$

Amiatt, hogy a $D$ pont egyértelműen szerkeszthető az
$AD=BC$ egyenlőség alapján is és a
$m(\widehat{ABD})=10^{\circ}$ egyenlőség alapján is,
elégséges igazolni, hogy ha $m(\widehat{ABD})=10^{\circ},$ akkor
$AD=BC.$ Ez belátható a szinusztétel segítségével, hisz
az $ABD$ háromszögben $AD=AB\frac{\sin 10^{\circ}}{\sin
30^{\circ}}.$ Másrész $\frac{BC}{2}=AB\sin 10^{\circ},$ tehát
$AD=BC.$

\medskip

\textbf{Megjegyzés: } Az $AD=BC$ egyenlőség belátható csak
kong\-ru\-encia segítségével, ha megszerkesztjük az
$A$-ból a $BD$-re és a $BC$-re húzott merőlegeseket.

\medskip
\textbf{6. feladat III. megoldása: } Az $ABC$ háromszög
tekinthető egy szabályos $18$ oldalú sokszög részének,
amint a mellékelt ábra mutatja ($A$ a sokszög köré írt
kör középpontja, $B$ és $C$ két egymás melletti
csúcs). Ha ebben a sokszögben meghúzzuk az $X_1X_9$ átlót
és az $AX_5$ szakaszt, akkor $AX_5 \perp X_1X_9$ (mert
$X_5X_1=X_5X_9$). Ugyan\-akkor az $X_1X_9$ és $X_2X_{11}$
átlók szöge $30^{\circ}$-os, tehát ha \linebreak
$\{D\}=X_1X_9\cap X_2X_{11}$ és $\{M\}=AX_5\cap X_1X_9,$ akkor az
$ADM$ derékszögű háromszögben $AM=\frac{AD}{2}.$
Másrészt az $X_1X_9X_{10}$ háromszögben $AM$ középvonal,
tehát $AM=\frac{X_9X_{10}}{2}.$ Ez alapján
$AD=X_9X_{10}=X_1X_2=BC,$ tehát az $X_2X_{11}$ és $X_1X_9$
átlók metszéspontja megegyezik a feladatban (az $AD=BC$
feltétel alapján) megszerkesztett $D$ ponttal. így világos,
hogy $m(\widehat{DBE})=m(\widehat{EDB})=30^{\circ},$ és ez
alapján következik, hogy az $EFK$ háromszög egyenlő
oldalú.

\centerline{\includegraphics[width=0.7\textwidth]{sokszog18.eps}}

\medskip


\textbf{6. feladat IV. megoldása: } Szerkesszük meg az $M$ pontot
úgy, hogy \linebreak $m(\widehat{MAB})=80^{\circ}$ és $MB=AB$
($M$ és $C$ az $AB$-hez viszonyítva ugyanabban a félsíkban
van). A szerkesztés alapján a $BAM$ háromszög egybevágó
az $ABC$ háromszöggel, tehát $MA=BC.$ A szerkesztés
alapján $m(\widehat{MAC})=60^{\circ},$ tehát az $MAD$
háromszög egyenlő oldalú. így $D$ és $B$ az $MA$
oldalfelező merőlegesén van, tehát a $BAM$ egyenlő
szárú háromszögben $BD$ az $ABM$ szög szögfelezője.
Ez alapján  $m(\widehat{ABD})=10^{\circ},$ tehát
$m(\widehat{BDE})= m(\widehat{DBE})=30^{\circ}$ és ebből
következik, hogy $EFK_{\triangle}$ egyenlő oldalú.

\centerline{\includegraphics[width=0.9\textwidth]{sokszog18_2.eps}}

\medskip

\textbf{Megjegyzések: } A jobb oldali ábrán látható, hogy a negyedik megoldás alapötlete ugyanaz, mint a
harmadik megoldás alapöt\-lete, pontosabban a szabályos $18$
oldalú sokszögbe való beágyazás. A két beágyazás a
sokszögnek a háromszöghöz viszonyított hely\-zetében
különbözik.

\smallskip
A második megoldásban, ha nem a fordított irányt
vizsgáljuk, hanem az $AD=BC$ feltétel alapján szeretnénk
meghatározni a $\widehat{ABD}$ szög mértékét, akkor az
$\alpha=m(\widehat{ABD})$ jelöléssel a $$\frac{\sin
\alpha}{\sin(\alpha+20^{\circ})}=\frac{\sin 10^{\circ}}{\sin
30^{\circ}}$$ egyenlethez jutunk (az $ABD_{\Delta}$ és az eredeti
háromszögben felírt szinusztétel alapján).
Származtatással a $\tg \alpha=\tg 10^{\circ}$ egyenlőséget
kapjuk, ahonnan $\alpha=10^{\circ}.$


\end{document}
