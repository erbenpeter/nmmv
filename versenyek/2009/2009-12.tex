\documentclass[a4paper,10pt]{article} 
\usepackage[utf8]{inputenc}
\usepackage{t1enc}
\usepackage{graphicx}
\usepackage{amssymb}
\usepackage{pstricks, pstricks-add}
\voffset - 20pt
\hoffset - 35pt
\textwidth 450pt
\textheight 650pt 
\frenchspacing 

\pagestyle{empty}
\def\ki#1#2{\hfill {\it #1 (#2)}\medskip}

\def\tg{\, \hbox{tg} \,}
\def\ctg{\, \hbox{ctg} \,}
\def\arctg{\, \hbox{arctg} \,}
\def\arcctg{\, \hbox{arcctg} \,}

\begin{document}
\begin{center} \Large {\em XVIII. Nemzetközi Magyar Matematika Verseny} \end{center}
\begin{center} \large{\em Gyula, 2009. március 12-16.} \end{center}
\smallskip
\begin{center} \large{\bf 12. osztály} \end{center}
\bigskip 

{\bf 1. feladat: }
Igazoljuk, hogy tetszőleges $x$ valós számra teljesülnek a következő
egyenlőtlenségek!
$$ -\frac 54\leq \sin{x}+\cos{x}+\sin{2x}\leq 1+\sqrt{2}. $$

\ki{Kovács Béla }{Szatmárnémeti}\medskip

{\bf 2. feladat: } 
2009 számjegyei három ,,köz''-t határoznak meg: $2\_0\_0\_9$.
A számon a következő átalakítást végezzük: kiválasztunk egy tetszőleges 10-es
számrendszerbeli számjegyet, az első közbe beírjuk, a második közbe kétszer írjuk be, a
harmadik közbe háromszor. Így egy következő számhoz jutunk. Ez persze hosszabb és így
számjegyei több közt határoznak meg. Újból elvégezzük a fenti átalakítást: újból választunk
egy számjegyet és a közökbe ezt írjuk (az $i$-edik közbe $i$ darabot). Ezt az eljárást folytatjuk.
Igazoljuk, hogy eljárásunk során soha sem kaphatunk 3-mal osztható számot.


\ki{Bíró Bálint }{Eger}\medskip

{\bf 3. feladat: } 
Jelölje $AC$ és $BD$ az egység sugarú kör két merőleges átmérőjét.Az $AB, BC,
CD$ és $DA$ negyedköríveken felvesszük a $P, Q, R$ és $T$ pontokat úgy, hogy $APBQCRDT$ egy
konvex nyolcszög lesz. Hogyan válasszuk meg a $P, Q, R, T$ pontokat ahhoz, hogy a kialakított
nyolcszög oldalainak négyzetösszege minimális legyen.

\ki{Bíró Bálint }{Eger}\medskip

{\bf 4. feladat: }
Az $a_1, a_2, a_3, a_4,\dots, a_{101}, a_{102}$ az $1, 2, 3, 4,\dots, 101, 102$ számok egy
tetszőleges sorbaállítása. Igazoljuk, hogy az $a_1+1, a_2+2, a_3+3, a_4+4,\dots, a_{101}+101, a_{102}+102$
számok közt lesz két olyan, amelyek 102-vel osztva azonos maradékot adnak!

\ki{Balázsi Borbála }{Beregszász}\medskip

%\eject
{\bf 5. feladat: }
Egy $ABCD$ négyzet alakú papír $A$ csúcsát a $BC$ oldal egy $X$ belső pontjához
mozgatjuk és a papírlapot behajtjuk. A behajtott $AD$ oldal az ábrán látható módon a $C$
csúcsnál levág egy $XEC$ háromszöget. Hogyan válasszuk meg az $X$ pontot ahhoz, hogy a
levágott háromszög beírt körének sugara a lehető legnagyobb legyen?

\centerline{\psset{xunit=0.4cm,yunit=0.4cm,algebraic=true,dotstyle=o,dotsize=3pt 0,linewidth=0.8pt,arrowsize=3pt 2,arrowinset=0.25}
\begin{pspicture*}(1.27,-3.63)(12.71,6.54)
\psline(2.98,-3)(11.1,-3)
\psline(2.98,-3)(2.98,5.12)
\psline(11.1,5.12)(11.1,-3)
\psline(2.98,5.12)(11.1,5.12)
\psline(11.1,0.81)(7.93,-3)
\psline(4.85,6)(11.1,0.81)
\psline(4.85,6)(4.12,5.12)
\psline(4.12,5.12)(7.93,-3)
\psdots[dotstyle=*](2.98,-3)
\rput[bl](2,-3.29){$A$}
\psdots[dotstyle=*](11.1,-3)
\rput[bl](11.3,-3.14){$B$}
\psdots[dotstyle=*](2.98,5.12)
\rput[bl](2.68,5.26){$D$}
\psdots[dotstyle=*](11.1,5.12)
\rput[bl](11.2,5.26){$C$}
\psdots[dotstyle=*](11.1,0.81)
\rput[bl](11.37,0.68){$X$}
\psdots[dotstyle=*](5.91,5.12)
\rput[bl](6,5.26){$E$}
\end{pspicture*}}
 
\ki{Egyed László }{Baja}\medskip

{\bf 6. feladat: } 
Egy $n$ elemű halmaz három elemű részhalmazaiból kiválasztunk néhányat
úgy, hogy semelyik három ne tartalmazzon egynél több közös elemet. Igazoljuk, hogy a
kiválasztott hármasok száma nem haladhatja meg $\displaystyle{\frac {n(n-1)}{3}}$-at!

\ki{Róka Sándor }{Nyíregyháza}\medskip

\vfill
\end{document}
