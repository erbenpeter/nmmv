\documentclass[a4paper,10pt]{article} 
\usepackage[utf8]{inputenc}
\usepackage{t1enc}
\usepackage{graphicx}
\usepackage{amssymb}
\voffset - 20pt
\hoffset - 35pt
\textwidth 450pt
\textheight 650pt 
\frenchspacing 

\pagestyle{empty}
\def\ki#1#2{\hfill {\it #1 (#2)}\medskip}

\def\tg{\, \hbox{tg} \,}
\def\ctg{\, \hbox{ctg} \,}
\def\arctg{\, \hbox{arctg} \,}
\def\arcctg{\, \hbox{arcctg} \,}

\begin{document}
\begin{center} \Large {\em XVIII. Nemzetközi Magyar Matematika Verseny} \end{center}
\begin{center} \large{\em Gyula, 2009. március 12-16.} \end{center}
\smallskip
\begin{center} \large{\bf 11. osztály} \end{center}
\bigskip 

{\bf 1. feladat: }
Állítsuk öt párba az $1, 2, 3, 4, 5, 6, 7, 8, 9, 10$ számokat úgy, hogy a
párokban lévő számok különbségeinek abszolút értékei rendre $1, 2, 3, 4, 5$-t adjanak!
Megtehető-e ez a párosítás (természetesen hat párba), ha az $1, 2, 3, 4, 5, 6, 7, 8, 9, 10, 11, 12$
számokkal dolgozunk, úgy hogy ezek a különbségek $1, 2, 3, 4, 5, 6$ legyenek? Indokoljuk a
választ!

\ki{Hajnal Péter }{Szeged}\medskip

{\bf 2. feladat: } 
Létezik-e két olyan egymástól különböző, pozitív racionális szám,
amelyeknek számtani, mértani és harmonikus közepe egy derékszögű háromszög
oldalhosszai?

\ki{Olosz Ferenc }{Szatmárnémeti}\medskip

{\bf 3. feladat: } 
Egy osztály minden tanulója vagy úszik, vagy kosarazik, esetleg
mindkettőt csinálja.
Lehetséges-e, hogy az osztályban több a lány, mint a fiú a következő esetekben:

a) ha az úszóknak és a kosarasoknak is $60 \%$-a fiú?

b) ha az úszók $60 \%$-a és a kosarasok $75 \%$-a fiú ?


\ki{Katz Sándor }{Bonyhád}\medskip

{\bf 4. feladat: }
Legyen $ABCD$ egy olyan téglalap, amelybe szabályos háromszög írható
úgy, hogy a háromszög egyik csúcsa az $A$ pont, a másik kettő pedig a téglalap egy-egy olyan
oldalán fekszik, amelyen az $A$ pont nincs rajta. Bizonyítsuk be, hogy ekkor a téglalapból a
szabályos háromszög által lemetszett háromszögek egyikének a területe a két másik lemetszett
háromszög területének összegével egyenlő!

\ki{Pintér Ferenc }{Nagykanizsa}\medskip

{\bf 5. feladat: }
Bizonyítsuk be, hogy a $3^k+3^n$ alakban felírt négyzetszámokból végtelen
sok van, ahol $k$ és $n$ különböző pozitív egész számok! Mi a helyzet, ha a 3 helyett a 4, az 5, a
6 és a 7 számokat írjuk?
 
\ki{Kántor Sándor }{Debrecen}\medskip

{\bf 6. feladat: } 
Adott háromszögbe szerkesztettünk két egybevágó, közös belső pont
nélküli, maximális sugarú kört. Mekkora ez a sugár? Hogyan történhet a szerkesztés?


\ki{Bogdán Zoltán }{Cegléd}\medskip

\vfill
\end{document}
