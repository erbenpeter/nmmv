\documentclass[a4paper,10pt]{article} 
\usepackage[utf8]{inputenc}
\usepackage[a4paper]{geometry}
\usepackage[magyar]{babel}
\usepackage{t1enc}
\usepackage{amsmath}
\usepackage{amssymb}
\usepackage{caption}
\usepackage{pgf,tikz}
\usepackage{pstricks,pstricks-add}
\frenchspacing 
\pagestyle{empty}
\newcommand{\ki}[2]{\hfill {\it #1 (#2)}\medskip}
\newcommand{\vonal}{\hbox to \hsize{\hskip2truecm\hrulefill\hskip2truecm}}
\newcommand{\degre}{\ensuremath{^\circ}}
\newcommand{\tg}{\mathop{\mathrm{tg}}\nolimits}
\newcommand{\ctg}{\mathop{\mathrm{ctg}}\nolimits}
\newcommand{\arc}{\mathop{\mathrm{arc}}\nolimits}
\renewcommand{\vec}[1]{\mathbf{#1}}

\begin{document}


\begin{center} \Large {\em XVIII. Nemzetközi Magyar Matematika Verseny} \end{center}
\begin{center} \large{\em Gyula, 2009. március 12--16. } \end{center}
\smallskip
\begin{center} \large{\bf 11. osztály} \end{center}
\bigskip 


{\bf 1. feladat:} Állítsuk öt párba az $1, 2, 3, 4, 5, 6, 7, 8, 9, 10$ számokat úgy, hogy a
párokban lévő számok különbségeinek abszolút értékei rendre $1, 2, 3, 4, 5$-t adjanak!
Megtehető-e ez a párosítás (természetesen hat párba), ha az $1, 2, 3, 4, 5, 6, 7, 8, 9, 10, 11, 12$
számokkal dolgozunk, úgy hogy ezek a különbségek $1, 2, 3, 4, 5, 6$ legyenek? Indokoljuk a
választ!

\ki{Hajnal Péter }{Szeged}\medskip

{\bf Megoldás:} Az {\bf első esetben a kért párosítás lehetséges.} Íme egy megoldás:

$$(2-1);(9-7);(6-3);(8-4);(10-5)$$

A második esetben úgy kell kialakítanunk a párokat, hogy a párokban szereplő számok különbségeinek abszolút értékei rendre $1, 2, 3, 4, 5, 6$ legyenek.

Legyen $x_k$ a $k$-dik párban szereplő két szám közül a kisebb, így a másik szám $x_k+k$ lesz, ahol $k\in {1; 2; 3; 4; 5; 6}$- A kapott párokban szereplő számokat összeadva 
$$(2x_1+1), (2x_2+2), (2x_3+3), (2x_4+4), (2x_5+5), (2x_6+6)$$ 
számokat kapunk, melyeknek összege éppen $1+2+3+...+12$ kell legyen. Ha $X$-szel jelöljük a $\sum\limits_{k=1}^{6} x_k$ összeget, akkor

$$\sum_{k=1}^6 (2x_k+k)=2\sum_{k=1}^6 x_k+\sum_{k=1}^6 k=2X+\frac{6\cdot 7}{2},$$

innen

$$2X+\frac{6\cdot 7}{2}=\frac{12\cdot 13}{2} \Rightarrow 2X+21=78 \Rightarrow 2X=57.$$

A kapott egyenletnek nincs megoldása a természetes számok halmazán, tehát a {\bf második esetben a kért párosítás nem lehetséges.}

\medskip
\vonal

{\bf 2. feladat:} Létezik-e két olyan egymástól különböző, pozitív racionális szám,
amelyeknek számtani, mértani és harmonikus közepe egy derékszögű háromszög
oldalhosszai?

\ki{Olosz Ferenc }{Szatmárnémeti}\medskip

{\bf Megoldás:} Legyen $a, b \in \mathbb{Q}$ úgy, hogy $a>b>0$. Az $\frac{a+b}{2}>\sqrt{ab}>\frac{2ab}{a+b}$ egyenlőtlenségek alapján az átfogó hossza csak $\frac{a+b}{2}$ lehet.

Felírjuk a Pitagorasz-tételt $\left(\frac{a+b}{2}\right)^2=\left(\sqrt{ab}\right)^2+\left(\frac{2ab}{a+b}\right)^2$, majd közös nevezőre hozás után, egyenértékű átalakításokkal rendre kapjuk, hogy

\begin{eqnarray*}
(a+b)^4&=&4ab(a+b)^2+16a^2b^2\cr
(a+b)^2[(a+b)^2-4ab]&=&16a^2b^2\cr
(a+b)^2(a-b)^2&=&16a^2b^2\cr
(a^2-b^2)^2&=&16a^2b^2
\end{eqnarray*}

ahonnan következik, hogy $a^2-b^2=4ab$ (mivel $a>b$). \*
Átrendezve az $a^2-4ab-b^2=0$ egyenlethez jutunk, melyet végigosztunk $b^2$-tel (mert $b\neq0$).\*
Az $(\frac{a}{b})^2-4(\frac{a}{b})-1=0$ egyenletet megoldva kapjuk, hogy $\frac{a}{b}=2\pm\sqrt{5}$

Az $a>b>0$ feltétel alapján csak az $\frac{a}{b}=2+\sqrt{5}$ lehetséges.

Az $\frac{a}{b}-2=\sqrt{5}\not\in \mathbb{Q}$ ellentmond annak a feltételnek, hogy $a,b \in Q$, azaz $\frac{a}{b}-2\in \mathbb{Q}$.

Tehát nem létezik a feltételeknek megfelelő két különböző pozitív racionális szám.


\medskip
\vonal

{\bf 3. feladat:} Egy osztály minden tanulója vagy úszik, vagy kosarazik, esetleg
mindkettőt csinálja.
Lehetséges-e, hogy az osztályban több a lány, mint a fiú a következő esetekben:

a) ha az úszóknak és a kosarasoknak is $60 \%$-a fiú?

b) ha az úszók $60 \%$-a és a kosarasok $75 \%$-a fiú ?


\ki{Katz Sándor }{Bonyhád}\medskip


{\bf Megoldás:} Az a kérdés, hogy lehetséges-e, hogy $d+e+f>a+b+c$.

\begin{center}
\psset{xunit=1.0cm,yunit=1.0cm,algebraic=true,dimen=middle,dotstyle=o,dotsize=3pt 0,linewidth=0.8pt,arrowsize=3pt 2,arrowinset=0.25}
\begin{pspicture*}(-2.2,-0.24)(4.18,5.16)
\psline(-2,5)(4,5)
\psline(4,5)(4,0)
\psline(4,0)(-2,0)
\psline(-2,0)(-2,5)
\pscircle(0.04,2.5){1.64}
\pscircle(1.96,2.5){1.64}
\rput[tl](-1.7,4.88){Fiúk}
\rput[tl](-1.7,0.52){Lányok}
\psline(-2,2.5)(4,2.5)
\rput[tl](-0.48,3.5){$a$}
\rput[tl](0.88,3.5){$b$}
\rput[tl](2.34,3.5){$c$}
\rput[tl](-0.54,2.3){$d$}
\rput[tl](0.84,2.3){$e$}
\rput[tl](2.3,2.3){$f$}
\rput[tl](-0.6,4.5){Úszik}
\rput[tl](1.76,4.5){Kosarazik}
\end{pspicture*}
\end{center}

a)
   
        $a+b=0,6(a+b+d+e)\enspace\enspace$ és $\enspace\enspace c+b=0,6(c+b+f+e)$
        
        $0,4a+0,4b=0,6d+0,6e\enspace\enspace$ és $\enspace\enspace0,4e+0,4b=0,6f+0,6e$.
        
        $d+e=\frac{2}{3} a+\frac{2}{3} b \enspace (1) \enspace $ és $ \enspace \enspace f+e=\frac{2}{3} c + \frac{2}{3} b \enspace \enspace (2)$
        
        Az (1) és (2) alapján: $d+2e+f=\frac{2}{3} a +\frac{4}{3} b+\frac{2}{3} c$.
        
        Innen
        
        $$d+e+f=\frac23 a+\frac43 b+\frac23 c-e=a+b+c-\frac13 a+\frac13 b-\frac13 c-e$$
        
$$(d+e+f)-(a+b+c)=-\frac13 a+\frac13 b - \frac13 c-e$$
        
        A kért egyenlőtlenség akkor teljesül, ha $-\frac13 a+\frac13 b - \frac13 c-e >0$, azaz $b>a+c-3e$.
        
        Tehát úgy lehet a teljes létszám nagyobb része lány, hogy a fiúk (mindegyike vagy nagy része) mindkét sportot űzik, a lányok meg csak az egyiket. \\*
        Ha pl. $a=c=e=0$, akkor $2b=3d=3f$. Így, hogy ,,osztálynyian'' legyenek $b=15$ és $d=f=10$. (De nem szükséges, hogy $a=c=e=0$ legyen, pl. $a=c=2$, $e=1$, $b=10$ és $d=f=7$ esetén még mindig több a lány (15), mint a fiú (14) az osztályban.) \\*
        Összefoglalva, tehát ennél az aránynál {\bf lehetséges}, hogy több a lány, mint a fiú.
  
\medskip        
b)

$a+b=0,6(a+b+d+e)\enspace\enspace$ és $\enspace\enspace c+b=0,75(c+b+f+e)$
        
        $0,4a+0,4b=0,6d+0,6e\enspace\enspace$ és $\enspace\enspace 0,25c+0,25b=0,75f+0,75e$.
        
        $d+e=\frac{2}{3} a+\frac{2}{3} b \enspace (3) \enspace $ és $ \enspace \enspace f+e=\frac{1}{3} c + \frac{1}{3} b \enspace \enspace (4)$
        
        A (3) és (4) alapján: $d+2e+f=\frac{2}{3} a + b+\frac{1}{3} c$.
        
        Innen
        
        $$d+e+f=\frac23 a+ b+\frac 13 c-e=a+b+c-\frac13 a-\frac 23 c-e$$
        
$$(d+e+f)-(a+b+c)=-\frac13 a - \frac 23 c-e$$
        
        A kért egyenlőtlenség akkor teljesül, ha $-\frac13 a+ - \frac 23 c-e >0$, ami nyilván nem lehetséges.
        
        Összefoglalva, tehát ennél az aránynál már \textbf{nem lehetséges}, hogy több a lány, mint a fiú.
\medskip
\vonal

{\bf 4. feladat:} Legyen $ABCD$ egy olyan téglalap, amelybe szabályos háromszög írható
úgy, hogy a háromszög egyik csúcsa az $A$ pont, a másik kettő pedig a téglalap egy-egy olyan
oldalán fekszik, amelyen az $A$ pont nincs rajta. Bizonyítsuk be, hogy ekkor a téglalapból a
szabályos háromszög által lemetszett háromszögek egyikének a területe a két másik lemetszett
háromszög területének összegével egyenlő!

\ki{Pintér Ferenc }{Nagykanizsa}\medskip


{\bf I. megoldás:} Legyen $AB = a$ és $BC = b$, az $M$ pont pedig az $AFE$ szabályos háromszög $AF$
oldalának felezőpontja.

\begin{center}
\psset{xunit=1.0cm,yunit=1.0cm,algebraic=true,dimen=middle,dotstyle=o,dotsize=3pt 0,linewidth=0.8pt,arrowsize=3pt 2,arrowinset=0.25}
\begin{pspicture*}(-1.24,-0.24)(2.58,3.94)
\psline(-0.84,0.36)(2.16,0.36)
\psline(2.16,0.36)(2.16,3.49)
\psline(2.16,3.49)(-0.84,3.49)
\psline(-0.84,3.49)(-0.84,0.36)
\psline(-0.84,0.36)(2.16,1.42)
\psline(2.16,1.42)(-0.26,3.49)
\psline(-0.26,3.49)(-0.84,0.36)
\psline(-0.84,3.49)(0.66,0.89)
\psline(0.66,0.89)(2.16,3.49)
\psline[linestyle=dashed,dash=1pt 1pt](-0.26,3.49)(0.66,0.89)
\psline(0.66,0.89)(0.66,0.36)
\begin{scriptsize}
\psdots[dotstyle=*](-0.84,0.36)
\rput[bl](-0.94,-0.06){$A$}
\psdots[dotstyle=*](2.16,1.42)
\rput[bl](2.24,1.54){$F$}
\psdots[dotstyle=*](-0.26,3.49)
\rput[bl](-0.18,3.6){$E$}
\psdots[dotstyle=*](-0.84,3.49)
\rput[bl](-0.76,3.6){$B$}
\psdots[dotstyle=*](2.16,3.49)
\rput[bl](2.24,3.6){$C$}
\psdots[dotstyle=*](2.16,0.36)
\rput[bl](2.1,-0.12){$D$}
\psdots[dotstyle=*](0.66,0.89)
\rput[bl](0.84,0.64){$M$}
\psdots[dotstyle=*](0.66,0.36)
\rput[bl](0.58,-0.06){$K$}
\end{scriptsize}
\end{pspicture*}
\end{center}

Mivel $ABE\sphericalangle  = 90^\circ  = AME \sphericalangle$, 
ezért a $B$ és $M$ pontok rajta vannak az $AE$ szakasz Thalész-körén. De $MAE \sphericalangle  = 60 ^\circ$, ezért a kerületi szögek tétele miatt az $EBM\sphericalangle  = 60^\circ$.
Mivel $ECF\sphericalangle  = 90^\circ = FME\sphericalangle$, 
ezért a $C$ és $M$ pontok rajta vannak az $FE$ szakasz Thalész-körén. De $MFE \sphericalangle  = 60^\circ$, ezért a kerületi szögek tétele miatt az $ECM\sphericalangle  = 60^\circ$.
Az $EBM\sphericalangle  = 60^\circ = ECM\sphericalangle$, tehát a $BMC$ háromszög szabályos.

Ez azt jelenti, hogy $M$ pont a $BC$ oldaltól $\frac{b\sqrt 3}{2}$, míg
az $AD$ oldaltól $a-\frac{b\sqrt 3}{2}$ távolságra van. Az $AFD$ háromszögben $MK$ középvonal, így $DF=2\left(a- \frac{b\sqrt 3}{2}\right)=2a-b\sqrt 3$. 

Hasonló gondolatmenettel vagy az $ABE$ és az $AFD$ háromszögekben felírt Pitagorasz-tétel
segítségével belátható, hogy $BE = 2b-a\sqrt 3$. Következésképpen:

$\displaystyle{T_{ABE\Delta}=\frac{1}{2}a\left(2b-a\sqrt 3\right)=ab-\frac{a^2\sqrt 3}{2}}$

$\displaystyle{T_{ADF\Delta}=\frac{1}{2}b\left(2a-b\sqrt 3\right)=ab-\frac{b^2\sqrt 3}{2}}$

$\displaystyle{
T_{ECF\Delta}=
\frac{1}{2}\left(a\sqrt 3-b\right)\left(b\sqrt 3-a\right)
=2ab-\frac{a^2\sqrt 3}{2}-\frac{b^2\sqrt 3}{2}
=T_{ABE\Delta}+T_{ADF\Delta}}$, 

amit igazolni kellett.
\medskip


{\bf II. megoldás:} Legyen $BAE \sphericalangle  = \alpha$. Felhasználva, hogy az $AFE$ szabályos háromszög
($AE = EF = AF = x$) következik, hogy $FAD \sphericalangle  = 30 ^\circ-\alpha$, illetve $CEF \sphericalangle  = 30 ^\circ+\alpha$.

\begin{center}
\psset{xunit=1.0cm,yunit=1.0cm,algebraic=true,dimen=middle,dotstyle=o,dotsize=3pt 0,linewidth=0.8pt,arrowsize=3pt 2,arrowinset=0.25}
\begin{pspicture*}(-1.24,-0.24)(2.58,3.94)
\psline(-0.84,0.36)(2.16,0.36)
\psline(2.16,0.36)(2.16,3.49)
\psline(2.16,3.49)(-0.84,3.49)
\psline(-0.84,3.49)(-0.84,0.36)
\psline(-0.84,0.36)(2.16,1.42)
\psline(2.16,1.42)(-0.26,3.49)
\psline(-0.26,3.49)(-0.84,0.36)
\psline(-0.84,3.49)(0.66,0.89)
\psline(0.66,0.89)(2.16,3.49)
\psline[linestyle=dashed,dash=1pt 1pt](-0.26,3.49)(0.66,0.89)
%\psline(0.66,0.89)(0.66,0.36)
\begin{scriptsize}
\psdots[dotstyle=*](-0.84,0.36)
\rput[bl](-0.94,-0.06){$A$}
\psdots[dotstyle=*](2.16,1.42)
\rput[bl](2.24,1.54){$F$}
\psdots[dotstyle=*](-0.26,3.49)
\rput[bl](-0.18,3.6){$E$}
\psdots[dotstyle=*](-0.84,3.49)
\rput[bl](-0.76,3.6){$B$}
\psdots[dotstyle=*](2.16,3.49)
\rput[bl](2.24,3.6){$C$}
\psdots[dotstyle=*](2.16,0.36)
\rput[bl](2.1,-0.12){$D$}
\psdots[dotstyle=*](0.66,0.89)
\rput[bl](0.84,0.64){$M$}
%\psdots[dotstyle=*](0.66,0.36)
%\rput[bl](0.58,-0.06){$K$}
\end{scriptsize}
\end{pspicture*}
\end{center}

Az $ABE$ derékszögű háromszögben
$AB = x\cos(\alpha)$ és
$BE = x\sin(\alpha)$, tehát
$T_{ABE\Delta} = \frac{1}{2}x^2\sin(\alpha)\cos(\alpha) = 
\frac{1}{4}x^2\sin(2\alpha)$ 
(ahol felhasználtuk, hogy $\sin( \alpha  ) \cos( \alpha  ) =
\frac{\sin(2\alpha)}{2}$).

Az $ADF$ derékszögű háromszögben 
$AD = x\cos( 30 ^\circ  - \alpha  )$ 
és 
$FD = x \sin( 30 ^\circ  - \alpha  )$, tehát
$T_{ADF\Delta} = \frac 12 x^2 \sin( 30 ^\circ  - \alpha)\cos( 30 ^\circ-\alpha) = \frac 14 x^2 \sin( 60 ^\circ  - 2 \alpha)$.

Az $ECF$ derékszögű háromszögben $CE = x \cos( 30 ^\circ+\alpha)$ és 
$FC = x\sin( 30 ^\circ+\alpha)$, tehát
$T_{ECF\Delta}= \frac 12 x^2 \sin( 30 ^\circ  + \alpha  ) \cos( 30 ^\circ  + \alpha  ) = \frac 14 x^2 \sin( 60 ^\circ  + 2 \alpha  )$.

Felhasználva, hogy $\sin( p + q ) - \sin( p - q ) = 
2 \cos( p ) \sin( q )$, következik, hogy

$$T_{ECF\Delta}-T_{ADF\Delta} =\frac{1}{4}
x^2 \sin( 60 ^\circ  + 2 \alpha  ) -\frac{1}{4} x^2 \sin( 60 ^\circ  - 2 \alpha  ) = 
\frac 14 x^2\left[ \sin( 60 ^\circ  + 2 \alpha  ) - \sin( 60 ^\circ  - 2 \alpha  )\right] =$$
$$= \frac 14 x^2 2 \cos( 60 ^\circ  ) \sin( 2 \alpha  ) 
= \frac 14 x^2 2 \frac{1}{2} \sin( 2 \alpha  ) 
= \frac 14 x^2 \sin( 2 \alpha  ) = T_{ABE\Delta},$$

azaz $T_{ECF\Delta} = T_{ABE\Delta}+T_{ADF\Delta}$, amit igazolni kellett.

\medskip

\vonal

{\bf 5. feladat:} Bizonyítsuk be, hogy a $3^k+3^n$ alakban felírt négyzetszámokból végtelen
sok van, ahol $k$ és $n$ különböző pozitív egész számok! Mi a helyzet, ha a 3 helyett a 4, az 5, a
6 és a 7 számokat írjuk?
 
\ki{Kántor Sándor }{Debrecen}\medskip

{\bf Megoldás:} Azt kell megvizsgálnunk, hogy 
az $a^m+a^k$ (ahol $m \ne k$ és $a, m, k \in \mathbb{N}^{+}$) alakú
számok a kért esetekben mikor lesznek négyzetszámok. Ha ugyanis 
egy $a^m+a^k$ alakú szám
négyzetszám, akkor végtelen sok ilyen alakú négyzetszám van, mert minden $n$ pozitív egész
szám esetén 
$a^{m+2n}+a^{k+2n} = a^m \cdot a^{2n}+a^k \cdot a ^{2n} 
= a^{2n}\left(a^m+a^k\right) = 
\left(a^n\right)^2\left(a^m+a^k\right)$.

\medskip
Az $a = 3$ eset.

Mivel $3^3+3^2 = 27 + 9 = 36 = 6^2$, 
ezért végtelen sok $3^m+3^k$ alakú négyzetszám van.

\medskip\newpage
Az $a = 4$ eset.

Az általánosság leszűkítése nélkül
feltételezhetjük, hogy $m>k$.

Ekkor $4^m+4^k = 4^k\left(4^{m-k}+1\right) = \left(2^k\right)^2\left(4^{m-k}+1\right)$, azaz azt kell megvizsgálnunk, mikor lesz a 
$4^{m-k}+1$ teljes négyzet. 
Ha $4^{m-k}+1=c^2$, akkor $c^2-4^{m-k} = c^2 -\left(2^{m-k}\right)^2
= \left(c-2^{m-k}\right)\left(c+2^{m-k}\right) = 1$.

Felhasználva, hogy a $c$ és a $2^{m-k}$ egész számok, 
innen a $2^{m-k} = 0$ következne, ami nem
lehetséges. Tehát a 4 két különböző pozitív egész kitevős hatványának összege nem lehet
négyzetszám.

\medskip
Az $a = 5$ eset.

Az $5^n$ szám minden $n > 1$ esetben 25-re végződik, míg $5^1 = 5$. Ezek alapján az $5^m+5^k$ szám
vagy 30-ra vagy 50-re végződik. Viszont ha egy négyzetszám 10-el osztható, akkor 100-al is
osztható, így nem végződhet sem 30-ra, sem 50-re. Tehát az 5 két különböző pozitív egész
kitevős hatványának összege nem lehet négyzetszám.

\medskip
Az $a = 6$ eset.

A 6-nak bármely pozitív egész hatványa 6-ra végződik. Így a 
$6^m+6^k$ szám utolsó számjegye
2 lesz, de négyzetszám 2-re nem végződhet. Tehát a 6 két különböző pozitív egész kitevős
hatványának összege nem lehet négyzetszám.

\medskip
Az $a = 7$ eset.

A 7-nek 3-al való osztási maradéka 1, ezért a 7 minden pozitív egész kitevős hatványának a
3-mal való osztási maradéka szintén 1 
(hiszen a $7^n = \left( 2 \cdot  3 + 1\right)^n = 3 M + 1$ alakú lesz). Így a $7^m + 7^k$ szám 3-al osztva 2-t ad maradékul.

Ha $c$ egy természetes szám és 3-nak többszöröse, akkor $c^2$ is 3 többszöröse lesz, míg ha
$c = 3 n \pm 1$ alakú, akkor a négyzetének 3-al való osztási maradéka 1 lesz. Összegezve, egy
négyzetszám hárommal osztva csak 0 vagy 1 maradékot adhat.

Tehát a 7 két különböző pozitív egész kitevős hatványának összege nem lehet négyzetszám.

\medskip
\vonal


{\bf 6. feladat:} Adott háromszögbe szerkesztettünk két egybevágó, közös belső pont
nélküli, maximális sugarú kört. Mekkora ez a sugár? Hogyan történhet a szerkesztés?


\ki{Bogdán Zoltán }{Cegléd}\medskip


{\bf Megoldás:} Legyen $ABC$ az adott háromszög, melynek oldalai $a$, $b$, $c$. Nyilván mindkét kör
érinti például az $a$ oldalt és az egyik a $b$-t, a másik a $c$-t, valamint egymást is érintik. Az egyik
kör középpontja a $B$ csúcsból, a másiké a $C$ csúcsból kiinduló belső szögfelezőn lesz. A körök
középpontjai legyenek $E$ és $F$, míg a szögfelezők metszéspontja (a háromszögbe írható kör
középpontja) pedig $O$.

\centerline{
\psset{xunit=1.0cm,yunit=1.0cm,algebraic=true,dimen=middle,dotstyle=o,dotsize=3pt 0,linewidth=0.8pt,arrowsize=3pt 2,arrowinset=0.25}
\begin{pspicture*}(-0.54,0.5)(7.86,5.2)
\psline(2.84,4.64)(7.26,1.26)
\psline(7.26,1.26)(0.18,1.26)
\psline(0.18,1.26)(2.84,4.64)
\psline(0.18,1.26)(4.51,3.36)
\psline(7.26,1.26)(1.67,3.15)
\psline(2.26,2.27)(4.28,2.27)
\psline(4.28,2.27)(4.28,1.26)
\psline(2.26,2.27)(2.26,1.26)
\pscircle(2.26,2.27){1.01}
\pscircle(4.28,2.27){1.01}
\psline(3.09,2.67)(3.09,1.26)
\begin{scriptsize}
\psdots[dotstyle=*](2.84,4.64)
\rput[bl](2.92,4.76){$A$}
\psdots[dotstyle=*](7.26,1.26)
\rput[bl](7.24,0.8){$B$}
\psdots[dotstyle=*](0.18,1.26)
\rput[bl](0.14,0.86){$C$}
\psdots[dotstyle=*](3.09,2.67)
\rput[bl](3.16,2.8){$O$}
\psdots[dotstyle=*](2.26,2.27)
\rput[bl](2.34,2.38){$E$}
\psdots[dotstyle=*](2.26,1.26)
\rput[bl](2.16,0.86){$M$}
\psdots[dotstyle=*](4.28,2.27)
\rput[bl](4.36,2.38){$F$}
\psdots[dotstyle=*](4.28,1.26)
\rput[bl](4.22,0.84){$N$}
\psdots[dotstyle=*](3.09,1.26)
\rput[bl](3.02,0.82){$D$}
\end{scriptsize}
\end{pspicture*}
}

Legyen a háromszögbe írható kör sugara $r$, a keresett sugár pedig $x$. Ekkor az 
$EM = FN = x$,
illetve az $EMN\sphericalangle = FNM \sphericalangle = 90^\circ$ összefüggések alapján következik, hogy az $EFNM$
négyszög egy téglalap. Mivel $EF \parallel BC$, könnyen belátható a $BCO$ és $FEO$ háromszögek
hasonlósága. Figyelembe véve, hogy $OD = r$, a két háromszög hasonlóságából felírható,
hogy $\frac{r-x}{r}=\frac{2x}{a}$, ahonnan az $x=\frac{ar}{a+2r}$ lesz.

Ha $x$ kifejezésében $a$-val egyszerűsítünk, azaz $x =\frac{r}{1+\frac{2r}{a}}$ 
lesz, akkor az látszik, hogy a
nevező akkor a legkisebb, ha a mindkét kört érintő $a$ oldal a három oldal közül a legnagyobb.
Ekkor lesz az $x$ sugár az adott háromszögben maximális.

Az $x$ egy lehetséges megszerkesztése:
\begin{itemize}
\item adott $a$, $b$, $c$ hosszúságú szakaszokkal megszerkesztjük az $ABC$ háromszöget
\item  megszerkesztjük a háromszög két belső szögfelezőjét, így azok metszéspontjából
megkapjuk a háromszögbe írható kör középpontját, illetve sugarát
\item  egy szög egyik szárára felmérjük az $a+2r$ és az $a$ hosszúságú szakaszokat, a másik
szárára pedig egy $r$ hosszúságút, majd párhuzamos szelők segítségével megkapjuk az
$x$ hosszúságú szakaszt
\item  a $BC$-vel párhuzamost húzunk $x$ távolságra, a párhuzamos és a belső szögfelezők
metszéspontjai megadják a keresett középpontokat

\end{itemize}



\end{document}