\documentclass[a4paper,10pt]{article} 
\usepackage[utf8]{inputenc}
\usepackage[a4paper]{geometry}
\usepackage[magyar]{babel}
\usepackage{amsmath}
\usepackage{amssymb}
\frenchspacing 
\pagestyle{empty}
\newcommand{\ki}[2]{\hfill {\it #1 (#2)}\medskip}
\newcommand{\vonal}{\hbox to \hsize{\hskip2truecm\hrulefill\hskip2truecm}}
\newcommand{\degre}{\ensuremath{^\circ}}
\newcommand{\tg}{\mathop{\mathrm{tg}}\nolimits}
\newcommand{\ctg}{\mathop{\mathrm{ctg}}\nolimits}
\newcommand{\arc}{\mathop{\mathrm{arc}}\nolimits}
\begin{document}
\begin{center} \Large {\em 25. Nemzetközi Magyar Matematika Verseny} \end{center}
\begin{center} \large{\em Budapest, 2016. március 11-15.} \end{center}
\smallskip
\begin{center} \large{\bf 10. osztály} \end{center}
\bigskip 

{\bf 1. feladat: } Egy diák megírt már néhány dolgozatot, és az utolsó megírása előtt számolgat: Ha az utolsót $97$ pontosra írom, akkor az átlagom $90$ pont lesz, ha csak $73$ pontra sikerül, akkor $87$ pont lesz az átlagom. Hány dolgozatot írt eddig a diák, és mennyi volt az átlagpontszáma?

\ki{Katz Sándor}{Bonyhád}\medskip

{\bf 2. feladat: } Hányféleképpen lehet sorrendbe állítani a RENDETLENÜL szó betűit úgy, hogy ne álljon két E betű egymás mellett?  (Minden betűt pontosan egyszer használunk fel.)

\ki{Bálint Béla}{Zsolna}\medskip

{\bf 3. feladat: } Adott a síkban két egymásra merőleges egyenes, $f$ és $g$, valamint a $g$ egyenesen két pont, $A$ és $B$, amelyek egymástól is és a két egyenes metszéspontjától is különböznek. Az $f$ egyenes egy tetszőleges $P$ pontját az adott pontokkal összekötő egyenesekre merőlegeseket állítunk az $A$ és $B$ pontokban. Határozza meg a merőlegesek metszéspontjainak a halmazát, ha $P$ végigfut az $f$ egyenesen.

\ki{Kántor Sándorné}{Debrecen}\medskip

{\bf 4. feladat: } Legyen az $AB$ átmérőjű $k_1$ kör egy $A$-tól és $B$-től különböző pontja $C$. \mbox{Bocsássunk} merőlegest a $C$ pontból $AB$-re, a merőleges talppontja $T$. A $C$ középpontú, $CT$ sugarú $k_2$ kör a $k_1$ kört a $D$ és $E$ pontokban metszi. A $DE$ és $CT$ szakaszok metszéspontja $M$, a $CA$ és $DE$, valamint a $CB$ és $DE$ szakaszok metszéspontjai rendre $X$ és $Y$. Bizonyítsa be, hogy $MX=MY$.

\ki{Bíró Bálint}{Eger} \medskip

{\bf 5. feladat: } Bizonyítsa be, hogy  $n+1$ darab különböző, $2n$-nél kisebb  pozitív egész szám közül kiválasztható  három különböző úgy, hogy ezek közül kettő összege megegyezzen a harmadikkal.

\ki{Bencze Mihály}{Bukarest}\medskip

{\bf 6. feladat: } Képezzük az $\{1, 2, 3, \ldots, 2016\}$ halmaz minden nemüres részhalmazát. Az egy részhalmazban lévő számokat  szorozzuk össze  és vegyük a szorzat reciprokát, majd ezeket adjuk össze. (Ha a halmaz egyelemű, akkor egytényezős szorzatnak tekintjük és ennek vesszük a reciprokát.) Mekkora az így kapott összeg?

\ki{Kántor Sándor}{Debrecen}\medskip


\end{document}