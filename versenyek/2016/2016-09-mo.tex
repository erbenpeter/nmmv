\documentclass[a4paper,10pt]{article} 
\usepackage[utf8]{inputenc}
\usepackage[a4paper]{geometry}
\usepackage[magyar]{babel}
\usepackage{amsmath}
\usepackage{amssymb}
\usepackage{graphics}
\frenchspacing 
\pagestyle{empty}
\newcommand{\ki}[2]{\hfill {\it #1 (#2)}\medskip}
\newcommand{\vonal}{\hbox to \hsize{\hskip2truecm\hrulefill\hskip2truecm}}
\newcommand{\degre}{\ensuremath{^\circ}}
\newcommand{\tg}{\mathop{\mathrm{tg}}\nolimits}
\newcommand{\ctg}{\mathop{\mathrm{ctg}}\nolimits}
\newcommand{\arc}{\mathop{\mathrm{arc}}\nolimits}
\begin{document}
\begin{center} \Large {\em 25. Nemzetközi Magyar Matematika Verseny} \end{center}
\begin{center} \large{\em Budapest, 2016. március 11-15.} \end{center}
\smallskip
\begin{center} \large{\bf 9. osztály} \end{center}
\bigskip 

{\bf 1. feladat: } Nevezzünk egy számot prímösszegűnek, ha a tízes számrendszerben felírt szám számjegyeinek összege prímszám. Legfeljebb hány prímösszegű szám lehet öt egymást követő pozitív egész szám között?

\ki{Róka Sándor}{Nyíregyháza}\medskip

{\bf 1. megoldás: } Az öt szám között van három olyan, amely ugyanabba a tízes blokkba tartozik. 

\smallskip

\noindent Ez a három szám egymástól csak az utolsó számjegyben különbözik, a számjegyösszegeiket kiszámolva tehát három egymást követő számot kapunk.


\smallskip

\noindent Három egymást követő szám nem lehet mind prímszám, tehát az ugyanabba a tízes blokkba tartozó három szám nem lehet mind prím. Ezért az öt egymást követő egész szám között legfeljebb négy prímösszegű szám lehet.


\smallskip

\noindent Megmutatjuk, hogy létezik öt olyan egymást követő egész szám, amelyek között négy prímösszegű szám van. Ilyenek például: $199, 200, 201, 202, 203$, ahol a számjegyösszegek rendre $19, 2, 3, 4, 5$; vagy $197, 198, 199, 200, 201$, ahol a számjegyösszegek rendre $17, 18, 19, 2, 3$.

\medskip


{\bf 2. megoldás: } Egy szám számjegyösszege $3$-mal osztva ugyanannyi maradékot ad, mint maga a szám. 
Öt egymást követő szám számjegyösszegei között tehát mindig van $3$-mal osztható.

\smallskip

\noindent Ha a 3-mal osztható számjegyösszeg nem prím, akkor legfeljebb négy prímösszegű számunk lehet.
Mind az öt összeg csak úgy lehet prím, ha a $3$-mal osztható számjegyösszeg a $3$.

\smallskip

\noindent Egy $n$ szám számjegyösszege háromféleképpen lehet $3$: egy $3$-as után valahány $0$; egy $2$-es, egy $1$-es és valahány $0$; vagy három $1$-es és valahány $0$.

\smallskip

\noindent Az $n$ végződése tehát $0, 1, 2$ vagy ($n = 3$ esetén) $3$. 

\noindent Ezért $n+1$ számjegyösszege mindegyik esetben $4$ (hiszen nem történik tízes átlépés), vagyis nem prím. Mind az öt összeg csak úgy lehet prím, ha $n$ az ötödik helyen álló szám.

\smallskip

\noindent Az ötödik helyen álló $n$ szám nem lehet a $3$, mert akkor a számok között negatív számok is lennének.

\noindent Minden más esetben $n$ legalább kétjegyű. A hárommal osztható $n-3$ szám végződése $7$, $8$ vagy $9$, számjegyeinek összege nem lehet $3$, ezért nem prím.\hfill (2~pont)%

\smallskip

\noindent Megmutatjuk, hogy létezik öt olyan egymást követő egész szám, amelyek között négy prímösszegű szám van.
Ilyenek például: $199$, $200$, $201$, $202$, $203$, ahol a számjegyösszegek rendre $19$, $2$, $3$, $4$, $5$; vagy $197$, $198$, $199$, $200$, $201$, ahol a számjegyösszegek rendre $17$, $18$, $19$, $2$, $3$.\hfill (2~pont)%

\medskip

{\bf 3. megoldás: } Két egymást követő szám számjegyösszege tízes átlépéskor lehet azonos paritású (pl. $9$, $10$ vagy $19$, $20$), máskor mindig különböző, mert csak az utolsó számjegy különbözik $1$-gyel.
Öt egymást követő szám számjegyösszegei között ezért mindig van legalább két páros.

\smallskip

\noindent Ez a két páros szám csak úgy lehetne egyaránt prím, ha mindegyikük $2$.
Egy $n$ szám jegyeinek összege kétféleképpen lehet $2$: 
egy $2$-es után valahány $0$; vagy két $1$-es és valahány $0$.

\smallskip

\noindent A szám végződése tehát $0, 1$ vagy ($n = 2$ esetén) $2$.

\noindent Ha $n = 2$, akkor az $n + 2 = 4$ is a számok között van, amely nem prím.

\noindent Minden más esetben az $n+1$-től $n+4$-ig terjedő számok számjegyösszege nagyobb, mint az $n$ számjegyösszege. Az $n-4$-től $n-1$-ig terjedő számoknak vagy $1$ a számjegyösszege, vagy pedig jegyeik között van $6$, $7$, $8$ vagy $9$, egyik esetben sem lehet az összeg $2$. 

\noindent Nincs tehát $n-4$-től $n + 4$-ig más olyan szám, amelyben a számjegyösszeg 2.

\noindent Ezért az öt egymást követő egész szám között legfeljebb négy prímösszegű szám lehet.

\smallskip

\noindent Megmutatjuk, hogy létezik öt olyan egymást követő egész szám, amelyek között négy prímösszegű szám van.
Ilyenek például: $199$, $200$, $201$, $202$, $203$, ahol a számjegyösszegek rendre $19$, $2$, $3$, $4$, $5$; vagy $197$, $198$, $199$, $200$, $201$, ahol a számjegyösszegek rendre $17$, $18$, $19$, $2$, $3$.

\vonal

{\bf 2. feladat: } Melyek azok az $x$ egész számok, amelyekre $x^2+3x+24$ négyzetszám?

\ki{Szabó Magda}{Zenta--Szabadka}\medskip

{\bf 1. megoldás: } Keressük az $x^2+3x+24=y^2$ egyenlet egész megoldásait. Ezt 4-gyel szorozva:  $4x^2+12x+96=4y^2$. \hfill 

\smallskip

\noindent Átrendezve:
\begin{align*}
(2x+3)^2+87&=4y^2\\(2x+3)^2-4y^2&=-87.
\end{align*} 
Szorzattá alakítva:
\[(2x+3-2y)(2x+3+2y)=-87.\]

\noindent A $87$ osztói $\pm1$, $\pm3$, $\pm29$, $\pm87$, a két tényező ezek közül kerülhet ki.
Feltehetjük, hogy $y$ pozitív, ekkor a második tényező a nagyobb.
Megoldandók tehát a következő egyenletrendszerek:
\[
\left\{\begin{aligned}2x+3-2y&=-1\\2x+3+2y&=87,\end{aligned}\right.\qquad
\left\{\begin{aligned}2x+3-2y&=-3\\2x+3+2y&=29,\end{aligned}\right.\]
\[\left\{\begin{aligned}2x+3-2y&=-29\\2x+3+2y&=3,\end{aligned}\right.\qquad
\left\{\begin{aligned}2x+3-2y&=-87\\2x+3+2y&=1.\end{aligned}\right.
\]


\noindent Ezek megoldása rendre:
\[\begin{aligned}x&=20\\(y&=22),\end{aligned}\qquad
\begin{aligned}x&=5\\(y&=8),\end{aligned}\qquad
\begin{aligned}x&=-8\\(y&=8),\end{aligned}\qquad
\begin{aligned}x&=-23\\(y&=22).\end{aligned}
\] 
A keresett $x$ egész számok $-23$, $-8$, $5$, $20$. Ezek valóban megoldásai az egyenletnek.

\medskip

{\bf 2. megoldás } Keressük az $x^2+3x+24=y^2$ egyenlet egész megoldásait. Ezt $4$-gyel szorozva:  $4x^2+12x+96=4y^2$. 

\smallskip

\noindent Átrendezve:
\begin{align*}
(2x+3)^2+87&=4y^2\\4y^2-(2x+3)^2&=87.
\end{align*} 

\noindent Két négyzetszám különbsége $87$. Tudjuk, hogy az egymást követő négyzetszámok különbségei a páratlan
számok, $87$ tehát valahány egymást követő páratlan szám összege.



\smallskip

\noindent A $87$ páratlan, tehát páratlan darab páratlan számot keresünk, ezek összege osztható a tagok számával. A $87$ osztói $1, 3, 29, 87$. A tagok száma lehet $1$ és lehet $3$, de $29$ vagy $87$ nem lehet. 

\smallskip

\noindent Egy tag esetén: $87=44^2-43^2$, ekkor $2x+3=43$, azaz $x=20$; vagy $2x+3=-43$, azaz $x=-23$.

\noindent Három tag esetén: $87=27+29+31=16^2-13^2$, ekkor $2x+3=13$, azaz $x=5$; vagy $2x+3=-13$, azaz $x=-8$.

\noindent A keresett $x$ egész számok $-23$, $-8$, $5$, $20$. Ezek valóban megoldásai az egyenletnek. 

\medskip

{\bf 3. megoldás } Keressük az $x^2+3x+24$ négyzetszámot $(x+k)^2=x^2+2kx+k^2$ alakban.
Ekkor $3x+24=2kx+k^2$.
Átrendezve: $x(2k-3)=24-k^2$, vagyis $2k-3 \mid k^2-24$. 

\smallskip

\noindent Ekkor $2k-3\mid 2k^2-48$ következik, és $2k-3\mid 2k^2-3k$ is nyilván igaz.
Ezekből $2k-3 \mid $ $\mid 48- 3k$, ezért $2k-3 \mid 96 -6k$.
Ugyanakkor $2k-3\mid 9 -6k$,
tehát $2k-3 \mid 96-9 =87$.

\smallskip

\noindent A $87$ osztói $+1$, $-1$, $+3$, $-3$, $+29$, $-29$, $+87$, $-87$, ezek közül kerülhet ki $2k-3$.
Ekkor $k$ értéke rendre $2$, $1$, $3$, $0$, $16$, $-13$, $45$, $-42$, innen $x$ értéke rendre $20$, $-23$, $5$, $-8$, $-8$, $5$, $-23$, $20$. 

\noindent A keresett $x$ egész számok $-23$, $-8$, $5$, $20$. Ezek valóban megoldásai az egyenletnek. 

\medskip

{\bf 4. megoldás: } Keressük az $x^2+3x+24=y^2$ egyenlet egész megoldásait.
Átrendezve: $x^2+3x+(24-y^2)=0$. A másodfokú egyenlet megoldóképlete alapján
\[x=\frac{-3\pm\sqrt{9-4(24-y^2)}}2=\frac{-3\pm\sqrt{4y^2-87}}2.\]
Így $x$ csak akkor lehet egész, ha $4y^2-87$ négyzetszám: $4y^2-87=n^2$.

\smallskip

\noindent Átrendezve: 
\begin{align}
4y^2-n^2&=87\notag\\(2y-n)(2y+n)&=87.
\end{align}

\noindent Feltehetjük, hogy $y$ és $n$ pozitív, ekkor a második tényező a nagyobb, és pozitív. A $87$ pozitív osztói: $1,3,29,87$. Megoldandók tehát a következő egyenletrendszerek:
\[\left\{\begin{aligned}2y-n&=1\\2y+n&=87,\end{aligned}\right.\qquad
\left\{\begin{aligned}2y-n&=3\\2y+n&=29.\end{aligned}\right.\]

\noindent Ezek megoldásai rendre: $y=22$, ekkor $x=-23$ vagy $x=20$; illetve $y=8$, ekkor $x=-8$ vagy $x=5$. 

\noindent A keresett $x$ egész számok $-23$, $-8$, $5$, $20$. Ezek valóban megoldásai az egyenletnek. 

\vonal

{\bf 3. feladat: } Bizonyítsa be, hogy az 
\[(n^2+7n)(n^2+7n+6)(n^2+7n+10)(n^2+7n+12)\] 
kifejezés minden egész $n$ esetén osztható $2016$-tal.


\ki{Nagy Piroska Mária}{Dunakeszi}

\ki{Szoldatics József}{Budapest}\medskip

{\bf 1. megoldás: } Alakítsuk szorzattá a tényezőket: 
\begin{align*}
n^2+7n&=n(n+7)\\
n^2+7n+6&=n^2+n+6n+6=(n+1)(n+6)\\
n^2+7n+10&=n^2+2n+5n+10=(n+2)(n+5)\\
n^2+7n+12&=n^2+3n+4n+12=(n+3)(n+4).
\end{align*}

\noindent A kérdéses kifejezés tehát $n(n+1)(n+2)(n+3)(n+4)(n+5)(n+6)(n+7)$ alakban írható. Ez $8$ egymást követő egész szám szorzata. 

\smallskip

\noindent Belátjuk, hogy a szorzat osztható $2^5\cdot 3^2\cdot7=2016$-tal.

\smallskip

\noindent A $8$ egymást követő szám között biztosan van $7$-tel osztható, így a szorzat osztható $7$-tel. 



\smallskip

\noindent Van közöttük legalább $2$ darab $3$-mal osztható, így a szorzat osztható $3^2$-nel. 


\smallskip

\noindent Van $4$ darab páros szám és van $4$-gyel is osztható, ezzel megvan a hiányzó $2$-es prímtényező, így a szorzat osztható $2^5$-nel. 

\smallskip

\noindent A kifejezés tehát osztható $2016$-tal.

\medskip

{\bf 2. megoldás: } Belátjuk, hogy a szorzat osztható $2016 = 2^5\cdot3^2\cdot7$-tel.

\smallskip

\noindent Vizsgáljuk sorban a $2^5$, $3^2$, illetve $7$ tényezőkkel való oszthatóságot.

\noindent Mivel $n^2$ és $7n$ azonos paritású, mind a négy tényező páros.
Az $n^2+7n$ és $n^2+7n+6$ különböző maradékot adnak $4$-gyel osztva, egyikük tehát $4$-gyel osztható, ezzel megvan a hiányzó $2$-es prímtényező, így a szorzat osztható $2^5$-nel. 

\smallskip

\noindent Ha $n=3k$ alakú, akkor $n^2+7n$ és $n^2+7n+6$ egyaránt osztható $3$-mal, így a szorzat osztható $3^2$-nel.

\noindent Ha  $n=3k-1$ alakú, akkor $n^2$-nek a $3$-as maradéka $1$, $7n$-nek a $3$-as maradéka $-1$. 
Az $n^2+7n$  és $n^2+7n+6$ tehát ilyenkor is osztható $3$-mal, így a szorzat osztható $3^2$-nel.

\noindent Ha $n=3k+1$ alakú, akkor $n^2+7n+10=9k^2+6k+1+21k+7+10\MathBrk{=}9k^2+27k+18=$ $=9(k^2+3k+2)$,
ez $9$-cel osztható, így a szorzat is osztható $3^2$-nel. 

\smallskip

\noindent A $7$-tel való oszthatóság szempontjából elég az $n^2$, $n^2+6$, $n^2+10$, $n^2+12$ számokat vizsgálni.

\noindent Ha  $n=7k$ alakú, akkor $n^2$ osztható $7$-tel.

\noindent Ha $n=7k\pm1$  alakú, akkor $n^2$-nek a 7-es maradéka $1$, tehát $n^2+6$ osztható $7$-tel.

\noindent Ha  $n=7k\pm2$ alakú, akkor $n^2$-nek a $7$-es maradéka $4$, tehát $n^2+10$ osztható $7$-tel.

\noindent Ha  $n=7k\pm3$ alakú, akkor $n^2$-nek a $7$-es maradéka $2$, tehát $n^2+12$ osztható $7$-tel. 

\smallskip

\noindent A kifejezés tehát osztható $2^5\cdot 3^2\cdot7 = 2016$-tal.


\vonal 

{\bf 4. feladat: } Egy egyenlő szárú háromszög alapon fekvő szögének felezője kétszer olyan hosszú, mint a szárak szögének felezője. Mekkorák a háromszög szögei?

\ki{Katz Sándor}{Bonyhád}\medskip

{\bf 1. megoldás: } Legyen a szóban forgó $ABC$ háromszögben $AC=BC$, valamint $AD$ és $CE$ a két szögfelező szakasz, a feltevés szerint $AD=2\cdot CE$. Legyen $\alpha =BAC\sphericalangle =CBA\sphericalangle$. 

\begin{center}
\includegraphics{9-4_1-eps-converted-to.pdf}
\end{center}

\noindent Állítsunk merőlegest az $AB$ alapra az $A$ pontban, jelölje $F$ ennek metszéspontját a $BC$ egyenessel. Mivel az $E$ pont felezi az $AB$ szakaszt, a párhuzamos szelők tétele miatt (vagy az $EBC$ és $ABF$ háromszögek hasonlósága folytán) $FA=2\cdot CE$. 


\smallskip

\noindent Az $FAD$ háromszög tehát egyenlő szárú, és ezért $AFD\sphericalangle =ADF\sphericalangle$. 

\smallskip

\noindent Ezek a szögek $\alpha$-val kifejezhetők az $ABF$ és az $ABD$ háromszög szögösszegét felhasználva:
\[AFD\sphericalangle =90^{\circ} -\alpha ,\quad\text{illetve}\quad ADF\sphericalangle =\frac{\alpha}{2}+\alpha\,.\]%

\noindent Az $\alpha$ szögre ezekből a $90^{\circ} -\alpha =(3/2)\alpha$ egyenletet kapjuk, ahonnan $\alpha =36^{\circ}$. A háromszög szögei tehát $36^{\circ}$, $36^{\circ}$ és $108^{\circ}$. 

\medskip

{\bf 2. megoldás: } Használjuk az 1.~megoldásban bevezetett $A$, $B$, $C$, $D$, $E$ és $\alpha$ jelöléseket. Tükrözzük a háromszög $C$ csúcsát az $AB$ alapra, a tükörképet jelölje $C'$. Ekkor $AC'\parallel CB$, így az $AC'DC$ négyszög trapéz. 

\begin{center}
%\includegraphics[width=.45\hsize]{9-4_2.eps}
\includegraphics{9-4_2-eps-converted-to.pdf}
\end{center}

\noindent A feltétel szerint $AD=2\cdot CE =CC'$, ezért az $AC'DC$ trapéz két átlója egyenlő, vagyis szimmetrikus trapézról van szó. 

\smallskip

\noindent Szimmetrikus trapézban az átlók egyenlő szögeket zárnak be az alapokkal, ezért $C'AD\sphericalangle =$ $=AC'C\sphericalangle$. 

\smallskip

\noindent Ezek a szögek $\alpha$-val kifejezve:
\[C'AD\sphericalangle = C'AB\sphericalangle + BAD\sphericalangle = \alpha +\frac{\alpha}{2}\quad\text{és}\quad AC'C\sphericalangle = ACC'\sphericalangle = 90^{\circ}-\alpha ,\]%
ahonnan $(3/2)\alpha =90^{\circ}-\alpha$, azaz $\alpha =36^{\circ}$. A háromszög szögei tehát $36^{\circ}$, $36^{\circ}$ és $108^{\circ}$. 

\vonal


{\bf 5. feladat: } Adva van a síkban $2016$ olyan pont, hogy minden ponthármasból kiválasztható két pont, amelyek $1$ egységnél kisebb távolságra vannak egymástól. Bizonyítsa be, hogy létezik olyan egységsugarú kör, amely a $2016$ pont közül legalább $1008$-at tartalmaz.

\ki{Bálint Béla}{Zsolna}\medskip

{\bf 1. megoldás: } Ha a pontok között nincs két olyan pont, amelyek távolsága legalább $1$, akkor készen vagyunk, a megadott pontok közül bármelyik körül rajzolt egységsugarú kör megfelel. 

\smallskip

\noindent Ha az $A$ és $B$ pontok távolsága legalább $1$, akkor tekintsük az $A$, illetve $B$ középpontú egységsugarú köröket. 

\smallskip

\noindent Ha egy $C$ pont egyik körben sincs benne, akkor az $ABC$ ponthármas ellentmond a feladat feltételeinek.

\smallskip

\noindent A $2016$ pontnak tehát mindegyike benne van a két kör közül legalább az egyikben.
Nem lehet mindkét körben $1008$-nál kevesebb pont.
(Legalább) az egyik kör tehát legalább 1008 pontot tartalmaz.

\smallskip

\noindent Létezik tehát olyan egységsugarú kör, amely a $2016$ pont közül legalább $1008$-at tartalmaz.

\medskip

{\bf 2. megoldás: } Legyen $A$ a $2016$ pont egyike. Tekintsük az $A$ körüli egységsugarú kört.
Ha ebben van legalább $1008$ pont, akkor készen vagyunk. 

\smallskip

\noindent Ha nincs, akkor legyen $B$ és $C$ két pont a körön kívül lévő legalább $1009$ pont közül. 

\smallskip

\noindent Mivel $AB$ és $AC$ nagyobb $1$-nél, a feladat feltétele szerint $BC < 1$.

\smallskip

\noindent Ugyanígy bármely két kívül lévő pont távolsága $1$-nél kisebb. 
A kívül lévő pontokat ezért tartalmazza például a $B$ körüli egységsugarú kör.

\smallskip

\noindent Létezik tehát olyan egységsugarú kör, amely a $2016$ pont közül legalább $1008$-at tartalmaz.

\vonal

{\bf 6. feladat: } Az $x_1, x_2, x_3, \ldots, x_n$ számok mindegyikének értéke $+1$ vagy $-1$. 
Bizonyítsa be, hogy ha  
\[x_1x_2x_3x_4+x_2x_3x_4x_5+x_3x_4x_5x_6+\ldots+x_{n-1}x_nx_1x_2+x_nx_1x_2x_3=0,\] 
akkor az $n$ szám $4$-gyel osztható.

\ki{Keke\v{n}ák Szilvia}{Kassa}\medskip

{\bf Megoldás: } Az összeg mindegyik tagja $+1$ vagy $-1$.

\smallskip

\noindent Nézzük meg, mi történik, ha az egyik szám, $x_i$ előjelét megváltoztatjuk.

\smallskip

\noindent Az $x_i$ szám az összeg négy tagjában szerepel, ezen tagok előjele fog megváltozni.

\noindent Ha mind a négy tag azonos előjelű, az összeg (valamelyik irányban) $8$-cal változik.

\noindent Ha három tag azonos előjelű és a negyedik különböző, akkor az összeg (valamelyik irányban) $6-2 = 4$-gyel változik.

\noindent Ha két-két tag pozitív, illetve negatív előjelű, az összeg nem változik.

\smallskip

\noindent Az egyes számok előjelének megváltoztatása tehát az összeg $4$-es maradékát nem befolyásolja.
Ha minden szám értékét $+1$-re változtatjuk, az összeg értéke $n$ lesz.
Mivel az összeg eredetileg $0$ volt, azaz $4$-gyel osztható, $n$ is osztható $4$-gyel.


\end{document}