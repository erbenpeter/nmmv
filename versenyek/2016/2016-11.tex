\documentclass[a4paper,10pt]{article} 
\usepackage[utf8]{inputenc}
\usepackage[a4paper]{geometry}
\usepackage[magyar]{babel}
\usepackage{amsmath}
\usepackage{amssymb}
\frenchspacing 
\pagestyle{empty}
\newcommand{\ki}[2]{\hfill {\it #1 (#2)}\medskip}
\newcommand{\vonal}{\hbox to \hsize{\hskip2truecm\hrulefill\hskip2truecm}}
\newcommand{\degre}{\ensuremath{^\circ}}
\newcommand{\tg}{\mathop{\mathrm{tg}}\nolimits}
\newcommand{\ctg}{\mathop{\mathrm{ctg}}\nolimits}
\newcommand{\arc}{\mathop{\mathrm{arc}}\nolimits}
\begin{document}
\begin{center} \Large {\em 25. Nemzetközi Magyar Matematika Verseny} \end{center}
\begin{center} \large{\em Budapest, 2016. március 11-15.} \end{center}
\smallskip
\begin{center} \large{\bf 11. osztály} \end{center}
\bigskip 

{\bf 1. feladat: } Egy háromszög három oldalának mérőszáma, $a,b,c$ ebben a sorrendben egy mértani sorozat három egymást követő tagja. Bizonyítsa be, hogy $a^2+c^2<3ac$.

\ki{Minda Mihály}{Vác}\medskip

{\bf 2. feladat: } Egy interneten lebonyolított bajnokságon minden
résztvevő minden másik résztvevővel pontosan kétszer játszott. Egy
mérkőzésen a győztes $2$, a vesztes 0 pontot kapott, döntetlen esetén
mindkét játékosnak 1-1 pont járt. Az eredménylista összeállítói meglepve 
tapasztalták,
hogy az utolsó helyezett kivételével minden versenyző pontszáma úgy
adódik, hogy a közvetlenül mögötte végző pontszámához mindig ugyanazt a
páros számot hozzáadjuk. A győztes $2016$ pontot szerzett. Hányan vettek
részt a versenyen? 

\ki{Tóth Sándor}{Kisvárda}\medskip

{\bf 3. feladat: } Az $ABC$ derékszögű háromszögben az $A$ csúcsnál levő szög $\alpha$. Az $AB$ átfogóhoz tartozó magasság az átfogót a $D$ pontban metszi. Az $ADC$ háromszögbe olyan $DEFG$ négyzetet rajzolunk, amelynek $E$, $F$ és $G$ csúcsai rendre $DC$-re, $CA$-ra és $AD$-re illeszkednek, a $CDB$ háromszögbe pedig olyan $DHIJ$ négyzetet, amelynek $H$, $I$ és $J$ csúcsai $DB$-re, $BC$-re és $CD$-re esnek. Jelölje $t_1$ és $t_2$ a $DEFG$, illetve a $DHIJ$ négyzet területét. Bizonyítsa be, hogy 

\[\cos\alpha =\sqrt{\frac{t_1}{t_1+t_2}}\,.\]

\ki{Bíró Bálint}{Eger}\medskip

{\bf 4. feladat: } Az $a_n$ sorozatban $a_1=1$ és $a_n=n\left(a_1+a_2+a_3+\ldots+a_{n-1}\right)$, ha $n \geq 2$. Határozza meg $a_{2016}$ értékét.

\ki{Nagy Piroska Mária}{Dunakeszi}

\ki{Szoldatics József}{Budapest}\medskip

{\bf 5. feladat: } Jelölje $p_n$ az $n$-edik prímszámot ($p_1=2, p_2=3, \dots$). Bizonyítsa be, hogy minden $n$ pozitív egész szám esetén
\[\frac{1}{p_1p_2}+\frac1{p_2p_3}+\ldots+\frac{1}{p_np_{n+1}}<\frac13.\]

\ki{Bencze Mihály}{Bukarest}\medskip

{\bf 6. feladat: } Az $ABCD$ paralelogramma $A$ csúcsán áthaladó kör az $AB$, $AD$ oldalakat és az $AC$ átlót rendre az $M$, $N$, illetve $K$ pontokban metszi. Bizonyítsa be, hogy 
\[AB\cdot AM + AD\cdot AN = AC\cdot AK.\]

\ki{Róka Sándor}{Nyíregyháza}\medskip


\end{document}