\documentclass[a4paper,10pt]{article} 
\usepackage[utf8]{inputenc}
\usepackage[a4paper]{geometry}
\usepackage[magyar]{babel}
\usepackage{amsmath}
\usepackage{amssymb}
\frenchspacing 
\pagestyle{empty}
\newcommand{\ki}[2]{\hfill {\it #1 (#2)}\medskip}
\newcommand{\vonal}{\hbox to \hsize{\hskip2truecm\hrulefill\hskip2truecm}}
\newcommand{\degre}{\ensuremath{^\circ}}
\newcommand{\tg}{\mathop{\mathrm{tg}}\nolimits}
\newcommand{\ctg}{\mathop{\mathrm{ctg}}\nolimits}
\newcommand{\arc}{\mathop{\mathrm{arc}}\nolimits}
\begin{document}
\begin{center} \Large {\em XX. Nemzetközi Magyar Matematika Verseny} \end{center}
\begin{center} \large{\em Bonyhád, 2011. március 11--15.} \end{center}
\smallskip
\begin{center} \large{\bf 12. osztály} \end{center}
\bigskip 

{\bf 1. feladat: } 
Bizonyítsuk be, hogy ha az $a$, $b$,$c$ pozitív valós számok kielégítik az
\[5abc>a^3+b^3+c^3\]
egyenlőtlenséget, akkor létezik $a$, $b$, $c$ oldalú háromszög.

\ki{Oláh György}{Komárom}\medskip

{\bf 2. feladat: } 
Legyen $a_n$ ($n\in\mathbb{N}^+$) az $\sqrt{n}$-hez legközelebbi egész szám. (Ha $n$ négyzetszám, akkor $a_n=\sqrt{n}$.) Mennyi az
\[\frac{1}{a_1}+\frac{1}{a_2}+\ldots+\frac{1}{a_{2011}}\]
összeg értéke?

\ki{Kántor Sándor}{Debrecen}\medskip

{\bf 3. feladat: } 
Az $ABC$ háromszögbe írható kör $O$ középpontjára illeszkedő $e$ egyenes az $AB$ és $AC$ oldalakat $M$ és $N$ pontokban metszi. $D$ és $E$ a $BO$ és $CO$ egyenesek olyan pontjai, amelyre $ND\parallel ME\parallel BC$. Igazoljuk, hogy az $A$, $D$ és $E$ pontok egy egyenesre illeszkednek.

\ki{Katz Sándor}{Bonyhád}\medskip

{\bf 4. feladat: } 
Az $ABC$ háromszög $A$ csúcsához tartozó magasságának a $BC$ oldal egyenesén levő talppontja $D$. A $B$ és $C$ pontokból az $A$ csúcsból induló belső szögfelezőre bocsátott merőlegesek talppontjai rendre $E$ és $F$. Az $EF$ és $BC$ szakaszok metszéspontja $M$. Legyen az $ABC$ háromszög területe $T$, a $DEF$ háromszög területe $t$.

Bizonyítsuk be, hogy
\[\sqrt{\frac t T} = \frac{FM \cdot BM \cdot DE}{EM \cdot CM \cdot AB}.\]

\ki{Bíró Bálint}{Eger}\medskip

{\bf 5. feladat: } 
Adott egy tetszőleges poliéder. Lehet-e a csúcsaiba pozitív egész számokat írni a következő módon:
\begin{itemize}
\item[] 
ha él köt össze két csúcsot, akkor a csúcsokba írt számok relatív prímek;
\item[] 
ha két csúcs nincs éllel összekötve, akkor a csúcsokba írt számok legnagyobb közös osztója 1-nél nagyobb?
\end{itemize}

\ki{Kántor Sándor}{Debrecen}\medskip

{\bf 6. feladat: } 
Létezik-e olyan négyzetszám, amelynek a számjegyeinek összege  $2011^{2010}$?

\ki{Szabó Magda}{Szabadka}
\end{document}