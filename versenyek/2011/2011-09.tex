\documentclass[a4paper,10pt]{article} 
\usepackage[utf8]{inputenc}
\usepackage{graphicx}
\usepackage{amssymb}
\voffset - 20pt
\hoffset - 35pt
\textwidth 450pt
\textheight 650pt 
\frenchspacing 

\pagestyle{empty}
\def\ki#1#2{\hfill {\it #1 (#2)}\medskip}

%FoBe-féle hozzáadások
\usepackage[magyar]{babel}
%vége

\def\tg{\, \hbox{tg} \,}
\def\ctg{\, \hbox{ctg} \,}
\def\arctg{\, \hbox{arctg} \,}
\def\arcctg{\, \hbox{arcctg} \,}

\begin{document}
\begin{center} \Large {\em XX. Nemzetközi Magyar Matematika Verseny} \end{center}
\begin{center} \large{\em Bonyhád, 2011. március 11--15.} \end{center}
\smallskip
\begin{center} \large{\bf 9. osztály} \end{center}
\bigskip 

{\bf 1. feladat: }
,,Fanyűvő és én együtt 20 nap alatt vágnánk ki a Nagy Kerek Erdőt'' -- mondja Törzsök Jankó. ,,Bár ha Erdődöntögetővel dolgoznék, akkor ezt a munkát öt nappal előbb befejeznénk.'' ,,Nekem jobb ötletem van'' -- mondja Erdődöntögető. ,,Ha én dolgoznék együtt Fanyűvővel, akkor mi ketten egy ötödével kevesebb idő alatt végeznénk a munkával, mint ha Törzsök Jankóval dolgoznék.'' Mennyi idő alatt vágnák ki a Nagy Kerek Erdőt külön-külön ezek az erős emberek, és mennyi idő alatt végeznének a munkával, ha mindhárman együtt dolgoznának?

\ki{Peics Hajnalka}{Szabadka}\medskip

{\bf 2. feladat: } 
Bizonyítsuk be, hogy minden $n\in\mathbb{N}^+$ számra a $(2n+1)^2+(2n+2)^2+(2n+3)^2$ kifejezés felírható 4 különböző pozitív egész szám négyzetösszegeként.

\ki{Bencze Mihály}{Brassó}\medskip

{\bf 3. feladat: } 
Az $ABC$ egyenlő szárú háromszögben $A\sphericalangle=100^\circ$. Vegyük fel az $AB$ szár $B$-n túli meghosszabbításán a $D$ pontot úgy, hogy $AD=BC$ legyen. Mekkorák a $BCD$ háromszög szögei?

\ki{Katz Sándor}{Bonyhád}\medskip

{\bf 4. feladat: } 
Az $ABC$ háromszög $AB$, $BC$, $CA$ oldalait meghosszabbítjuk a $B$, $C$ és $A$ pontokon túl a $BB_1$, $CC_1$, $AA_1$ szakaszokkal úgy, hogy $BB_1=AC$, $CC_1=AB$, $AA_1=BC$ legyen. Jelölje továbbá az $ABC$, $AA_1B$, $BB_1C$, $CC_1A$ háromszögek területét $T_{ABC}$, $T_{AA_1B}$, $T_{BB_1C}$, $T_{CC_1A}$. Mutassuk meg, hogy $T_{AA_1B}+T_{BB_1C}+T_{CC_1A}\ge3T_{ABC}$.

\ki{Pintér Ferenc}{Nagykanizsa}\medskip

{\bf 5. feladat: } 
Legfeljebb hány oldalú az a konvex sokszög, amely feldarabolható olyan derékszögű háromszögekre, amelyek hegyesszögei 30 és 60 fokosak? (A feldarabolás során csak ilyen háromszög keletkezhet, másféle sokszög nem).

\ki{Kiss Sándor}{Nyíregyháza}\medskip

{\bf 6. feladat: } 
A 957 háromjegyű szám mögé írjunk három számjegyet úgy, hogy a kapott hatjegyű szám osztható legyen 9-cel, 5-tel és 7-tel is! Melyek ezek a háromjegyű számok?

\ki{Pintér Ferenc}{Nagykanizsa}\medskip

\vfill
\end{document}