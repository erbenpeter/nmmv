\documentclass[a4paper,10pt]{article} 
\usepackage[utf8]{inputenc}
\usepackage[a4paper]{geometry}
\usepackage[magyar]{babel}
\usepackage{t1enc}
\usepackage{amsmath}
\usepackage{amssymb}
\frenchspacing 
\pagestyle{empty}
\newcommand{\ki}[2]{\hfill {\it #1 (#2)}\medskip}
\newcommand{\vonal}{\hbox to \hsize{\hskip2truecm\hrulefill\hskip2truecm}}
\newcommand{\degre}{\ensuremath{^\circ}}
\newcommand{\tg}{\mathop{\mathrm{tg}}\nolimits}
\newcommand{\ctg}{\mathop{\mathrm{ctg}}\nolimits}
\newcommand{\arc}{\mathop{\mathrm{arc}}\nolimits}
\begin{document}
\begin{center} \Large {\em 23. Nemzetközi Magyar Matematika Verseny} \end{center}
\begin{center} \large{\em Csíkszereda, 2014. március 12-16.} \end{center}
\smallskip
\begin{center} \large{\bf 12. osztály} \end{center}
\bigskip

{\bf 1. feladat: } Az $ABC$ háromszögben $ACB\sphericalangle = 60^\circ$ és $AC\le
BC$. Legyen $D$ az $AC$ oldal egy belső pontja. Vedd fel az $E$
pontot a $BC$ oldal belsejében úgy, hogy $AD = BE$
teljesüljön. A $DE$ szakasz fölé rajzold meg a $DEF$
szabályos háromszöget úgy, hogy $DEF$ és $ABC$ azonos
körüljárásúak legyenek. Bizonyítsd be, hogy az $F$ pont
illeszkedik az $ABC$ háromszög köré írt körre!


\ki{Nemecskó István}{Budapest}\medskip

{\bf 2. feladat: } Az $ABCDEFGH$ kocka élének a hossza 1 cm. Egy hangya az $A$
csúcsból indulva egy 2014 cm hosszúságú utat jár be
úgy, hogy csak az éleken közlekedik (egy élen végig mehet
többször is). Melyik útból van több: amelyik az $A$
csúcsban, vagy amelyik a C csúcsban végződik?


\ki{Keke\v{n}ák Szilvia}{Kassa}\medskip

{\bf 3. feladat: } Adottak az $a,b,c \in \left\{ {0,1,2,...,9} \right\}$ számjegyek
úgy, hogy az $\overline {abc} $ háromjegyű szám prímszám.
Bizonyítsd be, hogy az $a{x^2} + bx + c = 0$ egyenletnek nincsenek
racionális gyökei!


\ki{dr. Bencze Mihály}{Bukarest}\medskip

{\bf 4. feladat: } Az $\displaystyle\frac{1}{2014!\cdot 2015!}$ racionális szám
tizedes tört alakja $$0,a_1a_2\ldots a_n(b_1b_2\ldots b_k),$$ ahol
$(b_1b_2\ldots b_k)$ az ismétlődő szakasz és az $n,$
illetve $k$ értéke a lehető legkisebb. Mennyi az $n$
értéke?

\ki{dr. Gecse Frigyes}{Kisvárda}\medskip

{\bf 5. feladat: } Adott a $p$ prímszám és $a$ darab számozott doboz, ahol
$a\ge 2.$ Felírtuk $p$ darab golyóra a számokat $1$-től
$p$-ig és a golyókat valahogyan elhelyeztük a dobozokban.
Számold meg, hogy hány különböző elhelyezésre lesz az
első dobozban található golyókon szereplő számok
összege osztható $p$-vel! (Egy üres dobozban a golyókon
szereplő számok összege egyezményesen $0.$)


\ki{dr. András Szilárd, dr. Lukács Andor}{Kolozsvár}\medskip

{\bf 6. feladat: } a) Határozd meg a síknak egységoldalú szabályos
háromszögekkel és egy\-ség\-oldalú négyzetekkel való
összes szabályos lefödését! Egy lefödés azt jelenti,
hogy a sokszögek hézag és átfödés nélkül (egyrétűen) lefödik a síkot. A lefödés szabályos, ha léteznek
olyan $a,b$ nullától különböző természetes számok,
a\-me\-lyekre minden keletkező csúcs körül pontosan $a$
darab háromszög és $b$ darab négyzet van, valamilyen
rögzített sorrendben.

b) Bizonyítsd be, hogy létezik végtelen sok, páronként
különböző, nem feltétlenül sza\-bá\-lyos lefödés
(az előbbi háromszögekkel és négyzetekkel), amelyekhez
hozzárendelhetők az $a,b$ nullától különböző
termé\-sze\-tes számok úgy, hogy minden keletkező csúcs
körül pontosan $a$ darab háromszög és $b$ darab négyzet
legyen, de ezeknek a sokszögeknek a sorrendje ne legyen minden
csúcspontban ugyanolyan.


\ki{Zsombori Gabriella}{Csíkszereda}


\ki{dr. András Szilárd, dr. Lukács Andor}{Kolozsvár}


\end{document}