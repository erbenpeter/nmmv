\documentclass[a4paper,10pt]{article} 
\usepackage[utf8]{inputenc}
\usepackage[a4paper]{geometry}
\usepackage[magyar]{babel}
\usepackage{amsmath}
\usepackage{amssymb}
\frenchspacing 
\pagestyle{empty}
\newcommand{\ki}[2]{\hfill {\it #1 (#2)}\medskip}
\newcommand{\vonal}{\hbox to \hsize{\hskip2truecm\hrulefill\hskip2truecm}}
\newcommand{\degre}{\ensuremath{^\circ}}
\newcommand{\tg}{\mathop{\mathrm{tg}}\nolimits}
\newcommand{\ctg}{\mathop{\mathrm{ctg}}\nolimits}
\newcommand{\arc}{\mathop{\mathrm{arc}}\nolimits}
\begin{document}
\begin{center} \Large {\em 23. Nemzetközi Magyar Matematika Verseny} \end{center}
\begin{center} \large{\em Csíkszereda, 2014. március 12-16.} \end{center}
\smallskip
\begin{center} \large{\bf 9. osztály} \end{center}
\bigskip 

{\bf 1. feladat: } Egy könyvtárban megszámozták az összes könyvet. A
számozáshoz $1$-től kez\-dő\-dő\-en egymást követő
természetes szá\-mo\-kat használtak, és ugyanazt a számot
nem írták rá két könyvre. A megszámozás során a
könyvekre háromszor annyi számjegyet kellett
ráírni, mint ahány könyv volt a könyvtárban.
Hány könyv volt a könyvtárban?

\ki{\fbox{Oláh György}}{Révkomárom}\medskip

{\bf 2. feladat: } Határozd meg az $(a+b)(b+c)(c+a)=1144$ egyenlet összes
nullától különböző természetes megoldását!

\ki{dr.~Hraskó András}{Budapest}\medskip

{\bf 3. feladat: } Az $ABCD$ konvex négyszögben $AB=1,$ $BC=2,$ $AD = \sqrt{2}$,
$A\sphericalangle = 105^{\circ}$ és $B\sphericalangle  = 60^\circ$.
Számítsd ki a $CD$ oldal hosszát!


\ki{Kovács Lajos}{Székelyudvarhely}

\ki{dr.~András Szilárd}{Kolozsvár}\medskip

{\bf 4. feladat: } Hány valós megoldása van a $3\left[  x\right]
=2x^{2}+x-4$ egyenletnek? ($\left[  x\right]  $ az $x$ valós
szám egész részét jelenti.)

\ki{Szabó Magda}{Szabadka}

\ki{Longáver Lajos}{Nagybánya}\medskip

{\bf 5. feladat: } Egy számítógép segítségével kinyomtatták a
$2^{2014}$ és az $5^{2014}$ hatványok értékét
tízes számrendszerben. Összesen hány számjegyet
nyomtattak? (Pl. a $11231$ szám kinyomtatásánál $5$
számjegyet nyomtatnának.)

\ki{dr.~Katz Sándor}{Bonyhád}\medskip

{\bf 6. feladat: } a) Határozd meg a síknak egységoldalú szabályos
háromszögekkel és egy\-ség\-oldalú szabályos
hatszögekkel való összes szabályos lefödését! Egy
lefödés azt jelenti, hogy a sokszögek hézag és
átfödés nélkül (egyrétűen) lefödik a síkot. A
lefödés sza\-bá\-lyos, ha léteznek olyan $a$ és $b$
nullától különböző természetes számok, amelyekre
minden keletkező csúcs körül pontosan $a$ darab
háromszög és $b$ darab hatszög van, valamilyen rögzített
sorrendben.

\smallskip
b) Bizonyítsd be, hogy van olyan, nem feltétlenül szabályos
lefödés is (az előbbi háromszögekkel és
hatszögekkel), amelyben létezik végtelen sok, páron\-ként
különböző mintázat, amely véges sokszor jelenik meg!
(Min\-tá\-zat alatt a lefödés véges sok sokszöge által
megha\-tá\-ro\-zott összefüggő alakzatot értünk.)



\ki{Zsombori Gabriella}{Csíkszereda}

\ki{dr.~András Szilárd, dr.~Lukács Andor}{Kolozsvár}



\end{document}