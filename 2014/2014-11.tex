\documentclass[a4paper,10pt]{article} 
\usepackage[utf8]{inputenc}
\usepackage[a4paper]{geometry}
\usepackage[magyar]{babel}
\usepackage{t1enc}
\usepackage{amsmath}
\usepackage{amssymb}
\frenchspacing 
\pagestyle{empty}
\newcommand{\ki}[2]{\hfill {\it #1 (#2)}\medskip}
\newcommand{\vonal}{\hbox to \hsize{\hskip2truecm\hrulefill\hskip2truecm}}
\newcommand{\degre}{\ensuremath{^\circ}}
\newcommand{\tg}{\mathop{\mathrm{tg}}\nolimits}
\newcommand{\ctg}{\mathop{\mathrm{ctg}}\nolimits}
\newcommand{\arc}{\mathop{\mathrm{arc}}\nolimits}
\begin{document}
\begin{center} \Large {\em 23. Nemzetközi Magyar Matematika Verseny} \end{center}
\begin{center} \large{\em Csíkszereda, 2014. március 12-16.} \end{center}
\smallskip
\begin{center} \large{\bf 11. osztály} \end{center}
\bigskip


{\bf 1. feladat: } Milyen szabály szerint írtuk le a $$2, 10, 16, 32, 42, 66, 80,
112, 130, 170$$ számokat? Ezt a szabályt folytatva, add meg a
    \[
        2, 10, 16, 32, 42, 66, 80, 112, 130, 170, \ldots
    \]
    sorozat általános tagjának a képletét!


\ki{dr. Kántor Sándorné}{Debrecen}\medskip

{\bf 2. feladat: } Oldd meg az $\displaystyle x^{\log_3 64} = x^2\cdot8^{\log_3 x} -
x^{\log_3 8}$ egyenletet a valós számok halmazán!

\ki{Balázsi Borbála}{Beregszász}\medskip

{\bf 3. feladat: } Legfeljebb hány elemet tartalmazhat a $$H = \{1,2,3,\dots, 25\}$$
halmaznak az a részhalmaza, amelyben bármely két szám
 szorzata nem négyzetszám? Adj meg egy ilyen részhalmazt!


\ki{Bíró Béla}{Sepsiszentgyörgy}\medskip

{\bf 4. feladat: } Az $e$ és $f$ egyenesek párhuzamosak és egymástól
egységnyi távolságra vannak. Vedd fel az $e$ egyenesen az
$A_1,$ $A_2,$ $A_3,$ $\dots, A_n, A_{n+1}$ pontokat és az $f$
egyenesen a $B_1, B_2, \dots, B_n,$ $B_{n+1}$ pontokat úgy, hogy
mindkét egyenesen bármely két szomszédos pont távolsága
egységnyi, és minden $i\in\mathbb{N}$, $1\le i\le n+1$ számra
az $A_iB_i$ szakasz merőleges az $e$ és $f$ egyenesekre. Kösd
össze az $A_1$ pontot a $B_n$ és $B_{n+1}$ pontokkal. Az
összekötő sza\-ka\-szok az $A_2B_2$ szakaszt rendre a $P$
és $Q$ pontokban metszik.

a) Van-e olyan pozitív egész $n$ szám, amelyre az $A_1PQ$
háromszög területe $\displaystyle \frac{1}{1802}$
területegység?

b) Van-e olyan pozitív egész $n$ szám, amelyre az $A_1PQ$
háromszög területe $\displaystyle \frac{1}{1860}$
területegység?

\ki{Bíró Bálint}{Eger}\medskip

{\bf 5. feladat: } Egy szabályos kilencszögben meghúztuk az összes átlót.
 Van-e a kilencszög belsejében olyan pont, amelyre legalább három átló
 illeszkedik?

\ki{Zsombori Gabriella}{Csíkszereda}

\ki{dr. András Szilárd}{Kolozsvár}\medskip

{\bf 6. feladat: } a) Határozd meg a síknak egy\-ségoldalú szabályos
háromszögekkel, egy\-ség\-oldalú négyzetekkel és
egységoldalú sza\-bályos hatszögekkel való összes
szabályos le\-fö\-dé\-sét! Egy lefödés azt jelenti, hogy
a sokszögek hézag és átfödés nélkül (egyrétűen)
lefödik a síkot. A lefödés szabályos, ha léteznek olyan
$a,b$ és $c$ nullától különböző természetes
számok, amelyekre minden keletkező csúcs körül pontosan
$a$ darab háromszög, $b$ darab négyzet és $c$ darab
hatszög van, valamilyen rögzített sorrendben.

b) Bizonyítsd be, hogy létezik végtelen sok olyan, nem
fel\-tétle\-nül szabályos lefödés, a\-mely\-hez
hozzárendelhető valamilyen $a,b$ és $c$ nullától
különböző természetes szám úgy, hogy minden
keletkező csúcs körül pontosan $a$ darab háromszög, $b$
darab négyzet és $c$ darab hatszög legyen!

\ki{Zsombori Gabriella}{Csíkszereda}

\ki{dr. András Szilárd, dr. Lukács Andor}{Kolozsvár}\medskip


\end{document}