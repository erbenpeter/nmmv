\documentclass[a4paper,10pt]{article} 
\usepackage[utf8]{inputenc}
\usepackage{t1enc}
\usepackage{graphicx}
\usepackage{amssymb}
\voffset - 20pt
\hoffset - 35pt
\textwidth 450pt
\textheight 650pt 
\frenchspacing 

\pagestyle{empty}
\def\ki#1#2{\hfill {\it #1 (#2)}\medskip}

\def\tg{\, \hbox{tg} \,}
\def\ctg{\, \hbox{ctg} \,}
\def\arctg{\, \hbox{arctg} \,}
\def\arcctg{\, \hbox{arcctg} \,}

\begin{document}
\begin{center} \Large {\em XIX. Nemzetközi Magyar Matematika Verseny} \end{center}
\begin{center} \large{\em Szatmárnémeti, 2010. március 19-22.} \end{center}
\smallskip
\begin{center} \large{\bf 11. osztály} \end{center}
\bigskip 

{\bf 1. feladat: } Határozzuk meg az $1,$ $2,$ $4,$ $5,$ $7,$ $9,$ $10,$ $12,$ $14,$
$16,$ $17,$ $19,$ $21,$ $23,$ $25,$ $\ldots$ sorozat $2010.$ tagját,
ha a sorozat tagjait úgy képeztük, hogy az $1$-es után leírtuk az őt
követő $2$ páros számot, majd a kapott számot követő $3$ páratlan
számot, az ezután kapott számot követő $4$ páros számot és így
tovább.

\ki{Pintér Ferenc}{Nagykanizsa}\medskip

\textbf{1. feladat I. megoldása: } Vegyük észre, hogy a leírt részsorozatok mindig négy\-zetszámmal végződnek. Teljes indukcióval belátjuk,
hogy ha $k^2$ után következő $k+1$ számot leírjuk az adott szabály
szerint, akkor $(k+1)^2$-hez jutunk. Ehhez csak azt kell
észrevennünk,
 hogy az első leírt szám $k^2+1$ és azt követően $k$ darab 2-est kell hozzáadnunk, hogy a ,,váltás'' bekövetkezzen.
Viszont $k^2+1+2k=(k+1)^2$.\\
Ha a leírtakat figyelembe vesszük és $k$-val jelöljük azt a
leg\-na\-gyobb egész számot, melynek a négyzete a $2010.$ tag el\H
ott van a sorozatban, akkor teljesülnie kell a következő
egyenlőtlenségnek:
$$1+2+3+\dots+k=\frac{k(k+1)}{2}\leq 2010.$$
Innen $k=62$, ekkor $\frac{k(k+1)}{2}=1953$. Következésképpen a
sorozat $2010.$ tagját megkapjuk, ha $62^2+1=3845$-höz hozzáadunk
 $(2010-1953-1)\cdot2=56\cdot2$-t, azaz
$3845+56\cdot2=3957$.

\medskip

\textbf{1. feladat II. megoldása: } Számítsuk ki a sorozat
első néhány tagját és figyeljük  az indexek és a
tagok közti összefüggést:
\begin{center}
\begin{tabular}{c|cc|ccc|cccc|c}
1&2&4&5&7&9&10&12&14&16&17\\
\hline
$x_1$&$x_2$&$x_3$&$x_4$&$x_5$&$x_6$&$x_7$&$x_8$&$x_9$&$x_{10}$&$x_{11}$
\end{tabular}
\end{center}
A ,,váltások'' az $1,3,6,10,15,\ldots$ indexeknél
következnek be, hisz a csoportok (a táblázatban függ\H
oleges vonallal elválasztott részek) rendre  $1,2,3,4,5,\ldots$
elemet tartalmaznak. így az első $k$ csoport összesen
$1+2+3+\ldots +k=\frac{k(k+1)}{2}$ számot tartalmaz, vagyis a
$k$-adik csoport utolsó elemének indexe $\frac{k(k+1)}{2}.$ Ha a
sorozatnak csak az $u_k=x_{\frac{k(k+1)}{2}}$ részsorozatát
nézzük, akkor a sorozat képzési szabálya alapján
$u_{k+1}=u_k+2k+1$ és $u_1=1,$ tehát
$$u_k=\underbrace{1+3+5+\ldots +(2k-1)}_{k \mbox{ tag}}=k^2.$$
Ugyanakkor $\frac{62\cdot 63}{2}=1953<2010<2016=\frac{63\cdot
64}{2},$ tehát $x_{2016}=63^2$ és ez a $63$-adik csoport
utolsó tagja. A csoporton belül az egymásutáni tagok
különbsége $2,$ tehát
$$x_{2010}=x_{2016}-2\cdot 6=3969-12=3957.$$

\medskip


{\bf 2. feladat: } Az $x_1,x_2,\ldots, x_n$ nemnegatív valós számok teljesítik
az $$x_1^2+x_2^2+\ldots +x_n^2 + 1\cdot x_1+2\cdot x_2+3\cdot
x_3+\ldots +n\cdot x_n=2010$$ egyenlőséget. Határozzuk meg az
$x_1+x_2+\ldots +x_n$ legkisebb lehetséges értékét!

\ki{Borbély József}{Tata}\medskip

\textbf{2. feladat I. megoldása: } Mivel az $x_1,x_2,\ldots, x_n$ számok
nemnegatívak,
ír\-hat\-juk, hogy
$$1\cdot x_1+2\cdot x_2+3\cdot
x_3+\ldots +n\cdot x_n\leq n( x_1+ x_2+ x_3+\ldots + x_n)
\,\,\,\mbox{és}$$
$$x_1^2+x_2^2+\ldots +x_n^2\leq (x_1+x_2+\ldots +x_n)^2,$$ tehát,
ha $x_1+x_2+\ldots +x_n=s,$ akkor $2010\leq ns+s^2.$ De az
$s^2+ns-2010=0$ egyenlet egyik gyöke negatív és a másik
pozitív, tehát az előbbi egyenlőtlenség alapján
$$s\geq \frac{-n+\sqrt{n^2+4\cdot 2010}}{2}.$$ Másrészt ezt az
értéket el is érhetjük, ha $$x_1=x_2=\ldots =x_{n-1}=0
\mbox{ és } x_n=\frac{-n+\sqrt{n^2+4\cdot 2010}}{2},$$ tehát az
$x_1+x_2+\ldots +x_n$ összeg minimuma $\frac{-n+\sqrt{n^2+4\cdot
2010}}{2}.$

\medskip
 
\textbf{2. feladat II. megoldása: }  Tekintsünk egy $(x_1,x_2,\ldots,
x_n)$ szám $n$-est, amely teljesíti a feltételeket. Ha
létezik olyan $j<n,$ amelyre $x_j\neq 0,$ akkor az $x_j$
értékét cseréljük  ki $0$-ra és az $x_n$ értékét
$u>0$-ra úgy, hogy az $(x_1,x_2,\ldots,x_{j-1},0,x_{j+1},\ldots,
u)$ számokra is teljesüljön a feltétel. Mivel egyszerre csak
két számot módosítunk, az $$u^2+nu=x_j^2+jx_j+x_n^2+nx_n$$
egyenlőségnek kell teljesülnie. Ugyanakkor az $u\to u^2+nu$
függ\-vény növekvő (mert $u>0$) és világos, hogy
$$(x_n+x_j)^2+n(x_n+x_j)>x_j^2+jx_j+x_n^2+nx_n,$$ tehát
$x_n+x_j>u.$ Ezzel beláttuk, hogy a végrehajtott csere során
az $x_1+x_2+\ldots + x_n$ összeg csökken. Mivel legfeljebb
$(n-1)$ cserével mindig eljuthatunk a $(0,0,\ldots, 0,x)$
számokhoz, ahol $$x=\frac{-n+\sqrt{n^2+4\cdot 2010}}{2},$$ ezért
az $x_1+\ldots +x_n$ összeg legkisebb lehetséges értéke
$$\frac{-n+\sqrt{n^2+4\cdot 2010}}{2}.$$

\medskip

{\bf 3. feladat: } Határozzuk meg a
$$\frac{\sqrt{1-x^2}}{x}=3-4x^2$$ egyenlet valós megoldásait!

\ki{Szilágyi Judit és Nagy Örs}{Kolozsvár}\medskip

\textbf{3. feladat I. megoldása: } Létezési feltételek: $x\neq0$ és $1-x^2\geq0$, azaz $D=[-1,1]\setminus\{0\}$.\\
$\frac{\sqrt{1-x^2}}{x}=3-4x^2 \Longleftrightarrow
\sqrt{1-x^2}=3x-4x^3$. Mivel a bal oldal po\-zitív, ezért a jobb
oldalnak is po\-zitívnak kell lennie, azaz $3x-4x^3=x(3-4x^2)\geq0$,
ahonnan a létezési feltételeket figyelembe véve $x\in\left[-1,
-\frac{\sqrt{3}}{2}\right]\cup\left(0,\frac{\sqrt{3}}{2}\right]$.

Négyzetre emelés és átcsoportosítás után a $$16x^6-24x^4+10x^2-1=0$$
egyenlethez jutunk. Bevezetve a $t=x^2,\ t\geq0$ jelölést, kapjuk,
hogy $$16t^3-24t^2+10t-1=0.$$ Észrevehető, hogy $t_1=\frac{1}{2}$
kielégíti az egyenletet, így a
$$\left(t-\frac{1}{2}\right)(16t^2-16t+2)=0$$ alakhoz jutunk, ahonnan
$t_2=\frac{2-\sqrt{2}}{4}$ és $t_3=\frac{2+\sqrt{2}}{4}$. Tehát
$$x_{1,2}=\pm\frac{\sqrt{2}}{2},\,\,
x_{3,4}=\pm\frac{\sqrt{2-\sqrt{2}}}{2},\,\,
x_{5,6}=\pm\frac{\sqrt{2+\sqrt{2}}}{2}.$$ A létezési
feltételek alapján
$$M=\left\{\frac{\sqrt{2}}{2},\frac{\sqrt{2-\sqrt{2}}}{2},-\frac{\sqrt{2+\sqrt{2}}}{2}\right\}.$$

\medskip

\textbf{3. feladat II. megoldása: } A négyzetgy\" ok létezéséhez szükséges, hogy \linebreak $x\in [-1,1].$ így viszont
létezik olyan $t\in \left [-\frac{\pi}{2},\frac{\pi}{2}\right ],$
amelyre $x=\sin t.$ A tört létezéséhez $x\neq 0,$ tehát
$t\neq 0.$ Elvégezve az $x=\sin{t}, \ t\in\left
[-\frac{\pi}{2},\frac{\pi}{2}\right ]$ helyettesítést a
$\cos{t}=\sin{3t}$ egyenlethez jutunk. Ez rendre a következ\H
oképpen alakítható:

$$\cos{\left(\frac{\pi}{2}-3t\right)}-\cos{t}=0$$

$$-2\sin\left(\frac{\pi}{4}-t\right)\sin\left(\frac{\pi}{4}-2t\right)=0,$$
ahonnan a vizsgált intervallumban csak a
$$t_1=\frac{\pi}{4},\ \  t_2=\frac{\pi}{8},\ \  t_3=-\frac{\pi}{8}$$
megoldások lehetségesek, tehát
$$M=\left  \{\sin \frac{\pi}{4},\sin \frac{\pi}{8},-\sin \frac{\pi}{8}\right \}$$
A $\sin t=\sqrt{\frac{1-\cos 2t}{2}}$ összefüggés alapján
$\sin \frac{\pi}{8}=\frac{\sqrt{2-\sqrt{2}}}{2},$ tehát a
megoldáshalmazt felírhatjuk
$$M=\left\{\frac{\sqrt{2}}{2},\frac{\sqrt{2-\sqrt{2}}}{2},-\frac{\sqrt{2+\sqrt{2}}}{2}\right\}$$
alakban is.

\medskip


{\bf 4. feladat: } Az $ABCD$ konvex négyszögben jelölje $\alpha$ a $d_1$ és $d_2$
hosszúságú $AC,$ illetve $BD$ átló által közrezárt szög
mértékét. Mutassuk ki, hogy $ABCD$ akkor és csakis akkor
négyzet, ha
$$(d_1\sin{A}\sin{C}+d_2\sin{B}\sin{D})\cdot\sin{\alpha}=
\sqrt{2(d_1^2+d_2^2)}.$$

\ki{Longáver Lajos}{Nagybánya}\medskip

\textbf{4. feladat I. megoldása: } A szinuszok majorálását, illetve a számtani és négy\-zetes középarányos közötti egyenlőtlenséget alkalmazzuk:
$$(d_1\sin{A}\sin{C}+d_2\sin{B}\sin{D})\cdot\sin{\alpha}\leq $$ $$\leq d_1\sin{A}\sin{C}+d_2\sin{B}\sin{D}\leq$$
$$\leq d_1+d_2\leq2\sqrt{\frac{d_1^2+d_2^2}{2}}=\sqrt{2(d_1^2+d_2^2)}.$$\\
Egyenlőséget pontosan akkor kapunk, ha
$$\sin{A}=\sin{B}=\sin{C}=\sin{D}=\sin{\alpha}=1$$ és $d_1=d_2$. Az
$ABCD$ szögei pontosan akkor derékszögek, ha $ABCD$ téglalap, és
egy téglalap átlói pontosan akkor merőlegesek egy\-másra, ha az
négyzet, tehát a bizonyítás teljes.

\medskip

\textbf{4. feladat II. megoldása: } A Cauchy-Buniakovski-Schwarz
egyenlőtlenség alapján
$$(d_1\sin{A}\sin{C}+d_2\sin{B}\sin{D})^2\leq$$
$$\leq \left(d_1^2+d_2^2 \right)\left (\sin^2 A \sin^2 C+\sin^2 B\sin^2 D
\right )\leq 2\left(d_1^2+d_2^2 \right),$$ tehát a  $\sin
\alpha\leq 1$ egyenlőtlenség alapján következik, hogy
$$(d_1\sin{A}\sin{C}+d_2\sin{B}\sin{D})\cdot\sin{\alpha}\leq
\sqrt{2(d_1^2+d_2^2)}.$$ Egyenlőség pontosan akkor teljesül,
ha
$$\sin{A}=\sin{B}=\sin{C}=\sin{D}=\sin{\alpha}=1$$ és a Cauchy-Buniakovski-Schwarz
egyenlőtlenségben is egyenlő\-ség van. A szögek
alapján következik, hogy a négyszög olyan téglalap,
amelyben az átlók merőlegesek, vagyis négyzet. Ebben az
esetben a Cauchy-Buniakovski-Schwarz egyenlőtlenségben is
teljesül az egyenlőség.

\medskip

\textbf{Megjegyzés:} A két megoldás gyakorlatilag nem sokban
kü\-lön\-bözik, de ha általánosítani próbáljuk a
feladatot a megoldásból kiindulva, akkor különböző
tulajdonságokhoz juthatunk a két megoldás alapján.

\medskip

{\bf 5. feladat: } Adjuk meg az összes olyan háromszöget, amelyben mindhárom
oldal hossza (méterben kifejezve) prímszám és a
terület mérőszáma (négyzet\-méterben) egész szám!

\ki{Mészáros Alpár Richárd}{Kolozsvár}\medskip

\textbf{5. feladat megoldása: } A megoldás során a szokásos jelöléseket használjuk.
A Heron-képlet alapján
\[16T^2=(a+b+c)(-a+b+c)(a-b+c)(a+b-c).\] Ha mindhárom
oldal hossza legalább $3$ lenne, akkor a jobb oldal páratlan
és a bal oldal páros volna. Ez viszont ellentmondás, tehát a
háromszögnek legalább az egyik oldala $2.$ A
háromszög-egyenlőtlenség alapján csak a $2,2,2$ vagy
$2,2,3$ illetve $2,p,p$ oldalhosszakkal rendelkező
háromszögeket kell megvizsgálni, ahol $p$ prímszám. Az
első kettő területe irracionális, míg a harmadik esetben
a $T^2=p^2-1$ egyenlőséghez jutunk, ahonnan $(p-T)(p+T)=1$ és
ez csak $T=0,$ $p=1$ esetén teljesülne, ami nem felel meg. Tehát
nem létezik olyan háromszög, amelynek minden oldala prímszám és
a területe egész szám.

\medskip

\textbf{Megjegyzés:} Igazolható az is, hogy ha egy háromszög minden
oldalának hossza prímszám, akkor a terület irracionális,
pontosabban, ha minden oldal hossza páratlan, akkor a terület
irracionális. A Heron-képlet alapján, ha $T$ racionális,
akkor $T$ irreducibilis alakjában a nevező csak $1,2$ vagy $4$
lehet. Feltételezhet\-jük, hogy mindhárom oldal hossza
legalább $3,$ mivel azoknak a háromszögeknek, amelyeknek
legalább egy oldala $2$ hosszúságú (és a másik kettő
prím), a területét már megvizsgáltuk. Ha a nevező $1$
vagy $2,$ akkor a jobb oldal páratlan és a bal oldal páros,
tehát egyenlőség nem teljesülhet. Ha $T=\frac{u}{4},$ $u\in
\mathbb{N},$ és $u$ páratlan, akkor az
\[u^2=(a+b+c)(-a+b+c)(a-b+c)(a+b-c)=4b^2c^2-(b^2+c^2-a^2)^2\]
egyenlethez jutunk, ahonnan  $$4b^2c^2=u^2+(b^2+c^2-a^2)^2.$$ Ez az
egyenlőség nem teljesülhet, mert két páratlan
négyzetszám összege $8M+2$ alakú és ez nem osztható
$4$-gyel.

\medskip
\eject
{\bf 6. feladat: } Jelölje $a_k$ a $k$ pozitív egész szám legnagyobb páratlan osztóját.
\begin{itemize}
\item[a)] Igazoljuk, hogy

\centerline{ $\displaystyle \left\{ {a_{2^k } ,a_{2^k  + 1}
,\ldots,a_{2^{k + 1} - 1} } \right\} = \left\{ {1,3,5,\ldots,2^{k +
1} - 1} \right\}.$}

\item[b)] Igazoljuk, hogy $\displaystyle \sum\limits_{k = 1}^{2^n  - 1} {a_k }  =
\frac{{4^n  - 1}}{3}.$
\end{itemize}

\ki{Dávid Géza}{Székelyudvarhely}\medskip


\textbf{6. feladat I. megoldása: }  a) Ha minden $k\in \mathbb{N}$ esetén $$A_k=\left\{ {a_{2^k } ,a_{2^k  + 1}
,...,a_{2^{k + 1} - 1} } \right\},$$ akkor $A_0=\{1\}$ és
$A_1=\{a_2,a_3\}=\{1,3\}.$ A bizonyításban matematikai
indukciót használunk, tehát feltételezzük, hogy
$$A_{k-1}=\{1,3,5,\ldots, 2^k-1\}.$$ Világos, hogy ha $k$ páratlan, akkor $a_k=k,$
míg $k=2v$ esetén $a_k=a_v.$ Ez alapján írhatjuk, hogy
$$A_k=\left\{ {a_{2^k } ,a_{2^k  + 1}
,...,a_{2^{k + 1} - 1} } \right\}=$$ $$=\left\{ {a_{2^k } ,a_{2^k
+2} ,...,a_{2^{k + 1} - 2} } \right\}\cup \left\{a_{2^k + 1}
,...,a_{2^{k + 1} - 1} \right\}=$$
$$=\left\{ {a_{2^{k-1} } ,a_{2^{k-1}
+1} ,...,a_{2^{k } - 1} } \right\}\cup \left\{2^k + 1,...,2^{k + 1}
- 1 \right\}=$$
$$=A_{k-1}\cup \left\{2^k + 1,...,2^{k + 1}
- 1 \right\}=$$
$$= \left\{ {1,3,5,...,2^{k + 1}  -
1} \right\}.$$

b) Az $A_{n-1}$ elemeinek összege $$1+3+\ldots +
(2^n-1)=(2^{n-1})^2=4^{n-1},$$ tehát az $S_n= \sum\limits_{k =
1}^{2^n - 1} {a_k }$ sorozat teljesíti az $S_{n+1}=S_n+4^n$
rekurziót. Ez alapján $$S_n=1+4+4^2+\ldots
+4^{n-1}=\frac{4^n-1}{3}.$$

\medskip


\textbf{6. feladat II. megoldása: }  a) Jelölje $H_1$
a bizonyítandó egyenlőség bal oldalán megjelenő
halmazt és $H_2$ a jobb oldalon megjelenő halmazt. Világos,
hogy $H_1\subseteq H_2$ és $H_1$ pontosan akkor tartalmaz
$$(2^{k + 1}  - 1) - (2^k  - 1) = 2^k $$ elemet, ha az $a_{2^k }, a_{2^k  + 1}
,\ldots, a_{2^{k + 1} - 1}$ számok páronként
különböznek. Másrészt, ha az $x>y$ számok legnagyobb
páratlan \linebreak osztója ugyanaz, akkor létezik olyan $v\in
\mathbb{N}^*,$ amelyre $x=2^v\cdot y.$ Ez viszont nem lehetséges,
ha $2^k\leq x,y\leq 2^{k+1}-1,$ mivel $2^k\leq y\leq 2^{k+1}-1 $
esetén $2y\geq 2^{k+1}.$ így a két halmaz egyenlő, mivel
ugyanannyi elemet tartalmaznak és $H_1\subseteq H_2.$

b) $\sum\limits_{k = 1}^{2^n  - 1} {a_k }  = \sum\limits_{k =
0}^{n-1} {\left (\sum\limits_{i = 2^k }^{2^{k+1}  - 1} a_i \right )}
 =\sum\limits_{k = 0}^{n-1} {\left(\sum\limits_{j = 1 }^{2^k} (2j
- 1) \right )}  =$ $$= \sum\limits_{k = 1}^{n} {\left( {2^{k -1} }
\right)} ^2 = \sum\limits_{k = 1}^{n} {4^{k - 1} }  = \frac{{4^n -
1}}{3}.$$

\medskip

\textbf{Megjegyzések: } $2^k$ és $2^{k+1}-1$ közt a számokat
osztályozhatjuk aszerint, hogy $2$-nek milyen hatványával
oszthatók. Ha a $2$ kitevője szerint csökkenő sorrendbe
rendezzük és az azonos kitevő esetén növekvő
sorrendbe, akkor gyakorlatilag a legnagyobb páratlan osztó
szerinti növekvő sorrendet kapjuk. Például $32$-től
$63$-ig a számokat a következőképpen osztályozhatjuk:

$32$-vel oszthatók: $32=32\cdot1,$

$16$-tal oszthatók: $48=16\cdot 3,$

$8$-cal oszthatók: $40=8\cdot5,$ $56=8\cdot 7,$

$4$-gyel oszthatók: $36=4\cdot 9,$ $44=4\cdot11,$ $52=4\cdot13,$
$60=4\cdot 15$

$2$-vel oszthatók: $34=2\cdot 17,$ $38=2\cdot 19,$ $42=2\cdot 21,$
$46=2\cdot 23,$ $50=2\cdot 25,$ $54=2\cdot 27,$ $58=2\cdot 29,$
$62=2\cdot 31$

és végül a $2$-vel nem oszthatók: $$33, 35, 37, 39, 41, 43,
45, 47, ..., 61, 63.$$ Látható, hogy ezáltal $32$-től $63$-ig a
természetes számokat a legnagyobb páratlan osztóik szerinti növekv\H
o sorrendbe rendeztük. 

\smallskip
A $2^k$ és $2^{k+1}-1$ közötti természetes
számok $2$-es számrendszerbeli reprezentációja $(k+1)$
számjegyet tartalmaz, és egy számból a legnagyobb páratlan
osztót úgy kapjuk meg, hogy elhagyjuk a végéről a
$0$-kat. Emiatt nyilvánvaló a $H_1=H_2$ egyenlőség.

\medskip


\end{document}
