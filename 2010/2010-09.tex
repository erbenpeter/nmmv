\documentclass[a4paper,10pt]{article} 
\usepackage[utf8]{inputenc}
\usepackage{t1enc}
\usepackage{graphicx}
\usepackage{amssymb}
\voffset - 20pt
\hoffset - 35pt
\textwidth 450pt
\textheight 650pt 
\frenchspacing 

\pagestyle{empty}
\def\ki#1#2{\hfill {\it #1 (#2)}\medskip}

\def\tg{\, \hbox{tg} \,}
\def\ctg{\, \hbox{ctg} \,}
\def\arctg{\, \hbox{arctg} \,}
\def\arcctg{\, \hbox{arcctg} \,}

\begin{document}
\begin{center} \Large {\em XIX. Nemzetközi Magyar Matematika Verseny} \end{center}
\begin{center} \large{\em Szatmárnémeti, 2010. március 19-22.} \end{center}
\smallskip
\begin{center} \large{\bf 9. osztály} \end{center}
\bigskip 

{\bf 1. feladat: } Mennyi a következő tört értéke?

$$\frac{2010201020102011\cdot4020402040204021-2010201020102010}{2010201020102010\cdot4020402040204021+2010201020102011}$$

\ki{dr. Katz Sándor}{Bonyhád}\medskip

{\bf 2. feladat: } Meseországban fityinggel és fabatkával lehet vásá\-rolni, ahol $1$
fitying $2010$ fabatkát ér. Fajankó és Vasgyúró összehasonlítják
megtakarított pénzüket. Mindketten megszámolják fityingjeiket és
fabatkáikat, majd megállapítják, hogy egyiküknek sincs $2010$-nél
több fityingje, és hogy Vasgyúró va\-gyo\-na $1003$ fabatkával több,
mint Fajankó vagyonának kétszerese. Fajankónak annyi fityingje van,
ahány fabatkája van Vasgyúrónak, és annyi fabatkája, ahány
fityingje van Vasgyúrónak. Mennyi megtakarított pénze van
Fajankónak?

\ki{dr. Péics Hajnalka}{Szabadka}\medskip

{\bf 3. feladat: } Oldjuk meg az
$\displaystyle{\frac{1}{x}-\frac{1}{y}+\frac{1}{z}=\frac{1}{xy}+\frac{1}{yz}-1}$
egyenletet, ha $x,\ y,\ z$ egész számok!

\ki{Bíró Bálint}{Eger}\medskip

{\bf 4. feladat: } A dubai Burj Khalifa felhőkarcoló $160$ emele\-tes.  Induljon el egy
lift a föld\-szint\-ről, és tételezzük fel, hogy útja során minden
emeleten pontosan egyszer áll meg. Mennyi az a leghosszabb út,
amit eközben megtehet, ha két szomszédos emelet közti
távolság $4$ méter?

\ki{Fejér Szabolcs}{Miskolc}\medskip

{\bf 5. feladat: } Az $ABC$ háromszög $AC$ és $BC$ oldalán felvesszük a
$C$-hez, illetve $B$-hez közelebb eső $N$ és $M$
harmadolópontot, valamint az $AB$ oldal $P$ felezőpontját,
majd az eredeti háromszöget kitöröljük. Szer\-kesszük
vissza az $M,N$ és $P$ pont alapján az eredeti háromszöget!

\ki{Kovács Lajos}{Székelyudvarhely}\medskip

{\bf 6. feladat: } Színezzük ki egy szabályos $2010$-szög csú\-csait $3$ színnel,
mindhárom színt ugyanannyiszor használva.  Igaz-e, hogy bár\-mely
színezés esetén lesz olyan szabályos háromszög, amelynek vagy minden
csúcsa azonos színű,  vagy a három csúcsa három különböző
színű?

\ki{Fejér Szabolcs}{Miskolc}\medskip

\vfill
\end{document}
