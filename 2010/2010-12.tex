\documentclass[a4paper,10pt]{article} 
\usepackage[utf8]{inputenc}
\usepackage{t1enc}
\usepackage{graphicx}
\usepackage{amssymb}
\voffset - 20pt
\hoffset - 35pt
\textwidth 450pt
\textheight 650pt 
\frenchspacing 

\pagestyle{empty}
\def\ki#1#2{\hfill {\it #1 (#2)}\medskip}

\def\tg{\, \hbox{tg} \,}
\def\ctg{\, \hbox{ctg} \,}
\def\arctg{\, \hbox{arctg} \,}
\def\arcctg{\, \hbox{arcctg} \,}

\begin{document}
\begin{center} \Large {\em XIX. Nemzetközi Magyar Matematika Verseny} \end{center}
\begin{center} \large{\em Szatmárnémeti, 2010. március 19-22.} \end{center}
\smallskip
\begin{center} \large{\bf 12. osztály} \end{center}
\bigskip 

{\bf 1. feladat: } Határozzuk meg az $E(x)=(1+\cos{x})\sin{x}$ kifejezés legnagyobb
értékét, ha $x$ tetszőleges valós szám! Milyen $x$ esetén veszi ezt
fel?

\ki{Kovács Béla}{Szatmárnémeti}\medskip

{\bf 2. feladat: } Határozzuk meg azt a három különböző
$\displaystyle\frac{n}{n-1},\ (n\in\mathbb{N}, \, n>1)$
alakú törtet, amelyek összege egész szám!

\ki{Nemecskó István}{Budapest}\medskip

{\bf 3. feladat: } Keressük meg azt a leghosszabb, szigorúan növekvő, egészekből álló
mértani sorozatot, amelynek tagjai a $[100,1000]$ intervallumban
vannak!

\ki{Szabó Magda}{Szabadka}\medskip

{\bf 4. feladat: } Az $(x_n)_{n\geq 1}$ valós számsorozat teljesíti az
$$(1+x_n)x_{n+1}=n, n\geq 1$$ rekurziót és $x_1=1.$ Igazoljuk,
hogy $n\geq 2$ esetén
$$\left (\sqrt{\frac{n+1}{2}}-1 \right)^2<1+\frac 1n\sum\limits_{k=1}^n x_k^2<\frac{n^2+n+2}{2n}.$$


\ki{Bencze Mihály}{Brassó}\medskip

{\bf 5. feladat: } Bizonyítsuk be, hogy négy különböző, nemnegatív valós szám közül
kiválasztható  kettő ($x$ és $y$), amelyekre
$$\frac{1+xy}{\sqrt{1+x^2}\cdot\sqrt{1+y^2}}>\frac{\sqrt{3}}{2}.$$

\ki{dr. Minda Mihály}{Vác}\medskip

{\bf 6. feladat: } Egy laktanya udvarán 2010 katona magasság szerint
  áll sorban. Egy perc alatt bármelyik két katona egymással helyet
  cserélhet (tudnak elég gyorsan futni). Egyszerre több helycsere
  is történhet, de egy katona egy perc alatt csak egy helycserében
  vehet részt. Legfeljebb hány perc szükséges ahhoz, hogy névsor
  szerint álljanak sorba? (A katonák különböző magasságúak, és a
  nevük is különbözik.)

\ki{dr. Kántor Sándor}{Debrecen}\medskip

\vfill
\end{document}
