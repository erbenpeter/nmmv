\documentclass[a4paper,10pt]{article} 
\usepackage[utf8]{inputenc}
\usepackage{t1enc}
\usepackage{graphicx}
\usepackage{amssymb}
\voffset - 20pt
\hoffset - 35pt
\textwidth 450pt
\textheight 650pt 
\frenchspacing 

\pagestyle{empty}
\def\ki#1#2{\hfill {\it #1 (#2)}\medskip}

\def\tg{\, \hbox{tg} \,}
\def\ctg{\, \hbox{ctg} \,}
\def\arctg{\, \hbox{arctg} \,}
\def\arcctg{\, \hbox{arcctg} \,}

\begin{document}
\begin{center} \Large {\em XIX. Nemzetközi Magyar Matematika Verseny} \end{center}
\begin{center} \large{\em Szatmárnémeti, 2010. március 19-22.} \end{center}
\smallskip
\begin{center} \large{\bf 11. osztály} \end{center}
\bigskip 

{\bf 1. feladat: } Határozzuk meg az $1,$ $2,$ $4,$ $5,$ $7,$ $9,$ $10,$ $12,$ $14,$
$16,$ $17,$ $19,$ $21,$ $23,$ $25,$ $\ldots$ sorozat $2010.$ tagját,
ha a sorozat tagjait úgy képeztük, hogy az $1$-es után leírtuk az őt
követő $2$ páros számot, majd a kapott számot követő $3$ páratlan
számot, az ezután kapott számot követő $4$ páros számot és így
tovább.

\ki{Pintér Ferenc}{Nagykanizsa}\medskip

{\bf 2. feladat: } Az $x_1,x_2,\ldots, x_n$ nemnegatív valós számok teljesítik
az $$x_1^2+x_2^2+\ldots +x_n^2 + 1\cdot x_1+2\cdot x_2+3\cdot
x_3+\ldots +n\cdot x_n=2010$$ egyenlőséget. Határozzuk meg az
$x_1+x_2+\ldots +x_n$ legkisebb lehetséges értékét!

\ki{Borbély József}{Tata}\medskip

{\bf 3. feladat: } Határozzuk meg a
$$\frac{\sqrt{1-x^2}}{x}=3-4x^2$$ egyenlet valós megoldásait!

\ki{Szilágyi Judit és Nagy Örs}{Kolozsvár}\medskip

{\bf 4. feladat: } Az $ABCD$ konvex négyszögben jelölje $\alpha$ a $d_1$ és $d_2$
hosszúságú $AC,$ illetve $BD$ átló által közrezárt szög
mértékét. Mutassuk ki, hogy $ABCD$ akkor és csakis akkor
négyzet, ha
$$(d_1\sin{A}\sin{C}+d_2\sin{B}\sin{D})\cdot\sin{\alpha}=
\sqrt{2(d_1^2+d_2^2)}.$$

\ki{Longáver Lajos}{Nagybánya}\medskip

{\bf 5. feladat: } Adjuk meg az összes olyan háromszöget, amelyben mindhárom
oldal hossza (méterben kifejezve) prímszám és a
terület mérőszáma (négyzet\-méterben) egész szám!

\ki{Mészáros Alpár Richárd}{Kolozsvár}\medskip

{\bf 6. feladat: } Jelölje $a_k$ a $k$ pozitív egész szám legnagyobb páratlan osztóját.
\begin{itemize}
\item[a)] Igazoljuk, hogy

\centerline{ $\displaystyle \left\{ {a_{2^k } ,a_{2^k  + 1}
,\ldots,a_{2^{k + 1} - 1} } \right\} = \left\{ {1,3,5,\ldots,2^{k +
1} - 1} \right\}.$}

\item[b)] Igazoljuk, hogy $\displaystyle \sum\limits_{k = 1}^{2^n  - 1} {a_k }  =
\frac{{4^n  - 1}}{3}.$
\end{itemize}

\ki{Dávid Géza}{Székelyudvarhely}\medskip

\vfill
\end{document}
