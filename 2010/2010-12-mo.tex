\documentclass[a4paper,10pt]{article} 
\usepackage[utf8]{inputenc}
\usepackage{t1enc}
\usepackage{graphicx}
\usepackage{amssymb}
\voffset - 20pt
\hoffset - 35pt
\textwidth 450pt
\textheight 650pt 
\frenchspacing 

\pagestyle{empty}
\def\ki#1#2{\hfill {\it #1 (#2)}\medskip}

\def\tg{\, \hbox{tg} \,}
\def\ctg{\, \hbox{ctg} \,}
\def\arctg{\, \hbox{arctg} \,}
\def\arcctg{\, \hbox{arcctg} \,}

\begin{document}
\begin{center} \Large {\em XIX. Nemzetközi Magyar Matematika Verseny} \end{center}
\begin{center} \large{\em Szatmárnémeti, 2010. március 19-22.} \end{center}
\smallskip
\begin{center} \large{\bf 12. osztály} \end{center}
\bigskip 

{\bf 1. feladat: } Határozzuk meg az $E(x)=(1+\cos{x})\sin{x}$ kifejezés legnagyobb
értékét, ha $x$ tetszőleges valós szám! Milyen $x$ esetén veszi ezt
fel?

\ki{Kovács Béla}{Szatmárnémeti}\medskip

\textbf{1. feladat I. megoldása: } A $\sin$ és $\cos$ függvény
periodikussága alapján elég\-séges a kifejezés maximumát
a $[0,2\pi]$ intervallumon meghatá\-rozni. Ugyanakkor látható,
hogy ha $x\not \in \left [0,\frac{\pi}{2}\right ],$ akkor a
kifejezés értéke növelhető azáltal, hogy $x$ helyett
$\pi-x$-et vagy $x-\pi$-t vagy $2\pi-x$-et helyettesítünk
(vagyis a változót visszavezetjük az első ne\-gyed\-re).
így feltételezhetjük, hogy $x\in \left [0,\frac{\pi}{2}\right
],$ azaz $t=\cos x\geq 0.$ A kifejezés ebben az esetben
$$E_1(t)=(1+t)\sqrt{1-t^2}$$ alakban írható. A számtani és a
mértani középarányos közti egyenlőtlenség
alapján
$$\sqrt[4]{\left(\frac{1+t}{3}\right )^3(1-t)}\leq \frac{1}{2},\mbox{ tehát }$$
 $$(1+t)\sqrt{1-t^2}\leq \frac{3\sqrt{3}}{4},$$ és egyenl\H
oség pontosan akkor teljesül, ha $\frac{1+t}{3}=1-t,$ vagyis ha
$t=\frac 12.$ Ez mutatja, hogy a kifejezés maximuma
$\frac{3\sqrt{3}}{4}$ és ezt a $\frac{\pi}{3}$-ban (illetve a
$\frac{\pi}{3}+2k\pi,$ $k\in \mathbb{Z}$ pontokban) veszi fel.

\medskip

\textbf{1. feladat II. megoldása: } A $\sin$ és $\cos$ periodikus függvények, főperiódusuk $2\pi$, ezért
elégséges, ha vizsgáljuk a kifejezés értékét a $[0,2\pi]$
intervallumon. A $[\pi,2\pi]$ intervallumon a $\sin{x}$ negatív, az
$1+\cos{x}$ pedig pozitív, ezért a kifejezés értéke is negatív, így
itt nem kapjuk meg a legnagyobb értéket. A $[\frac{\pi}{2},\pi]$
intervallumon $\sin{x}$ értéke po\-zi\-tív, $\cos{x}$ értéke
negatív, $1+\cos{x}$ értéke kisebb, mint $1,$ így a kifejezés
értéke a $[\frac{\pi}{2},\pi]$ intervallumon nem nagyobb, mint
$x=\frac{\pi}{2}$-ben. Emiatt a kifejezés legnagyobb értékét csak a
$[0,\frac{\pi}{2}]$ intervallumon kap\-hat\-juk meg. Itt $\sin{x}$
és $\cos{x}$ értéke is pozitív. Legyen tehát $x\in[0,\frac{\pi}{2}]$
és $t:=\mathrm{tg}\frac{x}{2}$. Ekkor
\[\sin{x}=\frac{2t}{1+t^2} \quad\mathrm{\acute{e}s}\quad \cos{x}=\frac{1-t^2}{1+t^2},\quad \mathrm{ahol} \quad t\in[0,1]. \]
Így az adott kifejezés:
\[\Bigg{(}1+\frac{1-t^2}{1+t^2}\Bigg{)}\cdot\frac{2t}{1+t^2}=\frac{4t}{(1+t^2)^2}.\]

Alkalmazzuk 4 pozitív valós szám számtani és mértani
középará\-nyosai közötti egyenlőtlenséget:
$$1+t^2=\frac{1}{3}+\frac{1}{3}+\frac{1}{3}+t^2\ge4\sqrt[4]{\frac{t^2}{27}}$$
$$(1+t^2)^2\ge16\sqrt{\frac{t^2}{27}}=\frac{16t}{3\sqrt{3}}$$
$$\frac{4t}{(1+t^2)^2}\le\frac{3\sqrt{3}}{4}$$
Egyenlőség $t=\frac{1}{\sqrt{3}}$ esetén áll fenn. Az alkalmazott
helyettesí\-tés kölcsönösen egyértelmű, tehát
$(1+\cos{x})\sin{x}\le\frac{3\sqrt{3}}{4}$, így a kifejezés
legnagyobb értéke $\frac{3\sqrt{3}}{4}$, amit $x=\frac{\pi}{3}$
esetén (vagyis általánosan $x=2k\pi+\frac{\pi}{3},$ $k\in
\mathbb{Z}$ esetén) vesz fel.

\medskip

\textbf{1. feladat III. megoldása: } A $\sin$ és $\cos$ periodikus függvények, fő\-pe\-ri\-ó\-du\-suk
$2\pi$, ezért elég, ha vizsgáljuk a kifejezés értékét a $[0,2\pi]$
intervallumon.

Legyen $f:[0,2\pi]\longrightarrow\mathbb{R}, \quad
f(x)=(1+\cos{x})\sin{x}$. Ekkor
$f'(x)=\cos{2x}+\cos{x}=2\cos{\frac{3x}{2}}\cos{\frac{x}{2}}$. A
Fermat-tétel alapján a függvény szélsőértékeit a
$2\cos{\frac{3x}{2}}\cos{\frac{x}{2}}=0$ egyenlet megoldásai között
ke\-res\-sük. A gyökök: $x_1=\frac{\pi}{3},$ $x_2=\pi$ és
$x_3=\frac{5\pi}{3}$. Mivel $f(\frac{\pi}{3})=\frac{3\sqrt{3}}{4},$
$f(\pi)=0$ és $f(x_3)<0$, így a kifejezés legnagyobb értéke
$\frac{3\sqrt{3}}{4}$, amit $x=\frac{\pi}{3}$ esetén vesz fel.

\medskip

{\bf 2. feladat: } Határozzuk meg azt a három különböző
$\displaystyle\frac{n}{n-1},\ (n\in\mathbb{N}, \, n>1)$
alakú törtet, amelyek összege egész szám!

\ki{Nemecskó István}{Budapest}\medskip

\textbf{2. feladat megoldása: } Legyen a három tört $\frac{x}{x-1},\ \frac{y}{y-1},\ \frac{z}{z-1}$.
Feltehetjük, hogy $x<y<z$.\\
Tudjuk, hogy
$\frac{x}{x-1}+\frac{y}{y-1}+\frac{z}{z-1}-\bigg{(}\frac{1}{x-1}+\frac{1}{y-1}+\frac{1}{z-1}\bigg{)}=3$.
Mivel a három keresett tört összege egész, ezért a zárójelben álló
kifejezés is egész. Továbbá akkor a legnagyobb, ha $x=2,\ y=3,\
z=4$, vagyis $\frac{1}{1}+\frac{1}{2}+\frac{1}{3}=\frac{11}{6}$,
tehát
\[\Bigg{(}\frac{1}{x-1}+\frac{1}{y-1}+\frac{1}{z-1}\Bigg{)}\le\frac{11}{6}.\]

Mivel  egész, akkor csak 1-gyel lehet egyenlő. Így megoldandó az
\[\Bigg{(}\frac{1}{x-1}+\frac{1}{y-1}+\frac{1}{z-1}\Bigg{)}=1\]
egyenlet.

1. eset. Ha $x=2$, akkor $\frac{1}{1}+\frac{1}{y-1}+\frac{1}{z-1}=1$
egyenlőségnek kell teljesülnie, de ez nem lehetséges.

2. eset. Ha $x=3$, akkor
$$\frac{1}{2}+\frac{1}{y-1}+\frac{1}{z-1}=1\Longleftrightarrow
\frac{y-1+z-1}{(y-1)(z-1)}=\frac{1}{2}.$$

Keresztbeszorzás és rendezés után:
$$yz-3y-3z+5=0\Longleftrightarrow
(y-3)(z-3)=4.$$

Mivel $y$ és $z$ egészek és $y<z$, ezért csak  az $y-3=1$ és $z-3=4$
lehetséges. Ebből $y=4$ és $z=7$, amelyek valóban megoldások.

3. eset: ha $x\ge4$, akkor a legnagyobb összeg
$\frac{1}{3}+\frac{1}{4}+\frac{1}{5}=\frac{47}{60}$ lehet. De ez
kisebb, mint 1, tehát nem lehetséges.

Következésképpen a feladatnak csak a $\frac{3}{2},\ \frac{4}{3},\
\frac{7}{6}$ törtek tesznek eleget.

\medskip


{\bf 3. feladat: } Keressük meg azt a leghosszabb, szigorúan növekvő, egészekből álló
mértani sorozatot, amelynek tagjai a $[100,1000]$ intervallumban
vannak!

\ki{Szabó Magda}{Szabadka}\medskip

\textbf{3. feladat megoldása: } Meg kell keresni azt a legnagyobb $n$ természetes számot, amelyre teljesül:
$$100\leq c<cq<cq^2<\dots<cq^{n-1}\leq1000,$$ ahol $c$ természetes szám, a $q$ pedig 1-nél nagyobb racionális szám.
Ha a $q=\frac{a}{b},$ ahol $a$ és $b$ relatív prímek és $a>b$, akkor
belátható, hogy $a=b+1$ esetén lesz a sorozat a leghosszabb, így

\centerline{$100\leq
c<c\left(\frac{b+1}{b}\right)<c\left(\frac{b+1}{b}\right)^2<\ldots<c\left(\frac{b+1}{b}\right)^{n-1}\leq1000,$}

\noindent ahol $ b^{n-1}|c.$ Ha $b=1$, akkor

\centerline{$1000\geq
c\left(\frac{b+1}{b}\right)^{n-1}=c\cdot2^{n-1}\geq100\cdot2^{n-1},$}
ezért $n\leq4.$

Ha $b=2$, akkor $$1000\geq
c\left(\frac{b+1}{b}\right)^{n-1}=c\cdot\left(\frac{3}{2}\right)^{n-1}\geq100\cdot\left(\frac{3}{2}\right)^{n-1},$$
ezért $n\leq6$.

Ha $b\geq3$, akkor $\frac{c}{b^{n-1}}\geq1$, mert $c\geq b^{n-1}$,
így $$1000\geq c\left(\frac{b+1}{b}\right)^{n-1}\geq(b+1)^{n-1}\geq
4^{n-1},$$ ezért $n\leq5$.

Tehát a leghosszabb mértani sorozat $6$ tagú és ez a következő:
$128, 192, 288, 432, 648, 972,$ mert $c=128$ és $q=\frac{3}{2}$.

\medskip

{\bf 4. feladat: } Az $(x_n)_{n\geq 1}$ valós számsorozat teljesíti az
$$(1+x_n)x_{n+1}=n, n\geq 1$$ rekurziót és $x_1=1.$ Igazoljuk,
hogy $n\geq 2$ esetén
$$\left (\sqrt{\frac{n+1}{2}}-1 \right)^2<1+\frac 1n\sum\limits_{k=1}^n x_k^2<\frac{n^2+n+2}{2n}.$$


\ki{Bencze Mihály}{Brassó}\medskip

\textbf{4. feladat megoldása: } A sorozat első néhány tagja: $x_1=1, x_2=\frac
12,$ $x_3=\frac 43,$ $x_4=\frac 97,$ $x_5=\frac 74.$ Vizsgáljuk
meg, hogy mi lenne elégséges a második egyenlőtlenség
induktív bizonyításához. Az indukciós feltevés az
$$n+\sum\limits_{k=1}^n x_k^2<\frac{n^2+n+2}{2}$$ egyenl\H
otlenség lenne és ebből kellene belátni az
$$n+1+x_{n+1}^2+\sum\limits_{k=1}^n x_k^2<\frac{(n+1)^2+(n+1)+2}{2}$$
egyenlőtlenséget. így elégséges volna belátni, hogy
$$1+x_{n+1}^2<\frac{(n+1)^2+(n+1)+2-n^2-n-2}{2}$$ vagyis $x_{n+1}^2<n.$
Másrészt a rekurzió alapján ezt csak akkor egyszerű
igazolni, ha alsó becslésünk is van, pontosabban ha
teljesülne az $x_n>\sqrt{n}-1$ egyenlőtlenség is. $n=1$
esetén ez nem teljesül, de $n=2$-re már igen, emiatt előbb
meg\-pró\-báljuk mate\-matikai indukció segítségével
belátni, hogy
\begin{equation}\label{segedegy}\sqrt{n}-1<x_n<\sqrt{n-1},\quad
n\geq 2.\end{equation} $n\in \{2,3\}$ esetén a kiszámított
értékek alapján látható, hogy az egyenlőtlenségek
teljesülnek. Ugyanakkor ha rögzített $n$ esetén
$$\sqrt{n}-1<x_n<\sqrt{n-1},$$ akkor
$$\frac{n}{1+\sqrt{n-1}}<\frac{n}{1+x_n}<\frac{n}{\sqrt{n}},$$
tehát a felső korlát megvan, és az alsóhoz elégséges
belátni, hogy $$\sqrt{n+1}-1<\frac{n}{1+\sqrt{n-1}}.$$ Ez viszont
igaz, mert
$$\displaystyle\sqrt{n+1}-1=\frac{n}{1+\sqrt{n+1}}<\frac{n}{1+\sqrt{n-1}},$$
tehát a matematikai indukció elve alapján  (\ref{segedegy})
igaz.

Az (\ref{segedegy}) alapján $$k-2\sqrt{k}+1<x_k^2<k-1,\quad k\geq
2,$$ tehát
$$\frac{n(n+1)}{2}-1-2\sum_{k=2}^n \sqrt{k}<\sum\limits_{k=2}^n (k+1-2\sqrt{k})<\sum\limits_{k=2}^n x_k^2<\frac{(n-1)n}{2}.$$
A jobb oldali egyenlőtlenség alapján $$1+\frac 1n
\sum\limits_{k=1}^n x_k^2<1+\frac{n^2-n+2}{2n}=\frac{n^2+n+2}{2n},$$
ami a bizonyítandó egyenlőtlenség jobb oldala. A másik
egyenlőtlenség igazolásához elégséges lenne belátni,
hogy $$\sum\limits_{k=1}^n\sqrt{k}<n\sqrt{\frac{n+1}{2}}$$ és ez
igaz a számtani és négyzetes középarányos közti
egyenlőtlenség alapján (vagy direkt módon is igazolható
például matematikai indukcióval). így

$$\frac{n(n+1)}{2}-2n\sqrt{\frac{n+1}{2}}<\frac{n(n+1)}{2}-2\sum_{k=2}^n \sqrt{k}<\sum\limits_{k=1}^n x_k^2$$
vagyis
$$\frac{n+1}{2}- 2\sqrt{\frac{n+1}{2}} +1<1+\frac 1n \sum\limits_{k=1}^n
x_k^2$$ ami épp a bizonyítandó egyenlőtlenség.

\medskip



{\bf 5. feladat: } Bizonyítsuk be, hogy négy különböző, nemnegatív valós szám közül
kiválasztható  kettő ($x$ és $y$), amelyekre
$$\frac{1+xy}{\sqrt{1+x^2}\cdot\sqrt{1+y^2}}>\frac{\sqrt{3}}{2}.$$

\ki{dr. Minda Mihály}{Vác}\medskip

\textbf{5. feladat I. megoldása: } Minden $a$ valós számhoz kölcsönösen egyértel\-mű\-en
hozzárendelhető az $\overrightarrow{a}(1,a)$ vektor. Így az $x$-hez az $\overrightarrow{x}(1,x)$, az $y$-hoz pedig az $\overrightarrow{y}(1,y)$ vektorok rendelhetők.\\
Legyen a két vektor hajlásszöge $\alpha$. Két vektor skaláris
szorzatának segítségével felírható, hogy
$$\cos{\alpha}=\frac{\overrightarrow{x}\cdot\overrightarrow{y}}{|\overrightarrow{x}|\cdot|\overrightarrow{y}|}=
\frac{1+xy}{\sqrt{1+x^2}\cdot\sqrt{1+y^2}}.$$ Tehát a feladat
állítása ekvivalens a $\cos\alpha> \frac{\sqrt{3}}{2}$
egyenlőtlenséggel, ahol $\alpha$ a két vektor hajásszögének
mértéke. Az $\overrightarrow{a}(1,a)$ ,,típusú'' vektorok
végpontjai az $x=1$ egyenes I. sík\-ne\-gyedben lévő pontjai,
valamint az abszcisszára eső pontja. Osszuk fel az I. síknegyedet 3
darab $O$ középpontú, $30^\circ$-os szögtartományra. A skatulya-elv
értelmében a négy vektorból legalább kettő ugyanabba a
szögtartományba (vagy annak határvonalára) esik. Így a két vektor
hajlásszögére igaz, hogy $\alpha\leq 30^\circ$, azaz
$\cos\alpha\geq\frac{\sqrt{3}}{2}$. Ugyanakkor az $Oy$ tengelyre nem
illeszkedhet a szerkesztett vektorok közül egyik sem, tehát
nem lehetséges az, hogy a négy vektor közül bármely kett\H
o szögének mértéke $\geq 30^{\circ}.$ Emiatt az egyenl\H
otlenség szigorú.

\medskip


\textbf{5. feladat II. megoldása: } Minden $x\geq 0$ valós szám
esetén létezik $\varphi\in \left [0,\frac{\pi}{2}\right)$ úgy,
hogy $x=\tg \varphi.$ Ugyanakkor, ha $x=\tg \varphi$ és $y=\tg
\omega,$ akkor
$$\frac{1+xy}{\sqrt{1+x^2}\cdot\sqrt{1+y^2}}=\frac{1+\tg \varphi \cdot \tg \omega}{\frac{1}{\cos \varphi}\frac{1}{\cos \omega}}=\cos|\varphi-\omega|.$$
így a feladat egyenértékű a következő
állítással:

Ha $\varphi_1,\varphi_2,\varphi_3$ és $\varphi_4$ a $\left
[0,\frac{\pi}{2} \right )$ intervallum elemei, akkor létezik
$i,j\in \{1,2,3,4\}$ úgy, hogy
$\cos|\varphi_i-\varphi_j|>\frac{\sqrt{3}}{2}.$

Ennek igazolása érdekében feltételezhetjük, hogy
$\varphi_1<\varphi_2<\varphi_3<\varphi_4,$ tehát a
$\varphi_{i+1}-\varphi_i$ különbségek közül a legkisebb
biztosan kisebb, mint $\frac{\pi}{6}.$ Ez viszont azt jelenti, hogy

\centerline{$\cos \min \limits_{1\leq i\leq
3}|\varphi_{i+1}-\varphi_i|>\cos \frac{\pi}{6}=\frac{\sqrt{3}}{2}.$}

\medskip

{\bf 6. feladat: } Egy laktanya udvarán 2010 katona magasság szerint
  áll sorban. Egy perc alatt bármelyik két katona egymással helyet
  cserélhet (tudnak elég gyorsan futni). Egyszerre több helycsere
  is történhet, de egy katona egy perc alatt csak egy helycserében
  vehet részt. Legfeljebb hány perc szükséges ahhoz, hogy névsor
  szerint álljanak sorba? (A katonák különböző magasságúak, és a
  nevük is különbözik.)

\ki{dr. Kántor Sándor}{Debrecen}\medskip

\textbf{6. feladat I. megoldása: } Megmutatjuk, hogy 2 perc elegendő ahhoz, hogy névsor szerint álljanak sorba.
Jelölje a katonákat nagyság sze\-rinti sorrendben $A_1, A_2, \ldots
,$ $A_{2010}.$ Készítsük el azt az irányított gráfot, amelynek ezek
a csúcsai, és $A_i$-ből $A_j$-be akkor vezet nyíl (irányított él),
ha az $A_i$ katona névsorban a $j$-edik. Azt mondjuk, hogy az
$A_{i_1}, A_{i_2}, \dots , A_{i_k}$ csúcsok ciklust alkotnak, ha
$A_{i_1}\rightarrow A_{i_2}\rightarrow \dots \rightarrow
A_{i_k}\rightarrow A_{i_1}$ (esetleg $k=1$). Nyilvánvaló, hogy
gráfunk diszjunkt ciklusok uniójára bomlik. Elegendő tehát egy
ciklusra igazolni, hogy az előírt cserék két perc alatt párcserékkel
megvalósíthatók, hiszen a diszjunkt ciklusokban egymástól
függet\-lenül, egy\-szerre történhetnek a cserék.

Az \ $A_{i_1}\rightarrow A_{i_2}\rightarrow \dots \rightarrow
A_{i_k}\rightarrow A_{i_1}$ \ ciklusban az első percben az $m$-edik
$\left (m = 1, 2, \dots , [\frac{k-1}{2}]\right )$ helyen álló
$A_{i_m}$ katona cseréljen helyet a $(k-m)$-edik helyen állóval, a
második percben pedig az $m$-edik $\left (m = 1, 2, \dots ,
[\frac{k}{2}]\right )$ helyen álló katona cseréljen helyet a
$(k-m+1)$-edik helyen állóval. A két csere végeredményeként az
eredetileg az $m$-edik helyen álló $A_{i_m}$ katona a
$k-(k-m)+1=(m+1)$-edik helyre kerül, az $A_{i_k}$ pedig az első
helyre, $A_{i_1}$ helyére. Tehát megvalósult a kívánt csere.
Háromtagú ciklusnál a csere egy lépésben nem valósítható meg, tehát
legalább két perc szükséges a cseréhez.

\medskip


\textbf{6. feladat II. megoldása: } A katonákat számozzuk meg a
nagyság sze\-rinti sorrendnek megfelelően $1$-től $2010$-ig
és jelöljük $\sigma(i)$-vel a névsor szerinti sorrendben
elfoglalt helyének sorszámát. így egy $\sigma$
permutációt értelmeztünk, amely diszjunk ciklusok
szorzatára bomlik, tehát elégséges a cseréket ciklusokon
belül elvégezni. ábrázoljuk a ciklus elemeit egy szabályos
sokszög csúcsaiban (a ciklusban elfoglalt sorrendnek megfelel\H
oen, trigonometrikus irányban) és számozzuk meg a sokszög
csúcsait ugyanazokkal a számokkal.

\centerline{\includegraphics[width=0.4\textwidth]{permut1.eps}}

így kezdetben minden csúcsnak a száma talál a csúcsba
írt számmal. A cserék végrehajtása után az $i$ csúcsra
a $\sigma(i)$ szám kerül, tehát gyakorlatilag a cserékkel a
csúcsokba írt számoknak a középpont körüli
$\frac{2\pi}{k}$ szögű elforgatását kell elérni, ahol $k$
a ciklus hossza. Ez viszont két tengelyes tükrözés
összetételével is elérhető. A tükrözések
leírásából az első megoldásban leírt cseréket
kapjuk.

\medskip

\textbf{Megjegyzés: } A mellékelt ábrán a $\sigma=\left (\begin{array}{ccccc} 1&2&3&4&5\\ 3&5&2&1&4 \end{array} \right )$
ciklus esetén szemléltettük a cseréket. Látható, hogy a
permutációnak transzpozíciókra való felbontását kell
létrehozni. Az ábrának megfelelő felbontás:

\centerline{$\sigma=(1,2)\cdot(4,5)\cdot(1,3)\cdot(2,4).$}

\end{document}
