\documentclass[a4paper,10pt]{article} 
\usepackage[utf8]{inputenc}
\usepackage[a4paper]{geometry}
\usepackage[magyar]{babel}
\usepackage{amsmath}
\usepackage{amssymb}
\frenchspacing 
\pagestyle{empty}
\newcommand{\ki}[2]{\hfill {\it #1 (#2)}\medskip}
\newcommand{\vonal}{\hbox to \hsize{\hskip2truecm\hrulefill\hskip2truecm}}
\newcommand{\degre}{\ensuremath{^\circ}}
\newcommand{\tg}{\mathop{\mathrm{tg}}\nolimits}
\newcommand{\ctg}{\mathop{\mathrm{ctg}}\nolimits}
\newcommand{\arc}{\mathop{\mathrm{arc}}\nolimits}
\begin{document}
\begin{center} \Large {\em XX. Nemzetközi Magyar Matematika Verseny} \end{center}
\begin{center} \large{\em Bonyhád, 2011. március 11--15.} \end{center}
\smallskip
\begin{center} \large{\bf 10. osztály} \end{center}
\bigskip 

{\bf 1. feladat: }
Legyen egy háromszög három oldalának a hossza $a$, $b$ és $c$. Bizonyítsuk be, hogy
\[3 \le \frac{(a+b+c)^2}{ab+bc+ca} \le 4\]
Mikor állhat fenn egyenlőség?

\ki{Kántor Sándorné}{Debrecen}\medskip

{\bf 2. feladat: } 
Oldjuk meg a következő egyenletrendszert a valós számok halmazán!
\begin{equation*}
\left.
\begin{aligned}
4x^2-3y&= xy^3 \\
x^2+x^3y^2&= 2y \\
\end{aligned}
\right\}
\end{equation*}

\ki{Balázsi Borbála}{Beregszász}\medskip

{\bf 3. feladat: } 
Az $AB$ szakaszon vegyük fel a $C$ és $D$ pontokat úgy, hogy $AC=CD=DB$ legyen és legyen $CDEF$ egy tetszőleges  paralelogramma. Legyen $G$ az $AE$ és $DF$, $H$ pedig a $BF$ és $CE$ metszéspontja. Bizonyítsuk be, hogy $AB=9GH$.

\ki{Olosz Ferenc}{Szatmárnémeti}\medskip

{\bf 4. feladat: } 
Egy $2n$ oldalú, szimmetria középponttal rendelkező konvex sokszöglap ($\mathcal{P}$) csúcspontjai közül kiválasztunk hármat, jelöljük őket $A$-val, $B$-vel és $C$-vel. Igazoljuk, hogy az $ABC$ háromszög $t$ területe nem nagyobb, mint  $\frac{T}{2}$, ahol $T$ a $\mathcal{P}$ sokszöglap területét jelöli.

\ki{Dálya Pál Péter}{Szeged}\medskip

{\bf 5. feladat: } 
Hány olyan egyenlőszárú trapéz létezik, amelynek a kerülete 2011 és az oldalak mérőszáma egész szám?

\ki{Szabó Magda}{Szabadka}\medskip

{\bf 6. feladat: } 
Adott nyolc különböző pozitív egész szám a tízes számrendszerben. Képezzük bármely kettő (pozitív) különbségét, majd az így kapott 28 számot szorozzuk össze. 6-nak melyik az a legnagyobb kitevőjű hatványa, amivel ez a szorzat biztosan osztható?

\ki{Kiss Sándor}{Nyíregyháza}
\end{document}